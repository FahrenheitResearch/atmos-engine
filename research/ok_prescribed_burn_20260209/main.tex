\documentclass[11pt, letterpaper]{article}

% --- Encoding and fonts ---
\usepackage[utf8]{inputenc}
\usepackage[T1]{fontenc}
\usepackage{lmodern}

% --- Page layout ---
\usepackage[margin=1in]{geometry}
\usepackage{setspace}
\onehalfspacing

% --- Graphics and color ---
\usepackage{graphicx}
\usepackage[dvipsnames]{xcolor}
\usepackage{subcaption}

% --- Tables ---
\usepackage{booktabs}
\usepackage{array}
\usepackage{tabularx}
\usepackage{adjustbox}

% --- Headers and footers ---
\usepackage{fancyhdr}
\pagestyle{fancy}
\fancyhf{}
\lhead{\small wxsection.com Atmospheric Research Platform}
\rhead{\small Oklahoma Prescribed Burns During Red Flag Conditions: A Mult}
\cfoot{\thepage}
\renewcommand{\headrulewidth}{0.4pt}

% --- Alert/callout boxes ---
\usepackage{tcolorbox}

% --- Hyperlinks ---
\usepackage[colorlinks=true, linkcolor=blue!70!black, urlcolor=blue!70!black, citecolor=blue!70!black]{hyperref}

% --- Misc ---
\usepackage{enumitem}
\usepackage{parskip}

% --- Title configuration ---
\newcommand{\reporttypecolor}{orange!80!black}
\newcommand{\reporttypelabel}{BULLETIN}

\title{\textcolor{\reporttypecolor}{\large\textsc{\reporttypelabel}\\[0.3em]\LARGE\textbf{Oklahoma Prescribed Burns During Red Flag Conditions: A Multi-Agent Investigation}}}
\author{5-Agent AI Research Swarm (wxsection.com)}
\date{February 9--10, 2026}

\begin{document}

\maketitle
\thispagestyle{fancy}

\begin{abstract}
On February 9, 2026, prescribed burn activity was reported across portions of western Oklahoma while Red Flag Warnings were in effect for Harper, Ellis, Woodward, and Roger Mills counties. This investigation deployed five autonomous AI research agents --- each using a distinct analytical domain from the wxsection.com platform --- to determine whether a data-driven justification exists for conducting prescribed burns on this day. The agents analyzed HRRR cross-sections (22 multi-panel images), frontal passage timing and overnight recovery, terrain complexity and fuel conditions, 625 METAR observations with model verification, and national fire risk context including SPC/NWS products. All five agents independently concluded that conditions on February 9 did not support prescribed burning, though the terrain and seasonal context make western Oklahoma appropriate for prescribed burns under normal February conditions.
\end{abstract}

\section{Synoptic Overview}

A strong cold front advanced across the southern High Plains during the afternoon of February 9, 2026. Ahead of the front, west-southwesterly winds of 20--25 mph (gusting to 40 mph) combined with relative humidity values of 10--15\% to produce critical fire weather conditions across the Oklahoma and Texas Panhandles.

The SPC Day 1 Fire Weather Outlook designated \textbf{CRITICAL} fire weather for NE New Mexico into the OK/TX Panhandles and \textbf{ELEVATED} conditions across the broader western Oklahoma corridor. The NWS Norman office issued Red Flag Warnings for Harper, Ellis, Woodward, and Roger Mills counties through 7 PM CST. NWS forecaster Ungar explicitly stated: ``burning is discouraged in and near the warning area.''

This investigation asks: despite these warnings, could prescribed burn operators have identified a defensible window based on temporal trends, frontal recovery, terrain advantages, or seasonal necessity?

\section{The Case for Prescribed Burns}

Before examining the atmospheric data, it is essential to understand the operational constraints facing prescribed burn managers in the southern Great Plains.

\subsection*{Seasonal Imperative}
February is the primary prescribed burn window for Oklahoma's grasslands. Native tallgrass and mixed-grass prairies must be burned during dormancy (December--March) before spring green-up renders burns ineffective. December and January often have snow cover or frozen ground; March can see early green-up in southern Oklahoma. February is frequently the only viable month.

\subsection*{Terrain Suitability}
All three investigation sites (Woodward, Gage, Elk City) were classified as \textbf{Rolling Hills} terrain with no canyon features, no steep segments, and maximum slopes under 5\textdegree. Average slopes were under 1\textdegree. This is textbook prescribed burn terrain --- flat, predictable, accessible to mechanized suppression equipment. The E-W transect showed only 0.10\% average grade over 253 miles.

\subsection*{Fuel Management Urgency}
Fuel assessments revealed \textbf{critically low fuel moisture} (2--8\%) across all sites, with zero precipitation for 8+ days at Gage. Paradoxically, the same drought conditions that make burns more dangerous also make them more necessary --- accumulated dead grass creates continuous fuel beds that increase wildfire risk. Prescribed burning is the primary tool for reducing this accumulated fuel load.

\subsection*{National Context}
Western Oklahoma ranked \textbf{7th of 8 regions} in the national fire weather scan (score 38/100, MODERATE). NE New Mexico was the worst area nationally (score 70, ELEVATED, RH as low as 6.4\%). The OK/TX Panhandles scored 52--56. Oklahoma's burn area was on the lower end of the fire weather spectrum --- still under Red Flag conditions, but not the worst location in the country.

\section{Atmospheric Cross-Section Analysis}

The Atmospheric Agent generated 22 multi-panel cross-section images and extracted surface statistics at 135 points (E-W) and 92 points (N-S) per forecast hour. Two transects were analyzed: an E-W transect across central Oklahoma (35.5\textdegree N, 99.5\textdegree W to 95.0\textdegree W) and a N-S transect through the Red Flag Warning zone (37.0\textdegree N to 34.5\textdegree N at 99.0\textdegree W).

\subsection*{Relative Humidity}
RH bottomed at \textbf{12.9\%} on the E-W transect at 3 PM CST and \textbf{13.0\%} on the N-S transect at 5 PM. Mean RH across the entire RFW zone dropped to 16.6\% by 5 PM. At no point during the 1--6 PM window did any location along the western OK transect exceed 30\% RH after 2 PM.

\subsection*{Dewpoint Depression}
Dewpoint depressions of \textbf{24--31\textdegree C} confirmed extremely dry air through the full atmospheric column --- not a surface artifact but a deep, well-mixed boundary layer phenomenon extending to $\sim$700 hPa ($\sim$3 km AGL).

\subsection*{Temperature}
Surface temperatures peaked at \textbf{78.6\textdegree F (25.9\textdegree C)} on the N-S transect --- 15--20\textdegree F above normal for early February. Combined with low RH, this produced extreme vapor pressure deficit (VPD) of 5.0+ kPa.

\subsection*{Boundary Layer Structure}
Potential temperature cross-sections revealed a deep, well-mixed boundary layer extending from the surface to $\sim$700 hPa by 3 PM. Steep (near-superadiabatic) lapse rates efficiently transported momentum from aloft to the surface and maintained extreme dryness through the full BL depth.

\begin{figure}[htbp]
\centering
\includegraphics[width=1.0\textwidth]{figures/ew_temporal_rh.png}
\caption{E-W transect: RH evolution from 1 PM to 8 PM CST (FHR 1/3/5/8). RH collapses to 12.9\% minimum at 3 PM with partial recovery after 7 PM.}
\end{figure}

\begin{figure}[htbp]
\centering
\includegraphics[width=1.0\textwidth]{figures/ns_temporal_rh.png}
\caption{N-S transect through Red Flag Warning zone: RH evolution. Mean RH drops to 16.6\% by 5 PM -- the entire western OK corridor was below critical thresholds.}
\end{figure}

\begin{figure}[htbp]
\centering
\includegraphics[width=1.0\textwidth]{figures/ew_temporal_wind_speed.png}
\caption{E-W transect: wind speed evolution showing sustained boundary layer winds through the afternoon heating cycle.}
\end{figure}

\begin{figure}[htbp]
\centering
\includegraphics[width=1.0\textwidth]{figures/ew_products_fhr3.png}
\caption{Three-panel comparison at 3 PM CST (peak conditions): wind speed, RH, and VPD along the E-W transect. Note co-location of strong winds and extreme dryness.}
\end{figure}

\begin{figure}[htbp]
\centering
\includegraphics[width=1.0\textwidth]{figures/ew_temporal_dewpoint_dep.png}
\caption{E-W dewpoint depression evolution. Values of 24-31 deg C confirm extreme dryness through the full atmospheric column, not just at the surface.}
\end{figure}

\begin{figure}[htbp]
\centering
\includegraphics[width=1.0\textwidth]{figures/ew_temporal_theta.png}
\caption{Potential temperature (theta) evolution showing the deep, well-mixed boundary layer extending to approximately 700 hPa. This mixing transports dry, windy air to the surface.}
\end{figure}

\section{Temporal Window Analysis}

The Frontal \& Temporal Agent investigated whether the approaching cold front could have provided a justifiable burn window --- the strongest potential argument for prescribed burning on this day.

\subsection*{Cold Front Timing}
The cold front passed Woodward/Gage at approximately \textbf{5:00 AM CST on February 10} (FHR 17) --- a full \textbf{16--18 hours} after any plausible morning burn ignition on February 9. This eliminates any argument that ``the front would bring relief by evening.''

\subsection*{Post-Frontal Conditions}
Critically, the front did NOT suppress fire weather. Post-frontal conditions:
\begin{itemize}
  \item \textbf{Woodward:} Wind reversal WSW$\rightarrow$N (97\textdegree\ shift), sustained \textbf{30.7 kt}, gusts \textbf{43 kt}. RH rose to only 33\%.
  \item \textbf{Gage:} Two-phase shift, sustained \textbf{32.8 kt}, gusts \textbf{46 kt}. RH rose to 38\%.
  \item \textbf{Elk City:} Sustained \textbf{31.0 kt}, gusts \textbf{43 kt}. RH \textbf{dropped to 16\%} at frontal passage --- zero humidity relief.
\end{itemize}

\subsection*{Overnight Recovery Classification}
\begin{itemize}
  \item \textbf{Woodward:} \texttt{frontal\_shift} --- NOT recovery. Winds 16--31 kt overnight. All firelines change direction.
  \item \textbf{Gage:} \texttt{partial\_recovery} --- RH improved to 53\% but winds remained 15+ kt. Fires moderate but don't go out.
  \item \textbf{Elk City:} \texttt{no\_recovery} --- Winds 30--34 kt, RH 16--24\% all night. Fires continue to run.
\end{itemize}

\subsection*{The ``Morning Burn'' Argument}
Could burns have been safely conducted in the morning before peak heating? The HRRR data shows conditions were already deteriorating by noon CST (model initialization). At 1 PM (F01), RH was 22--30\% across western OK --- already below the 25--30\% threshold used by most prescribed burn guidelines. Any fire ignited in the morning would still be active during the 3--5 PM peak. The \textbf{17 consecutive hours} of critically low humidity (noon Feb 9 through 5 AM Feb 10) left no safe temporal window.

\section{Terrain and Fuel Assessment}

\subsection*{Terrain Complexity}
All three sites were classified as \textbf{Rolling Hills} with no canyons, no steep segments, and maximum slopes of 3--5\textdegree:

\begin{tabular}{lrrrl}
\hline
\textbf{Site} & \textbf{Elevation (ft)} & \textbf{Relief (ft)} & \textbf{Max Slope} & \textbf{Canyons} \\
\hline
Woodward & 1,942 & 502 & 3.6\textdegree & None \\
Gage & 2,172 & 390 & 3.2\textdegree & None \\
Elk City & 1,929 & 512 & 5.0\textdegree & None \\
\hline
\end{tabular}

Fire behavior is entirely wind-dominated with no topographic acceleration effects. However, the flat terrain also means \textbf{no natural firebreaks} --- no rivers, canyons, or ridgelines to stop an escaped burn.

\subsection*{Fuel Conditions}
\begin{tabular}{lrrl}
\hline
\textbf{Site} & \textbf{Fuel Moisture} & \textbf{7-Day Precip} & \textbf{Min RH (Feb 9)} \\
\hline
Woodward & 5--8\% (Very Low) & 0.02 in (trace) & 22.6\% \\
Gage & 2--5\% (Critical) & 0.00 in (zero) & 10.2\% \\
Elk City & 2--5\% (Critical) & 0.00 in (zero) & 16.0\% \\
\hline
\end{tabular}

Fuel moisture was \textbf{well below the 8\% threshold for extreme fire behavior}. At 2--5\%, flame lengths in cured grass could exceed 10 feet with rate of spread exceeding 300 chains/hour (3.4 mph) under the observed winds.

\subsection*{The Central Paradox}
The terrain supports prescribed burning as a general management practice --- flat, predictable, accessible. But the conditions on February 9 were the \textit{most dangerous possible} within an otherwise valid burn season. The terrain's lack of natural firebreaks means containment failure is catastrophic rather than merely problematic.

\begin{figure}[htbp]
\centering
\includegraphics[width=1.0\textwidth]{figures/ns_products_fhr3.png}
\caption{N-S transect through Red Flag Warning zone at 3 PM CST: wind speed, RH, and VPD. The entire transect is below 30\% RH with minimum 13.8\%.}
\end{figure}

\begin{figure}[htbp]
\centering
\includegraphics[width=1.0\textwidth]{figures/ns_temporal_wind_speed.png}
\caption{N-S transect wind speed temporal evolution showing increasing boundary layer winds through peak afternoon heating in the Red Flag Warning counties.}
\end{figure}

\section{Surface Observations and Verification}

The Observations Agent parsed \textbf{625 METAR observations} across 7 Oklahoma stations and verified wind claims, model accuracy, and fire weather climatology.

\subsection*{Peak Observed Conditions}
\begin{itemize}
  \item \textbf{KGAG (Gage):} Peak gust \textbf{37 kt (43 mph)} at 15:58 CST. Sustained \textbf{27 kt (31 mph)} at 14:20 CST. RFW criteria met for \textbf{3 continuous hours} (14:20--17:20 CST) with 24 individual METAR observations meeting the threshold.
  \item \textbf{KWDG (Woodward):} Sustained 13 kt. Moderate but under RFW.
  \item \textbf{KOKC (OKC):} 21\% RH, 20 mph winds --- marginal, well below criteria.
  \item \textbf{KTUL (Tulsa):} 24\% RH, 16 mph --- below criteria.
\end{itemize}

\subsection*{Model Verification}
HRRR 18Z F03 was \textbf{remarkably accurate} at 21Z validation: temperature within 1\textdegree F, winds within 1 kt, RH within 2\%. The event was NOT overforecast --- actual conditions matched model predictions.

\subsection*{Spatial Gradient}
A sharp west-to-east gradient existed: KGAG had 10.2\% RH and 43 mph gusts while OKC had 21\% RH and 20 mph winds. The worst conditions were focused precisely on the Gage-Woodward corridor where burns were conducted.

\subsection*{Fire Weather Climatology}
February climatology for KGAG shows the observed conditions were a ``once-or-twice per year'' event. KOKC's observed 80.6\textdegree F was \textbf{27.6\textdegree F above the February normal} of 53\textdegree F.

\section{Fire Risk Quantification}

The Risk Agent ran quantitative fire risk analysis along both transects at four forecast hours.

\subsection*{Risk Score Evolution (E-W Transect)}
\begin{tabular}{llrrl}
\hline
\textbf{FHR} & \textbf{Time} & \textbf{Score} & \textbf{Level} & \textbf{RH Min} \\
\hline
F01 & 1 PM & 14 & LOW & 21.7\% \\
F03 & 3 PM & \textbf{40} & \textbf{MODERATE} & \textbf{12.9\%} \\
F05 & 5 PM & 17 & LOW & 17.0\% \\
F08 & 8 PM & 13 & LOW & 16.7\% \\
\hline
\end{tabular}

Risk peaked sharply at 3 PM (score 40, MODERATE) then declined. At 1 PM, the transect-averaged score was only 14 (LOW). \textbf{Important caveat:} these scores use column-averaged cross-section wind data, which dilutes surface wind speeds. NWS-reported surface gusts of 35--43 mph are not fully captured. The actual surface fire risk was likely \textbf{higher than these scores indicate}.

\subsection*{NWS Products}
\begin{itemize}
  \item SPC designated \textbf{CRITICAL} fire weather for NE NM into the OK/TX Panhandles
  \item \textbf{15 active NWS alerts} for Oklahoma including multiple Red Flag Warnings
  \item NWS Norman explicitly stated: ``\textit{Outdoor burning is not recommended}''
  \item NWS Norman forecaster: ``\textit{burning is discouraged in and near the warning area}''
  \item SPC Day 2 outlook: ``NO CRITICAL AREAS'' --- confirming this was a single-day event
\end{itemize}

\section{Balanced Assessment}

\subsection*{Arguments Supporting Burn Operations}
\begin{enumerate}
  \item \textbf{Seasonal imperative:} February is genuinely the narrowest prescribed burn window for Oklahoma's dormant grasslands. Missing it risks no burns until next year.
  \item \textbf{Terrain suitability:} Western OK's flat terrain (max slope 5\textdegree, no canyons) is ideal prescribed burn landscape under normal conditions.
  \item \textbf{Fuel management urgency:} Critically low fuel moisture (2--8\%) and zero recent precipitation indicate dangerous fuel accumulation. Prescribed burning reduces wildfire risk for the upcoming fire season.
  \item \textbf{Not the worst nationally:} Oklahoma ranked 7th of 8 regions in the national fire scan. NE New Mexico, the TX/OK Panhandles, and SE Wyoming all had worse conditions.
  \item \textbf{Morning conditions less severe:} At 1 PM, risk scores were LOW (14/100) and RH was still 22--30\%. Earlier morning hours would have been more favorable.
\end{enumerate}

\subsection*{Arguments Against Burn Operations}
\begin{enumerate}
  \item \textbf{Red Flag Warnings active:} NWS explicitly recommended against outdoor burning. These warnings represent the meteorological community's consensus assessment.
  \item \textbf{No safe temporal window:} RH was below 25\% across western OK by 1 PM and below 15\% by 3 PM. Any morning burn would still be active during peak conditions. 17 consecutive hours of critically low humidity left no window.
  \item \textbf{Front brings no relief:} The cold front arrived 16--18 hours after any morning ignition and brought \textit{increased} winds (30--46 kt gusts) with a violent wind reversal. At Elk City, RH actually \textit{dropped} at frontal passage.
  \item \textbf{No natural firebreaks:} Zero canyons, rivers, or ridgelines to stop an escaped burn. Miles of continuous dormant grass in every direction.
  \item \textbf{Model was accurate:} HRRR verification within 1\textdegree F / 1 kt / 2\% RH. The dangerous conditions were well-forecast and predictable.
  \item \textbf{Escape consequences catastrophic:} With 2--5\% fuel moisture and 43 mph gusts, an escaped burn would face rate of spread exceeding 300 chains/hour.
\end{enumerate}

\section{Conclusions and Recommendations}

Five autonomous AI research agents independently analyzed this event using 12 analytical modules, 40+ tool functions, 22 cross-section images, 625 surface observations, and national context data. \textbf{All five agents concluded that prescribed burns were not justifiable on February 9, 2026.}

\textbf{Key Findings:}
\begin{enumerate}
  \item Atmospheric conditions met or exceeded critical fire weather thresholds across western Oklahoma for 17 consecutive hours, confirmed by HRRR model, GFS model, METARs, SPC outlooks, and NWS warnings.
  \item The approaching cold front --- the strongest potential justification for ``burn now, recovery later'' --- provided no meaningful relief. Post-frontal winds actually \textit{increased} to 30--46 kt with a dangerous direction reversal.
  \item While terrain and seasonal context genuinely support prescribed burning in western Oklahoma during the January--March window, the specific conditions on February 9 were the most dangerous possible within that window.
  \item Oklahoma ranked 7th of 8 nationally --- not the worst, but still firmly within Red Flag Warning criteria with explicit NWS guidance against burning.
\end{enumerate}

\textbf{Recommendations:}
\begin{enumerate}
  \item Prescribed burns should not be conducted under active Red Flag Warnings, regardless of seasonal pressure.
  \item Oklahoma should require mandatory pre-burn weather briefings with documented NWS product review.
  \item Post-frontal windows (February 10 had ``NO CRITICAL AREAS'' per SPC) often provide the ideal burn conditions that February 9 lacked.
  \item Burn managers should evaluate frontal recovery quality, not just timing --- a front that maintains 30+ kt winds is not ``recovery.''
  \item Tools like wxsection.com cross-sections can help burn managers assess vertical atmospheric structure (boundary layer depth, momentum transport) beyond surface obs.
\end{enumerate}

\vspace{1em}
\begin{center}
\small\textit{Generated by 5 parallel AI agents using all 12 wxsection.com analytical modules. Total agent compute: 270K tokens across 122 tool invocations.}
\end{center}

\vfill
\begin{center}
\small\textit{Generated by wxsection.com AI Atmospheric Research Agent}
\end{center}

\end{document}