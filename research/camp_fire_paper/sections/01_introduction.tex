\section{Introduction}
\label{sec:introduction}

On the morning of 8 November 2018, a catastrophic wildfire ignited near the community of Pulga in Butte County, California, within the steep terrain of the Feather River Canyon. Driven by powerful offshore winds, the fire advanced westward at an extraordinary rate, overrunning the town of Paradise --- population approximately 26,800 --- in fewer than four hours. The Camp Fire, as it came to be named, destroyed 18,804 structures, burned 62,053 hectares (153,336 acres), and claimed 85 lives, making it the deadliest and most destructive wildfire in California's recorded history \citep{CAL_FIRE_2019}. The rapidity of the disaster, with an entire community effectively consumed between 0630 and 1030 Pacific Standard Time (PST), underscored both the extremity of the atmospheric environment and the limitations of existing evacuation infrastructure in the wildland--urban interface (WUI).

The atmospheric pattern responsible for the Camp Fire belongs to a well-documented class of offshore wind events in northern California, variously termed ``Diablo winds'' north of the San Francisco Bay Area and ``North winds'' in the Sacramento Valley and northern Sierra Nevada foothills \citep{Abatzoglou_etal_2013, Smith_etal_2018, Mass_Ovens_2019}. These events arise when synoptic-scale pressure gradients force air from the interior Great Basin westward across the Sierra Nevada and Cascade ranges. As air descends the western slopes, it undergoes adiabatic compression, producing characteristically warm, dry, and gusty conditions at the surface --- the classical foehn mechanism \citep{Brinkmann_1974}. The resulting fire weather environment is analogous to, and in some respects more severe than, the Santa Ana wind regime of southern California \citep{Raphael_2003, Hughes_Hall_2010, Guzman-Morales_etal_2016}, owing to the steeper and more channelized terrain of the northern Sierra foothills.

Downslope windstorms in mountainous terrain have been extensively studied in the context of the Rocky Mountains \citep{Durran_1990, Doyle_etal_2000}, the Alps \citep{Smith_1987}, and other major barriers, with the physical mechanisms involving mountain wave amplification, critical-level absorption, and hydraulic jump dynamics. In the specific context of California fire weather, prior work has documented the synoptic climatology of offshore events \citep{Abatzoglou_etal_2013, Guzman-Morales_Abatzoglou_2018}, the role of upper-level troughs in establishing cross-barrier pressure gradients \citep{Hughes_Hall_2010, Abatzoglou_etal_2021}, and the terrain channeling effects that amplify surface wind speeds in narrow canyons \citep{Sharples_etal_2012, Brewer_Clements_2020}. Several post-event analyses of the Camp Fire have examined its behavior from a fire science perspective \citep{Maranghides_etal_2021}, but a comprehensive three-dimensional reconstruction of the atmospheric environment using high-resolution model data has not been presented.

The High-Resolution Rapid Refresh (HRRR) model, operated by the National Oceanic and Atmospheric Administration (NOAA), provides a uniquely suitable dataset for this analysis. With 3-km horizontal grid spacing, hourly cycling, and 40 vertical pressure levels from 1013 to 50 hPa, the HRRR resolves mesoscale terrain interactions --- including canyon channeling, mountain wave dynamics, and slope flow acceleration --- that are poorly represented in coarser global models \citep{Benjamin_etal_2016, Dowell_etal_2022}. The HRRR's rapid update cycle, assimilating radar, satellite, surface, and radiosonde observations every hour, further ensures that the model initial conditions closely reflect the observed atmosphere at the time of fire ignition.

This study uses HRRR analysis and short-range forecast fields from the 0000 UTC 8 November 2018 cycle to reconstruct the three-dimensional atmospheric environment of the Camp Fire. Our analysis is based on vertical cross-sections extracted along strategically oriented transects that intersect the fire origin area, the Feather River Canyon, the town of Paradise, and the surrounding synoptic-scale environment. These cross-sections are generated using the wxsection.com platform, which provides programmatic access to archived HRRR fields through a cross-section rendering and data extraction API.

The objectives of this study are threefold:

\begin{enumerate}
    \item To document the synoptic-scale upper-level forcing that established the cross-barrier flow regime, including the position and intensity of the potential vorticity (PV) anomaly and the associated dynamic tropopause fold.
    \item To quantify the mesoscale wind, temperature, humidity, and vertical motion fields along and across the fire's path, with particular attention to the low-level jet structure within the Feather River Canyon and the extreme dryness of the descending air mass.
    \item To assess the multi-parameter extremity of the fire weather environment --- the simultaneous co-occurrence of strong winds, very low relative humidity, persistent subsidence, and terrain channeling --- that transformed a single ignition into an urban-scale disaster within hours.
\end{enumerate}

The remainder of this paper is organized as follows. Section~\ref{sec:synoptic_overview} presents the synoptic-scale overview, including upper-level PV structure, the 850-hPa temperature and wind fields, and the broad east--west subsidence pattern. Section~\ref{sec:wind} details the low-level jet structure and canyon wind channeling along the fire path. Section~\ref{sec:moisture} examines the extraordinary humidity deficit from the surface through the mid-troposphere. Section~\ref{sec:thermodynamics} analyzes the thermodynamic structure, including lapse rates, inversions, and the subsidence warming signature. Section~\ref{sec:vertical_motion} addresses the vertical motion field. Section~\ref{sec:fire_weather} synthesizes these findings in the context of fire behavior and discusses the implications for WUI hazard assessment. Section~\ref{sec:conclusions} provides concluding remarks.
