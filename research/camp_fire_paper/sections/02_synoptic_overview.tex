\section{Synoptic Overview}
\label{sec:synoptic_overview}

The atmospheric environment of the Camp Fire was governed by a deep upper-level trough over the eastern Pacific, which established a strong cross-barrier pressure gradient from the Great Basin westward across the Sierra Nevada. This section characterizes the synoptic-scale forcing using east--west vertical cross-sections at 39.8$^\circ$N, spanning from the northern California coast (123.0$^\circ$W) to the western Nevada border (119.5$^\circ$W) --- a transect of approximately 299~km that captures the full breadth of the coastal ranges, Sacramento Valley, Sierra Nevada, and the leeward Great Basin. All fields correspond to forecast hour 15 (1500 UTC, 0700 PST 8 November 2018), approximately 30 minutes after the reported ignition of the Camp Fire.

\subsection{Upper-Level Potential Vorticity Structure}
\label{sec:pv_structure}

The potential vorticity (PV) field along the east--west transect reveals the upper-level dynamic forcing responsible for the offshore wind event (Figure~\ref{fig:synoptic_pv_ew}). A pronounced PV anomaly was centered over northern California, with maximum values exceeding 4.0 PVU at 300--350~hPa. Specifically, the HRRR analysis shows:

\begin{itemize}
    \item At 350~hPa, the PV maximum reached 4.05~PVU at 121.1$^\circ$W, positioned directly over the Sierra crest near the latitude of the fire origin.
    \item At 300~hPa, PV values of 4.05~PVU extended eastward to 120.4$^\circ$W, indicating a broad zone of stratospheric air at the upper-tropospheric level.
    \item At 400~hPa, PV reached 3.00~PVU at 122.5$^\circ$W, west of the Sacramento Valley, consistent with a deep tropopause fold extending below the 400-hPa level.
    \item At 450~hPa, elevated PV of 2.43~PVU persisted at 122.8$^\circ$W, with the anomaly tilting westward with decreasing altitude --- a signature of baroclinic development.
\end{itemize}

This PV structure indicates that the dynamic tropopause had descended to approximately 350--400~hPa over the northern Sierra Nevada, bringing stratospheric air with high PV and low moisture content into the upper troposphere. The westward tilt with height is characteristic of an amplifying shortwave trough, and the positioning of the PV maximum directly over the Sierra crest was optimal for generating the strongest possible cross-barrier ageostrophic flow on the western (lee) side of the range. At lower levels, an anomalous PV feature of 3.09~PVU at 700~hPa near 121.0$^\circ$W likely reflects the diabatic PV generation associated with the sharp moisture gradient at the top of the extremely dry descending air mass.

\begin{figure}[htbp]
\centering
\includegraphics[width=\textwidth]{figures/synoptic_pv_ew_f15}
\caption{East--west vertical cross-section of potential vorticity (PVU) at 39.8$^\circ$N from 123.0$^\circ$W to 119.5$^\circ$W, valid 1500 UTC 8 November 2018 (HRRR FHR~15). The PV maximum exceeding 4.0~PVU at 300--350~hPa over the Sierra crest (121.0--121.1$^\circ$W) indicates a deep tropopause fold associated with the shortwave trough driving the offshore wind event. The westward tilt of the PV anomaly with height is consistent with a developing baroclinic system. Terrain is shaded in gray; surface pressure defines the lower boundary.}
\label{fig:synoptic_pv_ew}
\end{figure}

\subsection{850-hPa Temperature Structure and Adiabatic Warming}
\label{sec:temp_structure}

The east--west temperature cross-section (Figure~\ref{fig:synoptic_temp_ew}) reveals a pronounced warm anomaly at 850~hPa on the western slope of the Sierra Nevada, consistent with adiabatic compressional warming of the descending air mass. Along the 39.8$^\circ$N transect at 850~hPa, the HRRR analysis shows:

\begin{itemize}
    \item Over the eastern Sierra and Great Basin (119.5--120.5$^\circ$W), 850-hPa temperatures ranged from 0.1 to 0.8$^\circ$C, reflecting the cold continental air mass upstream of the barrier.
    \item At the Sierra crest (121.0$^\circ$W), the 850-hPa temperature was 2.9$^\circ$C.
    \item Over the western Sierra foothills (121.4--121.6$^\circ$W), temperatures increased sharply to 8.1--10.3$^\circ$C.
    \item The 850-hPa temperature maximum of 10.3$^\circ$C was located at 121.6$^\circ$W, in the immediate vicinity of Paradise.
\end{itemize}

The temperature increase of approximately 7$^\circ$C from the eastern Sierra (120.5$^\circ$W) to the lee slope maximum (121.6$^\circ$W) over a horizontal distance of roughly 100~km constitutes a strong thermal gradient. This warming is consistent with adiabatic descent of approximately 700--800~m, assuming a dry adiabatic lapse rate of 9.8$^\circ$C~km$^{-1}$. The magnitude and sharpness of this gradient distinguish the event from typical offshore flow episodes, where 850-hPa lee-side warming is typically 3--5$^\circ$C. Farther west, temperatures remained elevated at 9.3--9.8$^\circ$C across the Sacramento Valley (122.0--123.0$^\circ$W), indicating that the warm, subsidence-modified air mass had propagated well downstream of the terrain barrier.

\begin{figure}[htbp]
\centering
\includegraphics[width=\textwidth]{figures/synoptic_temp_ew_f15}
\caption{East--west vertical cross-section of temperature ($^\circ$C) at 39.8$^\circ$N from 123.0$^\circ$W to 119.5$^\circ$W, valid 1500 UTC 8 November 2018 (HRRR FHR~15). The warm anomaly at 850~hPa over the western Sierra slope (10.3$^\circ$C at 121.6$^\circ$W versus 0.1$^\circ$C at 120.3$^\circ$W) reflects compressional warming in the descending air mass. Note the steep horizontal temperature gradient between the Sierra crest and the western foothills, a thermodynamic fingerprint of the downslope windstorm.}
\label{fig:synoptic_temp_ew}
\end{figure}

\subsection{Broad-Scale Wind Field}
\label{sec:wind_field}

The east--west wind speed cross-section (Figure~\ref{fig:synoptic_wind_ew}) reveals a vertically stacked wind structure with two distinct maxima: an upper-level jet at 250--300~hPa and a low-level jet at 850--875~hPa over the western Sierra Nevada.

At the upper levels, the 250-hPa jet streak reached 63.8~kt (32.8~m~s$^{-1}$) near the coast, with values of 57.7--59.2~kt (29.7--30.5~m~s$^{-1}$) over the Sierra crest. The 300-hPa maximum was 56.9~kt (29.3~m~s$^{-1}$), also positioned near the coast with 51.8--53.1~kt (26.7--27.3~m~s$^{-1}$) over the crest. This upper-level wind maximum, positioned upstream and to the west, is consistent with the jet stream configuration associated with the amplified shortwave trough.

At the operationally critical low levels, the wind structure over the Sierra Nevada was dominated by a terrain-channeled jet:

\begin{itemize}
    \item At 875~hPa, the maximum wind speed reached 38.2~kt (19.7~m~s$^{-1}$) at 121.4$^\circ$W, directly over the western Sierra slope near the Feather River Canyon.
    \item At 850~hPa, winds of 37.7~kt (19.4~m~s$^{-1}$) were centered at 121.4$^\circ$W.
    \item At 900~hPa, the maximum was 33.7~kt (17.3~m~s$^{-1}$) at 121.4$^\circ$W.
    \item At 925~hPa, still within the terrain-influenced layer, winds reached 25.4~kt (13.1~m~s$^{-1}$) at 121.4$^\circ$W.
\end{itemize}

The low-level wind maximum was thus sharply localized over the western Sierra slope, with 875-hPa wind speeds decreasing to 15.6~kt at the Sierra crest (121.0$^\circ$W) and to less than 10~kt in the Sacramento Valley (122.0$^\circ$W). This spatial concentration of the wind maximum in the 875--850-hPa layer over the western slope is characteristic of downslope windstorm dynamics, where mountain wave amplification and hydraulic acceleration focus kinetic energy into the terrain-descending flow \citep{Durran_1990}. At 800~hPa, a secondary wind maximum of 39.9~kt appeared at 121.0$^\circ$W over the crest itself, indicating that the jet structure had a vertically complex morphology, with wind maxima at different longitudes depending on the pressure level.

The canyon-path cross-section through the Feather River Canyon (Figure~\ref{fig:overview_canyon_wind}) further resolves the low-level jet structure along the fire's approach path. Along the northeast--southwest transect from 40.2$^\circ$N, 121.0$^\circ$W to 39.4$^\circ$N, 121.9$^\circ$W, terrain-level winds of 35--39~kt were concentrated within the canyon, with the jet core positioned at 875--900~hPa directly over the fire origin area near Pulga and extending downstream toward Paradise.

\begin{figure}[htbp]
\centering
\includegraphics[width=\textwidth]{figures/synoptic_wind_ew_f15}
\caption{East--west vertical cross-section of wind speed (kt) at 39.8$^\circ$N from 123.0$^\circ$W to 119.5$^\circ$W, valid 1500 UTC 8 November 2018 (HRRR FHR~15). The low-level wind maximum of 38.2~kt at 875~hPa over the western Sierra slope (121.4$^\circ$W) represents the downslope jet driving the Camp Fire. The upper-level jet streak at 250~hPa (63.8~kt) is associated with the shortwave trough. Note the sharp decrease in low-level wind speed both upstream (eastern Sierra) and downstream (Sacramento Valley) of the terrain-channeled maximum.}
\label{fig:synoptic_wind_ew}
\end{figure}

\begin{figure}[htbp]
\centering
\includegraphics[width=\textwidth]{figures/overview_canyon_wind_f15}
\caption{Vertical cross-section of wind speed (kt) along the Feather River Canyon path from 40.2$^\circ$N, 121.0$^\circ$W (northeast, Sierra crest) to 39.4$^\circ$N, 121.9$^\circ$W (southwest, Sacramento Valley), valid 1500 UTC 8 November 2018 (HRRR FHR~15). The low-level jet of 35--39~kt at 875--900~hPa is channeled through the canyon, with the maximum positioned near the elevation of the canyon mouth and Paradise Ridge. Wind speeds decrease sharply below the jet core in the Sacramento Valley.}
\label{fig:overview_canyon_wind}
\end{figure}

\subsection{Subsidence and Vertical Motion}
\label{sec:omega_structure}

The omega ($\omega$, vertical velocity in pressure coordinates) field along the east--west transect (Figure~\ref{fig:synoptic_omega_ew}) reveals the mesoscale pattern of sinking motion that produced the adiabatic warming and extreme drying documented above. The convention follows standard meteorological practice, where positive $\omega$ denotes subsidence (downward motion).

The most prominent feature is a broad zone of strong subsidence extending from approximately 121.5$^\circ$W to 120.5$^\circ$W, centered over the western Sierra Nevada. At 850~hPa, subsidence values reached 5.9~hPa~hr$^{-1}$ near 121.4$^\circ$W, directly over the fire area. At 800~hPa, the subsidence maximum intensified to 6.5~hPa~hr$^{-1}$ at 121.2$^\circ$W, immediately upstream of the fire origin. Converting to approximate vertical velocity, 6.5~hPa~hr$^{-1}$ corresponds to roughly 0.18~m~s$^{-1}$, or approximately 650~m~hr$^{-1}$ of descent. At this rate, an air parcel originating at 700~hPa (approximately 3,000~m above sea level) would reach the 850-hPa surface in 3--4 hours, continuously warming at the dry adiabatic rate and maintaining its extremely low moisture content throughout the descent.

The subsidence was not confined to the immediate lee of the Sierra crest. Weak to moderate sinking motion of 3--4~hPa~hr$^{-1}$ extended westward across the Sacramento Valley and even to the coastal ranges near 123.0$^\circ$W, indicating that the entire region was under the influence of large-scale descent associated with the upper-level trough. A narrow band of weak upward motion near 121.8$^\circ$W over the Sacramento Valley foothills may represent the leading edge of the downslope flow or a weak hydraulic jump feature where the descending current encounters the ambient valley air mass.

\begin{figure}[htbp]
\centering
\includegraphics[width=\textwidth]{figures/synoptic_omega_ew_f15}
\caption{East--west vertical cross-section of omega (hPa~hr$^{-1}$) at 39.8$^\circ$N from 123.0$^\circ$W to 119.5$^\circ$W, valid 1500 UTC 8 November 2018 (HRRR FHR~15). Positive values (warm colors) indicate subsidence. The sinking motion maximum of 6.5~hPa~hr$^{-1}$ at 800~hPa over the western Sierra (121.2$^\circ$W) represents the core of the descending air mass responsible for the extreme adiabatic warming and drying over the fire area. The subsidence extends broadly across the Sacramento Valley and coastal ranges.}
\label{fig:synoptic_omega_ew}
\end{figure}

\subsection{Relative Humidity and the Dry Intrusion}
\label{sec:rh_structure}

The east--west relative humidity (RH) cross-section (Figure~\ref{fig:synoptic_rh_ew}) documents the extreme atmospheric dryness that characterized the Camp Fire environment. Along the entire 299-km transect at 850~hPa, RH values were uniformly below 12\%, with the lowest values concentrated over the western Sierra slope and Sacramento Valley. At 700~hPa, RH dropped below 6\% over much of the transect, and at 600~hPa, values as low as 3\% were present --- approaching the practical measurement limit for this quantity.

The depth and intensity of the dry air mass are striking: from the surface to approximately 500~hPa --- a column spanning roughly 5--6~km --- RH was below 15\% across the entire transect. This extreme dryness was not a shallow surface phenomenon produced by daytime heating but rather a deep, synoptically forced dry intrusion, with the air mass having originated at upper-tropospheric levels (400--500~hPa or higher) before descending adiabatically across the Sierra Nevada. The absence of any moist layer in the lower or middle troposphere meant that turbulent mixing could not entrain moisture from above, and the normal diurnal RH recovery cycle --- which depends on radiative cooling concentrating existing moisture into a shallow nocturnal boundary layer --- was completely overwhelmed by the continuous advection of desiccated air from aloft.

The canyon-path RH cross-section (Figure~\ref{fig:overview_canyon_rh}) shows that along the fire's actual approach path, terrain-level RH values were 10--13\% at the ignition area near Pulga and 6--9\% at 850~hPa over the canyon. These values are consistent with the analysis report finding of dewpoint depressions approaching 37$^\circ$C at 850~hPa --- conditions in which even live fuels lose moisture rapidly and become available for combustion.

\begin{figure}[htbp]
\centering
\includegraphics[width=\textwidth]{figures/synoptic_rh_ew_f15}
\caption{East--west vertical cross-section of relative humidity (\%) at 39.8$^\circ$N from 123.0$^\circ$W to 119.5$^\circ$W, valid 1500 UTC 8 November 2018 (HRRR FHR~15). The entire lower troposphere below 500~hPa exhibits RH values below 15\%, with minima below 5\% at 700~hPa over the Sierra crest. This deep, synoptically forced dry intrusion reflects the upper-tropospheric origin of the descending air mass. No moist layer is present at any level along the transect.}
\label{fig:synoptic_rh_ew}
\end{figure}

\begin{figure}[htbp]
\centering
\includegraphics[width=\textwidth]{figures/overview_canyon_rh_f15}
\caption{Vertical cross-section of relative humidity (\%) along the Feather River Canyon path from 40.2$^\circ$N, 121.0$^\circ$W (northeast) to 39.4$^\circ$N, 121.9$^\circ$W (southwest), valid 1500 UTC 8 November 2018 (HRRR FHR~15). Terrain-level RH values of 10--13\% at the fire origin near Pulga and 6--9\% at 850~hPa in the canyon document the extreme atmospheric moisture deficit along the fire's approach path. The dryness extended through the full depth of the troposphere below 500~hPa.}
\label{fig:overview_canyon_rh}
\end{figure}

\subsection{Canyon Temperature Structure}
\label{sec:canyon_temp}

The temperature cross-section along the Feather River Canyon path (Figure~\ref{fig:overview_canyon_temp}) complements the broad east--west view by resolving the thermal structure along the terrain gradient most relevant to the fire. The cross-section reveals:

\begin{itemize}
    \item A subsidence inversion at approximately 875--900~hPa, where the lapse rate was only 2--3$^\circ$C~km$^{-1}$ (far more stable than the standard atmosphere), trapping warm descending air at the elevation of the canyon and Paradise Ridge.
    \item Surface temperatures of 12--13$^\circ$C at Paradise (elevation $\sim$540~m) at 0700 PST, which is anomalously warm for a November morning and reflects the adiabatically warmed foehn air.
    \item A sharp temperature contrast at the base of the foothills, where the warm downslope flow ($\sim$15$^\circ$C at 950~hPa) overrode the cooler Sacramento Valley air mass, creating a surface-based temperature inversion in the valley.
\end{itemize}

This thermal configuration placed Paradise squarely within the warm core of the descending air mass. The subsidence inversion acted as a lid that concentrated the strongest winds in the lowest 1--2~km of the atmosphere and prevented the fire's convective column from developing significant vertical extent during the initial hours of rapid spread, forcing the fire's energy laterally rather than vertically and promoting the extraordinarily rapid horizontal rate of spread observed.

\begin{figure}[htbp]
\centering
\includegraphics[width=\textwidth]{figures/overview_canyon_temp_f15}
\caption{Vertical cross-section of temperature ($^\circ$C) along the Feather River Canyon path from 40.2$^\circ$N, 121.0$^\circ$W (northeast, Sierra crest) to 39.4$^\circ$N, 121.9$^\circ$W (southwest, Sacramento Valley), valid 1500 UTC 8 November 2018 (HRRR FHR~15). The warm nose at 850--900~hPa over the canyon reflects adiabatic compression of the descending air. The subsidence inversion at $\sim$875~hPa is evident from the compressed isotherms above the terrain surface, trapping the warmest and driest air at the elevation of Paradise.}
\label{fig:overview_canyon_temp}
\end{figure}

\subsection{Summary of Synoptic Forcing}

Table~\ref{tab:synoptic_summary} summarizes the key synoptic-scale parameters from the east--west cross-section analysis. The Camp Fire environment was characterized by a deep upper-level trough with a strong PV anomaly positioned directly over the Sierra crest, producing a well-defined downslope windstorm with an 875-hPa jet of 38~kt, 850-hPa adiabatic warming of $\sim$7$^\circ$C from the eastern to western Sierra, persistent subsidence of 5--6~hPa~hr$^{-1}$, and extreme dryness extending through the full depth of the lower troposphere. The synoptic pattern was, in effect, optimally configured to maximize every atmospheric parameter that promotes catastrophic wildfire behavior.

\begin{table}[htbp]
\centering
\caption{Summary of key synoptic-scale parameters from the HRRR east--west cross-section at 39.8$^\circ$N, valid 1500 UTC 8 November 2018 (FHR~15).}
\label{tab:synoptic_summary}
\begin{tabular}{lll}
\toprule
\textbf{Parameter} & \textbf{Value} & \textbf{Location} \\
\midrule
PV maximum (350 hPa) & 4.05 PVU & 121.1$^\circ$W \\
PV maximum (400 hPa) & 3.00 PVU & 122.5$^\circ$W \\
850-hPa $T$ maximum (lee) & 10.3$^\circ$C & 121.6$^\circ$W \\
850-hPa $T$ at Sierra crest & 2.9$^\circ$C & 121.0$^\circ$W \\
850-hPa $T$ gradient & $\sim$7$^\circ$C / 100 km & Crest to lee slope \\
875-hPa wind maximum & 38.2 kt (19.7 m s$^{-1}$) & 121.4$^\circ$W \\
850-hPa wind maximum & 37.7 kt (19.4 m s$^{-1}$) & 121.4$^\circ$W \\
250-hPa jet streak & 63.8 kt (32.8 m s$^{-1}$) & 123.0$^\circ$W \\
$\omega$ maximum (800 hPa) & 6.5 hPa hr$^{-1}$ & 121.2$^\circ$W \\
850-hPa RH (transect) & $<$12\% & Entire 299-km transect \\
700-hPa RH minimum & $<$6\% & Over Sierra crest \\
\bottomrule
\end{tabular}
\end{table}
