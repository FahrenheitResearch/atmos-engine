%!TEX root = ../main.tex
%----------------------------------------------------------------------
% Section 3: Four Corners Critical Area Analysis
% HRRR 06z 09 Feb 2026  |  Valid 10 Feb 2026
%----------------------------------------------------------------------

\section{Four Corners Critical Area Analysis (Arizona--New Mexico--Colorado)}
\label{sec:southern}

The Four Corners region of the American Southwest presents a uniquely
challenging fire weather environment for 10 February 2026.  Situated at the
confluence of the Colorado Plateau, the Mogollon Rim, and the upper Rio
Grande Valley, the complex terrain modulates wind flow, moisture transport,
and vertical mixing in ways that amplify fire danger well beyond what
synoptic-scale analysis alone would suggest.  This section presents a
detailed cross-section diagnosis of the HRRR 06z 09 February 2026 cycle
across four carefully designed transects that sample the critical terrain
features of this region.

\medskip
\noindent\textbf{Transect definitions:}
\begin{itemize}
  \item \textbf{E--W Main} (35.0\textdegree N, 113.0\textdegree W $\to$
        35.0\textdegree N, 105.0\textdegree W): 729~km east--west transect
        across central Arizona into central New Mexico, sampling the
        Mogollon Rim, the Tonto Basin, the White Mountains, and the Rio
        Grande Valley.
  \item \textbf{E--W Northern} (37.0\textdegree N, 112.0\textdegree W
        $\to$ 37.0\textdegree N, 105.0\textdegree W): 637~km transect
        across southern Utah/Colorado Plateau into the San Juan Mountains
        and the upper San Luis Valley.
  \item \textbf{N--S} (38.0\textdegree N, 109.0\textdegree W $\to$
        32.0\textdegree N, 109.0\textdegree W): 667~km meridional transect
        through the Four Corners point, from the La Sal Mountains of Utah
        south through the Arizona--New Mexico border country into the
        Chihuahuan Desert.
  \item \textbf{Mogollon Rim} (35.5\textdegree N, 112.5\textdegree W
        $\to$ 34.0\textdegree N, 108.0\textdegree W): 430~km
        northwest-to-southeast transect diagonally across the Mogollon Rim
        escarpment, capturing the dramatic 600--900~m terrain drop and its
        effects on downslope flow.
\end{itemize}

\medskip
\noindent\textbf{Seasonal context.}  Winter fire season in the Southwest
is an increasingly recognized hazard.  Unlike the traditional June
pre-monsoon fire season driven by lightning ignitions in cured fine fuels,
the February fire window is dominated by drought-stressed dormant grasslands
and failed early green-up.  La Ni\~{n}a--pattern winters suppress Pacific
moisture transport into Arizona and New Mexico, leaving fine fuel moisture
content anomalously low.  The combination of dormant $C_4$ grasses, sparse
winter precipitation, and strong diurnal heating under clear skies creates
a fire environment where wind-driven runs on grass and light brush fuels
can produce rapid rates of spread despite modest temperatures.

%----------------------------------------------------------------------
\subsection{Wind Field and Terrain Channeling}
\label{sec:southern:wind}

\begin{figure}[htbp]
  \centering
  \includegraphics[width=\textwidth]{figures/south_ew_main_wind_speed_f39.png}
  \caption{Wind speed cross-section along the E--W Main transect
    (35\textdegree N, 113\textdegree W $\to$ 105\textdegree W) valid 21z
    10 Feb 2026 (FHR~39, 2~PM MST).  Terrain is shaded brown; pink
    contour marks the freezing level.  Wind barbs show flow direction and
    speed.}
  \label{fig:south_ew_wind_f39}
\end{figure}

The wind analysis at peak heating (21z, 2~PM MST; Figure~\ref{fig:south_ew_wind_f39})
reveals a moderate low-level wind regime across the E--W Main transect.
Near-surface winds along the transect range from 0.6 to 17.3~kt, with a
mean of 8.4~kt.  The strongest near-surface winds (up to 17~kt) are found
at the western end of the transect near the 800~hPa level, corresponding
to the elevated terrain of the Hualapai Mountains and the transition from
the Mojave Desert into the Colorado Plateau.  The maximum wind speed below
700~hPa reaches 19.6~kt at the western terminus.

At the 700~hPa level---which lies near crest height for much of the
Mogollon Rim---winds are stronger, ranging from 4.7 to 19.6~kt with a
mean of 9.7~kt.  The 750~hPa level shows similar values (2.3--18.3~kt,
mean 9.3~kt).  These speeds, while not individually reaching the 25~kt
sustained threshold for Red Flag conditions, represent the ambient flow
that can be locally amplified by terrain channeling.

\begin{figure}[htbp]
  \centering
  \includegraphics[width=\textwidth]{figures/south_mogollon_wind_speed_f39.png}
  \caption{Wind speed cross-section along the Mogollon Rim transect
    (35.5\textdegree N, 112.5\textdegree W $\to$ 34.0\textdegree N,
    108.0\textdegree W) valid 21z 10 Feb (FHR~39).  Note the terrain
    profile capturing the Rim escarpment.}
  \label{fig:south_mogollon_wind_f39}
\end{figure}

The Mogollon Rim transect (Figure~\ref{fig:south_mogollon_wind_f39})
reveals the terrain influence on the wind field.  The Rim itself, rising
to approximately 700~hPa equivalent pressure altitude, acts as a barrier
to low-level flow.  Winds accelerate along the Rim escarpment, where
the terrain drops sharply from the Colorado Plateau (~2300~m) to the
Tonto Basin (~600~m).  This downslope acceleration zone is a well-known
fire weather hazard for central Arizona communities like Payson, Pine,
and Strawberry.

\begin{figure}[htbp]
  \centering
  \includegraphics[width=\textwidth]{figures/south_ns_wind_speed_f39.png}
  \caption{Wind speed cross-section along the N--S transect
    (38\textdegree N $\to$ 32\textdegree N at 109\textdegree W) valid 21z
    10 Feb (FHR~39).  This meridional transect crosses the La Sal
    Mountains, the Four Corners area, and extends south into the
    Chihuahuan borderlands.}
  \label{fig:south_ns_wind_f39}
\end{figure}

The N--S transect along 109\textdegree W (Figure~\ref{fig:south_ns_wind_f39})
shows the meridional wind structure through the heart of the Four Corners
region.  Winds are generally light to moderate through the boundary layer,
with the strongest low-level flow concentrated above the elevated terrain
of the San Juan Basin and the Chuska Mountains.  The complex terrain
fragmentation of the flow field is evident, with wind speed varying
significantly across short horizontal distances as the flow negotiates
canyon systems and mesa edges.

\begin{figure}[htbp]
  \centering
  \includegraphics[width=\textwidth]{figures/south_ew_main_wind_speed_f24.png}
  \caption{Wind speed temporal reference: E--W Main transect at 06z 10 Feb
    (FHR~24, overnight).  Note the quiescent overnight boundary layer with
    light surface winds and a developing low-level jet.}
  \label{fig:south_ew_wind_f24}
\end{figure}

\begin{figure}[htbp]
  \centering
  \includegraphics[width=\textwidth]{figures/south_ew_main_wind_speed_f42.png}
  \caption{Wind speed at 00z 11 Feb (FHR~42, late afternoon 5~PM MST).
    Winds remain elevated through the end of the afternoon heating cycle.}
  \label{fig:south_ew_wind_f42}
\end{figure}

The temporal evolution from overnight (Figure~\ref{fig:south_ew_wind_f24})
through late afternoon (Figure~\ref{fig:south_ew_wind_f42}) reveals the
diurnal wind cycle.  Overnight winds at 06z are relatively strong aloft
with a modest low-level jet feature near the 700~hPa level, while surface
winds are suppressed under the nocturnal inversion.  By 21z (peak
heating), turbulent mixing has coupled the boundary layer to the free
atmosphere, transporting mid-level momentum to the surface.  The wind
field remains active through 00z (5~PM MST), indicating that the fire
danger window extends through late afternoon.

Despite the moderate wind speeds, the 3\% of near-surface points exceeding
15~kt---concentrated along elevated terrain crests and ridges---represent
the locations most vulnerable to wind-driven fire spread.  The absence of
sustained winds above 25~kt suggests that this event does not meet the
traditional Red Flag wind threshold; however, when combined with the
extremely low humidity discussed in the following subsection, even these
moderate winds are operationally significant.

%----------------------------------------------------------------------
\subsection{Humidity Analysis}
\label{sec:southern:rh}

\begin{figure}[htbp]
  \centering
  \includegraphics[width=\textwidth]{figures/south_ew_main_rh_f39.png}
  \caption{Relative humidity (\%) along the E--W Main transect valid 21z
    10 Feb (FHR~39).  Green contour marks the freezing level; brown
    shading is terrain.  The entire low-level atmosphere is extremely dry.}
  \label{fig:south_ew_rh_f39}
\end{figure}

The humidity analysis is the most concerning element of this forecast.
At 21z (2~PM MST), the E--W Main transect
(Figure~\ref{fig:south_ew_rh_f39}) reveals a profoundly dry atmosphere
across the entire 729~km cross-section.  Near-surface relative humidity
ranges from 15.3\% to 32.7\%, with a transect mean of just 23.0\%.  The
driest air at the surface is found in eastern New Mexico near
107.9\textdegree W (the eastern flanks of the Manzano and Sandia
Mountains), where RH approaches the critical 15\% threshold at 15.3\%.

The 850~hPa level shows RH values of 15.4--29.5\% (mean 21.9\%),
indicating that the dryness extends well into the elevated boundary
layer rather than being confined to a shallow surface layer.  At 700~hPa,
RH recovers somewhat to 25.3--54.0\% (mean 38.1\%), but even at this
level the atmosphere remains far drier than climatological norms.

\begin{figure}[htbp]
  \centering
  \includegraphics[width=\textwidth]{figures/south_ew_north_rh_f39.png}
  \caption{Relative humidity along the E--W Northern transect
    (37\textdegree N, 112\textdegree W $\to$ 105\textdegree W) valid 21z
    10 Feb (FHR~39).  Southern Colorado and the San Juan Mountains show
    slightly higher RH values but remain critically dry.}
  \label{fig:south_ew_north_rh_f39}
\end{figure}

The E--W Northern transect at 37\textdegree N
(Figure~\ref{fig:south_ew_north_rh_f39}) shows a similar pattern across
southern Colorado, though with slightly higher RH values driven by higher
terrain and cooler temperatures.  Even here, the boundary layer remains
critically dry, with RH values in the 20--35\% range across the San Juan
Basin and the lower elevations of the Colorado Plateau.  The San Juan
Mountains provide a modest moisture source through orographic uplift, but
this effect is confined to above 600~hPa and does not extend downward into
the fire-critical boundary layer.

\begin{figure}[htbp]
  \centering
  \includegraphics[width=\textwidth]{figures/south_ns_rh_f39.png}
  \caption{Relative humidity along the N--S transect (38\textdegree N
    $\to$ 32\textdegree N at 109\textdegree W) valid 21z 10 Feb (FHR~39).
    The low-level dry layer is deepest over the southern Chihuahuan
    terrain.}
  \label{fig:south_ns_rh_f39}
\end{figure}

The N--S transect (Figure~\ref{fig:south_ns_rh_f39}) reveals the
meridional gradient in moisture.  Moving southward from the La Sal
Mountains of Utah through the Four Corners into southern New Mexico, the
boundary layer becomes progressively drier.  The deepest dry layer---with
RH values below 25\% extending from the surface through 700~hPa---is
found over the lower-elevation terrain south of 34\textdegree N.  This
deep dry layer is a hallmark of the subsidence-driven aridification
discussed in Section~\ref{sec:southern:omega}.

\begin{figure}[htbp]
  \centering
  \includegraphics[width=\textwidth]{figures/south_mogollon_rh_f39.png}
  \caption{Relative humidity along the Mogollon Rim transect valid 21z
    10 Feb (FHR~39).  The sharp terrain drop of the Rim creates a
    distinctive moisture gradient.}
  \label{fig:south_mogollon_rh_f39}
\end{figure}

The Mogollon Rim transect (Figure~\ref{fig:south_mogollon_rh_f39})
highlights a terrain-driven moisture contrast.  Above the Rim, at
approximately 700~hPa, a moister layer (30--50\% RH) is maintained by
the higher terrain intercepting ambient moisture.  Below the Rim, the
Tonto Basin and lower Sonoran Desert fringe show dramatically drier
conditions.  This moisture gradient enhances downslope drying when winds
transport Rim-top air into the lower terrain.

\begin{figure}[htbp]
  \centering
  \includegraphics[width=\textwidth]{figures/south_ew_main_rh_f24.png}
  \caption{Relative humidity temporal evolution: E--W Main transect at 06z
    10 Feb (FHR~24, overnight).  The nocturnal inversion produces modest
    moisture recovery, but RH remains below 40\% even during the
    overnight recovery period.}
  \label{fig:south_ew_rh_f24}
\end{figure}

\begin{figure}[htbp]
  \centering
  \includegraphics[width=\textwidth]{figures/south_ew_main_rh_f42.png}
  \caption{Relative humidity at 00z 11 Feb (FHR~42, late afternoon).
    Afternoon drying has driven RH to near-critical levels across most
    of the transect.}
  \label{fig:south_ew_rh_f42}
\end{figure}

The diurnal humidity cycle (Figures~\ref{fig:south_ew_rh_f24}
and~\ref{fig:south_ew_rh_f42}) shows a striking feature: even during the
overnight recovery period at 06z, near-surface RH barely recovers above
40\%.  In a well-watered environment, nocturnal cooling would produce
recovery to 70--90\%.  The absence of meaningful overnight recovery
indicates that the soil moisture reservoir is severely depleted and that
dry air advection from the subsiding free atmosphere overwhelms radiative
cooling.  By late afternoon (00z), RH has returned to the 15--25\% range,
indicating that the fire danger window is both deep and prolonged.

\textbf{Key finding:} While the 15\% RH threshold is only barely reached
at the driest surface point (15.3\% at 107.9\textdegree W), the
\emph{depth} and \emph{spatial extent} of the dry layer---with mean
near-surface RH of 23\% and 850~hPa mean RH of 22\%---is the more
significant metric.  The entire boundary layer, from the surface through
at least the 700~hPa level, is critically dry.  This deep dry layer means
that any fire producing a convection column will entrain dry air at all
levels, promoting extreme spotting distances and drying of fuels well
ahead of the flame front.

%----------------------------------------------------------------------
\subsection{Temperature Structure and Mixing Height}
\label{sec:southern:temp}

\begin{figure}[htbp]
  \centering
  \includegraphics[width=\textwidth]{figures/south_ew_main_temperature_f39.png}
  \caption{Temperature (\textdegree C) along the E--W Main transect valid
    21z 10 Feb (FHR~39).  Isotherms show warm surface temperatures across
    the lower terrain.  The freezing level (pink) is at approximately
    700~hPa.}
  \label{fig:south_ew_temp_f39}
\end{figure}

The temperature structure at 21z (Figure~\ref{fig:south_ew_temp_f39})
reveals near-surface temperatures ranging from 4.6\textdegree C on the
highest terrain to 18.9\textdegree C in the lower basins, with a mean of
12.9\textdegree C.  These are moderate temperatures for February---warm
enough to drive active boundary layer mixing but not extreme.  The 850~hPa
temperature of 11.1--19.4\textdegree C (mean 16.4\textdegree C) indicates
a well-mixed, nearly adiabatic layer from the surface to the terrain-top
level.

The freezing level lies near 700~hPa (~3000~m MSL), placing it above
most of the fire-relevant terrain.  The 700~hPa temperatures of
0.3--4.1\textdegree C (mean 2.6\textdegree C) are notably warm for early
February, contributing to the elevated snowline and dry fuel conditions
at moderate elevations.  At 500~hPa, temperatures of $-$14.7 to
$-$16.4\textdegree C (mean $-$15.4\textdegree C) indicate a relatively
warm mid-tropospheric environment.

\begin{figure}[htbp]
  \centering
  \includegraphics[width=\textwidth]{figures/south_ew_main_temperature_f36.png}
  \caption{Temperature along the E--W Main transect at 18z 10 Feb
    (FHR~36, late morning).  Comparison with Figure~\ref{fig:south_ew_temp_f39}
    shows the 3-hour temperature rise during peak heating.}
  \label{fig:south_ew_temp_f36}
\end{figure}

The 700--500~hPa lapse rate is critically important for fire behavior.
The temperature difference from 700~hPa (mean 2.6\textdegree C) to
500~hPa (mean $-$15.4\textdegree C) yields a mean mid-tropospheric lapse
rate of approximately 6.5\textdegree C/km---near the moist adiabatic rate
but below the dry adiabatic rate.  However, the surface-to-700~hPa layer
is steeper, as confirmed by the lapse rate analysis below.

\begin{figure}[htbp]
  \centering
  \includegraphics[width=\textwidth]{figures/south_ew_main_lapse_rate_f39.png}
  \caption{Lapse rate (\textdegree C/km) along the E--W Main transect
    valid 21z 10 Feb (FHR~39).  Warm colors (reds) indicate steep
    (unstable) lapse rates exceeding 8\textdegree C/km.  Hatching marks
    regions approaching or exceeding the dry adiabatic rate.}
  \label{fig:south_ew_lapse_f39}
\end{figure}

The lapse rate cross-section (Figure~\ref{fig:south_ew_lapse_f39}) reveals
steep low-level lapse rates across the entire transect.  Near-surface
lapse rates of 8--10\textdegree C/km (approaching the dry adiabatic rate
of 9.8\textdegree C/km) dominate the lowest 100--150~hPa above the
terrain.  This indicates vigorous daytime convective mixing that extends
the mixing height to approximately 650--600~hPa over the lower terrain.

\begin{figure}[htbp]
  \centering
  \includegraphics[width=\textwidth]{figures/south_mogollon_lapse_rate_f39.png}
  \caption{Lapse rate along the Mogollon Rim transect valid 21z 10 Feb
    (FHR~39).  The Rim terrain enhances low-level lapse rates through
    terrain-amplified surface heating.}
  \label{fig:south_mogollon_lapse_f39}
\end{figure}

The Mogollon Rim lapse rate transect (Figure~\ref{fig:south_mogollon_lapse_f39})
shows enhanced instability along the Rim escarpment.  The dark terrain
of the Rim (ponderosa pine and mixed conifer) absorbs solar radiation
efficiently, producing steep surface lapse rates.  The combination of
steep lapse rates and the terrain drop creates conditions favorable for
downslope wind acceleration---a mechanism analogous to chinook/foehn
development, though on a smaller scale.

The deep mixing layer (surface to ~600~hPa, roughly 3~km AGL over the
basins) has critical fire weather implications.  First, it ensures that
dry mid-level air is continuously entrained to the surface through
convective eddies.  Second, it creates conditions favorable for erratic
fire behavior through plume-dominated convection.  Any fire that develops
a convection column within this unstable layer will exhibit rapid vertical
development, producing strong and unpredictable surface inflow winds.

%----------------------------------------------------------------------
\subsection{Vertical Motion and Subsidence}
\label{sec:southern:omega}

\begin{figure}[htbp]
  \centering
  \includegraphics[width=\textwidth]{figures/south_ew_main_omega_f39.png}
  \caption{Vertical velocity ($\omega$, Pa/s) along the E--W Main
    transect valid 21z 10 Feb (FHR~39).  Blue/purple shading indicates
    strong upward motion (negative $\omega$); red/brown indicates
    subsidence (positive $\omega$).  Mountain-wave signatures are
    prominent.}
  \label{fig:south_ew_omega_f39}
\end{figure}

The vertical motion field at 21z (Figure~\ref{fig:south_ew_omega_f39})
reveals a rich pattern of terrain-forced vertical motion.  Strong upward
motion (blue/purple, negative $\omega$) is concentrated in several
distinct columns, reaching from the terrain surface through the
mid-troposphere to above 500~hPa.  These updraft columns are spaced
roughly 50--100~km apart and correspond to the major terrain ridges along
the transect.

Between the updraft columns, compensating subsidence (red/warm colors)
is clearly visible.  This mountain-wave pattern is characteristic of
the Colorado Plateau's mesa-and-canyon topography, where the flow
alternately ascends terrain obstacles and descends in their lee.  The
subsidence between terrain features is a critical drying mechanism:
descending air warms adiabatically and its relative humidity decreases
dramatically.  This terrain-driven subsidence supplements the
synoptic-scale subsidence to produce the extreme dryness documented in
Section~\ref{sec:southern:rh}.

\begin{figure}[htbp]
  \centering
  \includegraphics[width=\textwidth]{figures/south_mogollon_omega_f39.png}
  \caption{Vertical velocity along the Mogollon Rim transect valid 21z
    10 Feb (FHR~39).  The dramatic terrain discontinuity of the Rim
    produces an intense mountain-wave response with strong
    ascent/descent couplets.}
  \label{fig:south_mogollon_omega_f39}
\end{figure}

The Mogollon Rim transect (Figure~\ref{fig:south_mogollon_omega_f39})
shows the most dramatic vertical motion signatures.  The Rim's abrupt
topographic discontinuity---a 600--900~m escarpment over less than
10~km horizontal distance---generates an intense mountain wave.  Strong
ascent on the windward (northwest) side of the Rim is followed by
equally strong lee-side descent.  These mountain-wave signatures extend
vertically through the entire troposphere, well above 500~hPa, indicating
a strong wave response.

The lee-side subsidence below the Rim is particularly concerning for fire
weather.  Air descending from 650--700~hPa to the Tonto Basin floor
(~950~hPa equivalent) experiences approximately 2500~m of descent,
producing adiabatic warming of roughly 25\textdegree C and a dramatic
reduction in RH.  This downslope drying mechanism explains why the
basins immediately below the Mogollon Rim are among the driest and most
fire-prone locations in Arizona.

\begin{figure}[htbp]
  \centering
  \includegraphics[width=\textwidth]{figures/south_ns_omega_f39.png}
  \caption{Vertical velocity along the N--S transect valid 21z 10 Feb
    (FHR~39).  Terrain-driven vertical motion patterns are evident across
    the full meridional extent.}
  \label{fig:south_ns_omega_f39}
\end{figure}

The N--S transect (Figure~\ref{fig:south_ns_omega_f39}) confirms that
terrain-forced vertical motion is a pervasive feature across the entire
Four Corners region.  The updraft/downdraft couplets extend from the La Sal
Mountains in the north through the Chuska Mountains and into the Gila
Wilderness of southern New Mexico.  Both the northern and southern ends
of the transect show large-scale subsidence in the free troposphere above
500~hPa, consistent with the synoptic-scale ridging discussed in the
large-scale overview.

%----------------------------------------------------------------------
\subsection{Vapor Pressure Deficit}
\label{sec:southern:vpd}

\begin{figure}[htbp]
  \centering
  \includegraphics[width=\textwidth]{figures/south_ew_main_vpd_f39.png}
  \caption{Vapor pressure deficit (VPD, hPa) along the E--W Main transect
    valid 21z 10 Feb (FHR~39).  Deep red shading indicates VPD exceeding
    the 13~hPa extreme fire danger threshold.}
  \label{fig:south_ew_vpd_f39}
\end{figure}

Vapor pressure deficit (VPD) integrates the effects of both temperature
and humidity into a single metric that directly relates to the
evaporative demand on fuels.  The VPD analysis at 21z
(Figure~\ref{fig:south_ew_vpd_f39}) reveals an alarmingly dry atmosphere
across the E--W Main transect.

Near-surface VPD values range from 5.9 to 17.8~hPa, with a transect mean
of 11.8~hPa.  The maximum VPD of 17.8~hPa is located at
110.7\textdegree W (approximately 212~km along the transect), in the
vicinity of the Tonto Basin below the Mogollon Rim.  This value exceeds
the 13~hPa extreme fire danger threshold by 37\%.  In total, 39\% of
near-surface grid points along the transect exceed 13~hPa (94 of 242
points), and a remarkable 89\% exceed 8~hPa (216 of 242 points).

\begin{figure}[htbp]
  \centering
  \includegraphics[width=\textwidth]{figures/south_ns_vpd_f39.png}
  \caption{VPD along the N--S transect valid 21z 10 Feb (FHR~39).  The
    highest VPD values are concentrated in the lower-elevation terrain
    of the southern portion of the transect.}
  \label{fig:south_ns_vpd_f39}
\end{figure}

The N--S transect VPD (Figure~\ref{fig:south_ns_vpd_f39}) shows a
clear meridional gradient, with VPD increasing from north to south as
temperatures rise and elevation decreases.  The highest VPD values
(exceeding 15~hPa) are found south of 34\textdegree N, where the
terrain opens into the lower Gila and Chihuahuan Desert country.

\begin{figure}[htbp]
  \centering
  \includegraphics[width=\textwidth]{figures/south_ew_north_vpd_f39.png}
  \caption{VPD along the E--W Northern transect (37\textdegree N) valid
    21z 10 Feb (FHR~39).  Lower temperatures at this latitude moderate
    VPD values slightly compared to the E--W Main transect.}
  \label{fig:south_ew_north_vpd_f39}
\end{figure}

The E--W Northern transect at 37\textdegree N
(Figure~\ref{fig:south_ew_north_vpd_f39}) shows lower VPD values than
the main transect, reflecting the cooler temperatures at higher latitude
and elevation.  However, near-surface VPD still broadly exceeds 8~hPa
across the lower terrain west of the San Juan Mountains, indicating that
fire weather concern extends well into southern Colorado.

The maximum VPD anywhere within the E--W Main cross-section reaches
32.7~hPa at elevated levels where warm temperatures combine with
extremely low humidity.  While this value occurs above the surface, it
represents the desiccating potential of air that can be mixed to the
surface through convective entrainment.

\textbf{Key finding:} The VPD analysis reveals that \emph{extreme}
atmospheric drying demand ($>$13~hPa) is not confined to isolated hot
spots but covers nearly 40\% of the 729~km E--W Main transect at the
surface.  This widespread extreme VPD, combined with the deep dry layer
documented in the RH analysis, creates an atmospheric environment that
will actively draw moisture from any fuels---including live fuels---at a
rate that can produce critical fire weather conditions even with moderate
winds.

%----------------------------------------------------------------------
\subsection{Fire Weather Composite}
\label{sec:southern:composite}

\begin{figure}[htbp]
  \centering
  \includegraphics[width=\textwidth]{figures/south_ew_main_fire_wx_f39.png}
  \caption{Fire weather composite along the E--W Main transect valid 21z
    10 Feb (FHR~39, peak fire danger).  Color shading integrates wind,
    humidity, and instability into a composite fire weather index.  Deep
    red/orange indicates the highest fire danger.}
  \label{fig:south_ew_fire_f39}
\end{figure}

The fire weather composite product (Figure~\ref{fig:south_ew_fire_f39})
synthesizes wind speed, relative humidity, temperature, and instability
into an integrated fire weather assessment.  At 21z along the E--W Main
transect, the composite shows elevated to critical fire weather conditions
across essentially the entire cross-section.  The surface and low-level
atmosphere are uniformly shaded in warm colors (orange through red),
indicating that no portion of this transect is exempt from fire weather
concern.

The most extreme composite values are concentrated in three areas:
\begin{enumerate}
  \item The central portion of the transect (approximately 200--350~km,
        near the AZ--NM border at 109\textdegree W), where the combination
        of moderate terrain elevation, strong VPD, and steep lapse rates
        produces the highest composite scores.
  \item The Mogollon Rim transition zone, where downslope drying and wind
        acceleration compound the ambient fire danger.
  \item The lower basins of central and western Arizona, where warm
        temperatures drive extreme VPD despite slightly higher RH values.
\end{enumerate}

\begin{figure}[htbp]
  \centering
  \includegraphics[width=\textwidth]{figures/south_ew_main_fire_wx_f36.png}
  \caption{Fire weather composite along the E--W Main transect at 18z
    10 Feb (FHR~36, late morning).  Fire danger is already elevated by
    late morning, three hours before peak.}
  \label{fig:south_ew_fire_f36}
\end{figure}

Comparison of the FHR~36 (18z, late morning;
Figure~\ref{fig:south_ew_fire_f36}) and FHR~39 (21z, early afternoon;
Figure~\ref{fig:south_ew_fire_f39}) composites reveals that fire danger
is already elevated by late morning and intensifies through the early
afternoon.  The fire danger window on this day is expected to span from
approximately 16z (9~AM MST) through at least 00z (5~PM MST)---a full
8-hour window of elevated conditions.

\begin{figure}[htbp]
  \centering
  \includegraphics[width=\textwidth]{figures/south_mogollon_fire_wx_f39.png}
  \caption{Fire weather composite along the Mogollon Rim transect valid
    21z 10 Feb (FHR~39).  The Rim escarpment and adjacent terrain show
    particularly high fire weather composite values.}
  \label{fig:south_mogollon_fire_f39}
\end{figure}

\begin{figure}[htbp]
  \centering
  \includegraphics[width=\textwidth]{figures/south_ns_fire_wx_f39.png}
  \caption{Fire weather composite along the N--S transect valid 21z
    10 Feb (FHR~39).  Fire danger increases from north to south as
    temperatures increase and elevation decreases.}
  \label{fig:south_ns_fire_f39}
\end{figure}

The Mogollon Rim composite (Figure~\ref{fig:south_mogollon_fire_f39})
confirms that the Rim country is one of the highest-concern areas within
the Four Corners domain.  The N--S composite
(Figure~\ref{fig:south_ns_fire_f39}) shows a clear meridional gradient,
with the most intense fire weather conditions south of 35\textdegree N.

\begin{figure}[htbp]
  \centering
  \includegraphics[width=\textwidth]{figures/south_ew_north_fire_wx_f39.png}
  \caption{Fire weather composite along the E--W Northern transect
    (37\textdegree N) valid 21z 10 Feb (FHR~39).  Southern Colorado
    and the San Juan Basin show moderately elevated fire weather
    conditions.}
  \label{fig:south_ew_north_fire_f39}
\end{figure}

Even along the E--W Northern transect at 37\textdegree N
(Figure~\ref{fig:south_ew_north_fire_f39}), the fire weather composite
shows elevated conditions across the Colorado Plateau and the San Juan
Basin.  The lower terrain between the San Juan Mountains and the La Plata
Range displays orange shading, indicating that fire weather concern extends
well into southern Colorado.

%----------------------------------------------------------------------
\subsection{Summary and Risk Assessment}
\label{sec:southern:summary}

Table~\ref{tab:south_summary} summarizes the quantitative values extracted
from the HRRR cross-section data at peak fire danger (21z 10 February 2026)
along the E--W Main transect.

\begin{table}[htbp]
\centering
\caption{Fire weather parameter summary for the E--W Main transect
  (35\textdegree N, 113\textdegree W $\to$ 105\textdegree W) at 21z
  10 Feb 2026 (FHR~39).}
\label{tab:south_summary}
\begin{tabular}{lcccc}
\hline
\textbf{Parameter} & \textbf{Min} & \textbf{Max} & \textbf{Mean} &
  \textbf{Threshold} \\
\hline
Near-surface wind (kt) & 0.6 & 17.3 & 8.4 & $>$25 sustained \\
850~hPa wind (kt) & 0.7 & 13.7 & 7.5 & --- \\
700~hPa wind (kt) & 4.7 & 19.6 & 9.7 & --- \\
Near-surface RH (\%) & 15.3 & 32.7 & 23.0 & $<$15 critical \\
850~hPa RH (\%) & 15.4 & 29.5 & 21.9 & --- \\
700~hPa RH (\%) & 25.3 & 54.0 & 38.1 & --- \\
Near-surface temp (\textdegree C) & 4.6 & 18.9 & 12.9 & --- \\
700~hPa temp (\textdegree C) & 0.3 & 4.1 & 2.6 & --- \\
500~hPa temp (\textdegree C) & $-$16.4 & $-$14.7 & $-$15.4 & --- \\
Near-surface VPD (hPa) & 5.9 & 17.8 & 11.8 & $>$13 extreme \\
VPD $>$ 13~hPa coverage & \multicolumn{3}{c}{39\% of transect} & --- \\
VPD $>$ 8~hPa coverage & \multicolumn{3}{c}{89\% of transect} & --- \\
\hline
\end{tabular}
\end{table}

\noindent\textbf{Risk assessment.}  The Four Corners region on 10 February
2026 presents a fire weather scenario that is \emph{deceptively dangerous}.
The moderate wind speeds (below traditional Red Flag thresholds) may tempt
a downgrade in fire weather messaging, but the humidity and VPD analysis
reveals conditions that are unambiguously critical:

\begin{enumerate}
  \item \textbf{Deep, widespread dryness:} The entire boundary layer from
    surface to 700~hPa is critically dry, with mean RH values of
    21--23\% through the lowest 150~hPa.  This is not a shallow surface
    dry layer that can be easily modified---it extends through the full
    depth of the afternoon mixed layer.

  \item \textbf{Extreme VPD:} Nearly 40\% of the E--W Main transect
    exceeds the 13~hPa extreme VPD threshold at the surface, and 89\%
    exceeds 8~hPa.  The maximum VPD of 17.8~hPa indicates extraordinary
    atmospheric demand for moisture from fuels.

  \item \textbf{Terrain amplification:} The Mogollon Rim, the Chuska
    Mountains, and the numerous mesas and canyons of the Colorado Plateau
    produce local wind acceleration, mountain-wave subsidence, and
    downslope drying that push conditions beyond what the ambient values
    suggest.

  \item \textbf{Steep lapse rates:} Near-surface lapse rates approaching
    the dry adiabatic rate (8--10\textdegree C/km) create an unstable
    boundary layer that promotes erratic fire behavior, rapid plume
    development, and transport of firebrands well ahead of the fire
    front.

  \item \textbf{Failed overnight recovery:} The nocturnal humidity recovery
    to only 35--40\% (vs.\ the typical 70--90\%) indicates that dead
    fine fuels (1-hour and 10-hour) will not recover overnight, entering
    the next burning period with cumulative moisture deficits.

  \item \textbf{Winter fire season dynamics:} The Southwest's dormant
    grass fuels---cured but not yet replaced by green-up---respond rapidly
    to the extreme VPD and moderate winds.  Grass fire rates of spread in
    these conditions can exceed 3~mph, allowing fires to make significant
    runs before suppression resources can respond.
\end{enumerate}

\noindent The cross-section analysis supports maintaining elevated fire
weather concern across the entire Four Corners domain, with the highest
risk concentrated along the Mogollon Rim country of central Arizona, the
AZ--NM border region near the White Mountains and Zuni Mountains, and the
lower-elevation terrain of the San Juan Basin in northwest New Mexico.
While sustained winds do not meet the 25~kt Red Flag criterion, the
combination of extreme VPD, deep dry layer, and steep lapse rates
warrants a ``below criteria'' Red Flag discussion emphasizing the
non-wind components of the fire weather threat.
