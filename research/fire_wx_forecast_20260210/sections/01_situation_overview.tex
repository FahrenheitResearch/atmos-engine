\section{Situational Overview and Current Guidance}
\label{sec:situation}

The Storm Prediction Center (SPC) Day~1 Fire Weather Outlook issued at \textbf{0715~UTC on 9~February 2026} (valid 09/1200Z--10/1200Z) delineates two \textbf{Critical} fire weather areas across the central High Plains, with broad \textbf{Elevated} risk surrounding both corridors.  This section synthesizes the SPC technical discussion, active National Weather Service (NWS) warnings, current fire activity, and antecedent drought and fuel conditions that frame the threat for 10~February 2026.

%----------------------------------------------------------------------
\subsection{SPC Outlook Analysis}
\label{sec:spc_outlook}

Forecaster \textbf{Squitieri} identifies a synoptic pattern dominated by broad upper ridging overspreading the central and eastern CONUS, with a pronounced mid-level impulse poised to crest the ridge over the central Plains.  An elongated surface low developing across the central Plains drives two distinct mechanisms that produce the twin Critical areas.

\subsubsection{Central High Plains Critical Area}

The northern Critical area covers \textbf{far southeastern Wyoming into far western Nebraska}, centered along the Interstate~80 corridor east of the Laramie Range.  Behind a cold-frontal passage, dry northwesterly flow is forecast to produce \textbf{sustained surface winds of 20--25~mph} with relative humidity potentially falling to \textbf{15--20\%}.  The SPC discussion notes model uncertainty regarding the degree of post-frontal airmass dryness but concludes that ``stronger post-cold frontal winds atop drying fuels should compensate to support wildfire spread'' even if humidity does not reach the lowest projections.

The NWS Cheyenne office provides higher-resolution detail, forecasting a \textbf{65--70~kt mid-level jet} developing across the Laramie Range, with \textbf{700~mb winds of 50--55~kt} near the I-80 Summit area driving strong surface momentum transfer.  Afternoon gusts of \textbf{58+~mph} are expected to become ``more frequent and expand to more locations'' across the southeast Wyoming plains.  Minimum relative humidity is forecast in the \textbf{14--20\% range} east of the Laramie Range, with overnight recovery limited to only \textbf{35--40\%}---described as ``very poor'' by the Cheyenne forecast office, indicating that fire weather conditions may persist well beyond the peak afternoon heating window.

\subsubsection{Southern High Plains Critical Area}

The southern Critical area extends from \textbf{far northeastern New Mexico through the Oklahoma and Texas Panhandles, extreme northwestern Oklahoma, and far southwestern Kansas}.  The SPC identifies ``dry downslope flow, in combination with a very deep and dry boundary layer'' as the primary driver, supporting relative humidity values \textbf{as low as 10--15\%} with sustained west-southwesterly surface winds exceeding \textbf{20~mph for several hours}.

The NWS Amarillo discussion underscores the severity of conditions across the Panhandle corridor: afternoon temperatures are expected in the \textbf{mid-to-upper 70s~\textdegree F} (with some locations approaching \textbf{80~\textdegree F}), while RH values are forecast to \textbf{bottom out below 10\%}.  Portions of the Canadian River Valley face RH values potentially dropping to an extreme \textbf{5\%}.  Southwesterly winds of \textbf{20--25~mph sustained with gusts to 35--45~mph} are forecast.

The NWS Albuquerque office reports that \textbf{25--35~kt winds at 700~mb} will align with the eastern Interstate~40 corridor, producing surface winds of \textbf{25--30~kt} along the I-40 corridor and near the Texas/New Mexico border.  Temperatures will surge \textbf{15--22~\textdegree F above climatology} in the eastern zones, causing RH values to ``plummet further.''  The Albuquerque discussion specifically notes that ``the recent fire activity over the past few days suggests fuels are combustible,'' supporting the upgrade from a Fire Weather Watch to a Red Flag Warning.

\subsubsection{Elevated Risk Envelope}

Surrounding both Critical areas, broad Elevated risk extends across the central and southern High Plains, reflecting the regional-scale pattern of anomalous warmth, low humidity, and gusty winds.  NWS Riverton reports that \textbf{elevated fire weather conditions are expected across much of central and southern Wyoming} on Monday, with temperatures running \textbf{20--30~\textdegree F above normal} and widespread gusty winds of \textbf{20--40~mph}.  South Pass faces a \textbf{70\% probability of gusts exceeding 60~mph}, while southern Casper has an \textbf{80\% probability of gusts over 50~mph}.

The SPC Day~2 Outlook (also issued by Squitieri at 0748~UTC) indicates \textbf{no Critical areas} for 10--11~February, suggesting the threat window is concentrated in the Monday afternoon--evening period before the upper impulse translates east and the pattern relaxes.

%----------------------------------------------------------------------
\subsection{NWS Warnings and Watches}
\label{sec:warnings}

As of 09~February 2026, \textbf{thirteen Red Flag Warnings} are active across seven NWS forecast offices spanning six states.  The warnings collectively affect a corridor from southeastern Wyoming through the Nebraska Panhandle, across eastern Colorado, the Texas and Oklahoma Panhandles, southwestern Kansas, and east-central New Mexico.  Key warnings are summarized below.

\begin{itemize}
    \item \textbf{NWS Cheyenne (CYS):} Red Flag Warning for zones 430--437 (southeast Wyoming and the western Nebraska Panhandle) through \textbf{5~PM MST Monday}.  Westerly turning northwest winds \textbf{20--30~mph sustained with gusts to 45~mph}; RH \textbf{13--20\%}.

    \item \textbf{NWS Albuquerque (ABQ):} Red Flag Warning for zones NMZ104, 123, 125, 126 (San Miguel, Guadalupe, Quay, and Curry counties in east-central New Mexico) from \textbf{11~AM to 6~PM MST Monday}.  Southwest winds \textbf{20--25~mph sustained with gusts to 35~mph}; RH as low as \textbf{7\%}.

    \item \textbf{NWS Amarillo (AMA):} Red Flag Warning for Texas and Oklahoma Panhandle counties (TXZ001--013, 016--017, 317; OKZ001--003) from \textbf{noon to 6~PM CST Monday}.  Southwest winds \textbf{20--25~mph with gusts to 40~mph}; RH as low as \textbf{7\%}, with temperatures in the \textbf{70s~\textdegree F}.

    \item \textbf{NWS Pueblo (PUB):} Red Flag Warning for zones COZ221--222, 225, 228--230, 233, 237 (the greater I-25 corridor from Pueblo south to the New Mexico border, including Teller, Fremont, Pueblo, Huerfano, Las Animas, and Baca counties) from \textbf{10~AM to 6~PM MST Monday}.  West winds \textbf{10--20~mph with gusts to 40~mph}; RH as low as \textbf{7--9\%}.

    \item \textbf{NWS Denver/Boulder (BOU):} Red Flag Warning for zones 238, 242 (northeast Larimer and north Weld counties) from \textbf{11~AM to 5~PM MST Monday}.  West winds \textbf{10--20~mph with gusts to 35~mph}; RH as low as \textbf{13\%}.

    \item \textbf{NWS North Platte (LBF):} Red Flag Warning for zones 204, 206, 209 (Nebraska Panhandle and Sandhills) from \textbf{9~AM CST to 6~PM CST Monday}.  West winds \textbf{20--30~mph with gusts to 40~mph}; RH as low as \textbf{14\%}; temperatures up to \textbf{75--76~\textdegree F}.

    \item \textbf{NWS Dodge City (DDC):} Red Flag Warning for southwestern Kansas counties (Stanton, Grant, Haskell, Morton, Stevens, Seward, Meade, Hamilton, Kearny, Finney, Gray, Ford, Clark) from \textbf{noon to 7~PM CST Monday}.  Southwest winds \textbf{20--30~mph with gusts to 40~mph}; RH as low as \textbf{9--12\%}.

    \item \textbf{NWS Norman (OUN):} Red Flag Warning for northwestern Oklahoma (Harper, Ellis, Woodward counties) from \textbf{noon to 7~PM CST Monday}.  South winds \textbf{15--25~mph with gusts to 35~mph}; RH as low as \textbf{14\%}; temperatures up to \textbf{82~\textdegree F}.
\end{itemize}

The geographic breadth of these warnings---spanning from \textbf{41\textdegree N} in Nebraska to \textbf{34\textdegree N} in east-central New Mexico and from \textbf{106\textdegree W} to \textbf{100\textdegree W}---underscores the regional scale of this fire weather event.  The most extreme conditions (RH $\leq$~7\%, gusts $\geq$~40~mph) are concentrated in the two SPC Critical areas: the I-80 corridor of southeast Wyoming/western Nebraska and the Panhandle corridor from northeast New Mexico through the Texas/Oklahoma Panhandles into southwest Kansas.

%----------------------------------------------------------------------
\subsection{Current Fire Activity}
\label{sec:fire_activity}

As of the most recent NIFC Incident Management Situation Report (6~February 2026), the \textbf{National Preparedness Level stands at~1} (the lowest on the 1--5 scale), and fire activity nationwide remains light.  Year-to-date statistics through 6~February show \textbf{3,797 fires for 61,122 acres}, with only \textbf{three active large fires}---all in Florida---totaling \textbf{1,138~acres} and all at 70--80\% containment.  No large fires are being managed under a strategy other than full suppression.

Within the regions of concern for this outlook:

\begin{itemize}
    \item \textbf{Rocky Mountain Area (RMA):}  The RMA has experienced minimal fire activity in January 2026, with \textbf{30 fires totaling approximately 1,400 acres}, all contained by local initial-attack resources.  Wildfire activity has been concentrated in \textbf{southwestern South Dakota and along the Colorado Front Range}, consistent with the exposed fine-fuel areas most susceptible to wind-driven fire spread.

    \item \textbf{Southwest Area:}  A \textbf{25-acre fire} was reported on 1~February 2026 in the Magdalena Mountains on the Magdalena Ranger District of the Cibola National Forest \& National Grasslands, cause unknown.  The NWS Albuquerque forecast discussion notes that ``recent fire activity over the past few days suggests fuels are combustible,'' indicating additional small fires have occurred in eastern New Mexico leading into this event.

    \item \textbf{Front Range Corridor:}  The NIFC monthly outlook identifies the Colorado Front Range as an area of \textbf{above-normal significant fire potential} in February, expected to manifest in ``brief two- or three-day windows'' coinciding with wind events---precisely the pattern materializing on 9--10~February.
\end{itemize}

While the current national fire situation is quiet, the low preparedness level should not obscure the potential for rapid escalation.  The February 2024 precedent of a \textbf{60+~mph wind-driven grass fire near Cheyenne that closed both I-25 and I-80} demonstrates how quickly conditions in this corridor can produce significant incidents with major transportation and community impacts.

%----------------------------------------------------------------------
\subsection{Antecedent Conditions: Drought, Snowpack, and Fuels}
\label{sec:antecedent}

The fire weather threat on 10~February 2026 is underpinned by an exceptional antecedent drought and snowpack deficit across the western United States that has been building since late 2025.

\subsubsection{U.S. Drought Monitor (as of 3~February 2026)}

As of the most recent USDM release, \textbf{37.41\% of the United States} (44.53\% of the Lower 48) is classified in drought.  A mostly dry January led to large drought degradations across the West, with specific conditions in the affected fire weather areas as follows:

\begin{itemize}
    \item \textbf{Southeast Wyoming:}  Severe Drought (\textbf{D2}) expanded from southeastern Wyoming into northeastern Colorado and the Nebraska Panhandle.  Moderate Drought (D1) expanded across much of eastern Wyoming.  Lowland snow cover is at its \textbf{lowest area extent and depth in at least 20 years}.  Five SNOTEL stations report \textbf{record-low snow water equivalent (SWE)}.

    \item \textbf{Colorado:}  Extreme Drought (\textbf{D3}) increased across central Colorado.  The state reports \textbf{record-low statewide average SWE}, with most basins at \textbf{less than 60\% of median}.  \textbf{95\% of Colorado SNOTEL stations} are classified as being in snow drought.

    \item \textbf{New Mexico:}  \textbf{72\% of the state} is under drought conditions with an additional 27\% abnormally dry (as of late January).  All basins report \textbf{less than 50\% of median SWE}: the Upper Rio Grande at 48\%, Rio Grande--Elephant Butte at 40\%, and the Upper Canadian at just \textbf{28\% of median}.  \textbf{81\% of New Mexico SNOTEL stations} are in snow drought.

    \item \textbf{Montana:}  Limited recent precipitation, declining soil moisture, and below-normal streamflows.  \textbf{46\% of SNOTEL stations} report snow drought conditions.  Most basins at \textbf{75--85\% of median SWE}, with several locations approaching record lows.  The state is experiencing a \textbf{top-five warmest winter on record}.

    \item \textbf{Southern/Central Plains:}  Abnormal dryness (D0) and Moderate Drought (D1--D2) expanded across portions of Kansas, with the broader High Plains corridor experiencing continued deterioration.
\end{itemize}

\subsubsection{Snow Drought and Snowpack}

The drought conditions are compounded by a historic snow drought.  Snow cover across the western United States on 1~February 2026 was \textbf{139,322 square miles}---the \textbf{lowest February~1 snow cover in the MODIS satellite record} (since 2001).  January 2026 was exceptionally dry across central and eastern Montana and northwest North Dakota, with little to no precipitation reported since 8~January.  Most western states received \textbf{50\% or less of normal January precipitation}, and above-normal temperatures caused what precipitation did fall at higher elevations to arrive as rain rather than snow.

For the specific fire weather areas, this snowpack deficit translates directly to fire-receptive landscapes: the eastern Wyoming and Colorado plains and foothills are essentially \textbf{snow-free at low to mid elevations}, leaving dormant fine fuels (primarily cured grasslands) fully exposed.  The NIFC monthly outlook notes that ``exposed and dry fine fuels on the eastern plains and foothills continue to be the main concern'' and that ``infrequent snowfalls followed by warm dry periods have provided little relief for fine fuel moisture.''

\subsubsection{Fuel Conditions and Fire Season Outlook}

The NIFC National Significant Wildland Fire Potential Outlook issued 2~February 2026 identifies several critical fuel characteristics:

\begin{itemize}
    \item \textbf{Eastern New Mexico:}  ``Well above normal fine fuel loading,'' particularly across the northeastern plains, as the wind season begins.  Above-normal significant fire potential is expected through March and potentially into April.

    \item \textbf{Colorado Front Range:}  Above-normal significant fire potential forecast for February, occurring during transient wind events.  From February into March, southeastern Colorado and western Kansas are ``likely to experience elevated fire chances like the past two years.''

    \item \textbf{Wyoming--Montana:}  Larger fuels are drier than average but ``not nearing critical levels''; however, the fine-fuel (grass) component in eastern Wyoming is the dominant concern.  The alignment of dry fine fuels and gusty winds ``has been responsible for the fire growth observed so far this year.''

    \item \textbf{Fuel Moisture Classification:}  The SWCC tracks dead fuel moisture in categories from Wet ($\geq$15\%) through Extremely Dry ($\leq$3\%).  Current conditions across eastern New Mexico and the southern High Plains place fine fuels in the \textbf{Dry to Very Dry} categories, consistent with the SPC's characterization of ``continued drying fuels'' and ``fuel desiccation.''
\end{itemize}

NWS Senior Service Hydrologist Andrew Mangham summarized the seasonal outlook: ``Given the fact that the Climate Prediction Center is calling for an overall warm, dry winter the whole way through, we are really getting set up for a particularly dangerous fire season.''

\subsubsection{Synthesis}

The 10~February fire weather event occurs against a backdrop of historically low snowpack, expanding multi-category drought, and dormant fine fuels that have received minimal precipitation relief since early January.  While the national fire posture remains at Preparedness Level~1, the ingredients for rapid fire growth and spread---critically low RH (7--15\%), strong sustained winds (20--30~mph with gusts to 45--60~mph), anomalous warmth (15--30~\textdegree F above normal), and receptive fuels---are converging across a broad swath of the central and southern High Plains.  The SPC Critical risk area encompasses approximately \textbf{55,520 square miles} with a population of \textbf{553,237}, underscoring the potential societal impact of fire ignitions during this event window.

The cross-section analyses that follow will examine the vertical atmospheric structure driving these surface conditions, focusing on boundary-layer depth, momentum transfer mechanisms, and the interaction between terrain-channeled flow and the synoptic-scale pattern.
