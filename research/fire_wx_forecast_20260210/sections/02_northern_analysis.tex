\section{Northern Rockies Critical Area Analysis (Montana--Wyoming)}
\label{sec:northern}

This section presents a detailed mesoscale analysis of the Northern Rockies fire weather environment for February 10, 2026, using HRRR cross-section diagnostics initialized at 06z February 9, 2026. Four transects are examined across the critical area of central Montana and northern Wyoming: an east--west main transect at 47\textdegree N spanning the Continental Divide to the western Dakotas (113\textdegree W to 103\textdegree W), a southern east--west transect at 45\textdegree N crossing southern Montana and northern Wyoming (112\textdegree W to 104\textdegree W), a north--south transect along 108\textdegree W from the Canadian border to central Wyoming (49\textdegree N to 43\textdegree N), and a terrain-following transect from the Bitterroot Range southeast to the Powder River Basin (47.5\textdegree N, 114\textdegree W to 46\textdegree N, 104.5\textdegree W). Focus is placed on forecast hours 36 (18z, 11 AM MST) and 39 (21z, 2 PM MST), corresponding to the period of maximum surface heating and peak fire danger. Temporal evolution sequences from overnight (06z) through late afternoon (00z February 11) provide context for the diurnal fire weather cycle.

%=============================================================================
\subsection{Wind Field Analysis}
\label{sec:northern:wind}

\subsubsection{Surface and Low-Level Winds}

The HRRR wind speed cross-sections reveal a relatively benign surface wind environment across central Montana at peak heating on February 10. Along the E--W Main transect at 47\textdegree N (Figure~\ref{fig:north_ew_main_wind_f39}), surface winds range from 1.3 to 8.8~kt with a mean of 5.7~kt---well below the critical 25~kt fire weather threshold. The predominant surface wind direction is from the west-southwest (mean 238\textdegree), reflecting the prevailing synoptic-scale westerly flow pattern. The strongest surface winds are concentrated along the eastern portion of the transect near 103.7\textdegree W, where terrain transitions from the Montana plains to the more rolling topography near the North Dakota border.

\begin{figure}[htbp]
\centering
\includegraphics[width=\textwidth]{figures/north_ew_main_wind_speed_f39.png}
\caption{HRRR wind speed cross-section along the E--W Main transect (47\textdegree N, 113\textdegree W to 103\textdegree W) valid 21z February 10, 2026 (FHR 39). Wind barbs show flow direction and speed. Pink contour marks the freezing level. The jet stream core exceeds 100~kt above 300~hPa.}
\label{fig:north_ew_main_wind_f39}
\end{figure}

The low-level wind profile shows a gradual increase with altitude. At 850~hPa, maximum winds reach 10.9~kt, strengthening to 17.0~kt at 800~hPa near the eastern end of the transect (104.4\textdegree W). The 700~hPa level exhibits a more pronounced wind maximum of 20.4~kt centered near 108.3\textdegree W, reflecting the developing mid-level flow. This 700~hPa wind maximum is significant because it represents the upper boundary of the afternoon mixed layer and could be entrained to the surface through turbulent mixing during peak heating hours.

\begin{figure}[htbp]
\centering
\includegraphics[width=\textwidth]{figures/north_ew_main_wind_speed_f36.png}
\caption{HRRR wind speed cross-section along the E--W Main transect valid 18z February 10, 2026 (FHR 36, 11 AM MST). Early heating phase shows the beginning of boundary layer development and wind acceleration.}
\label{fig:north_ew_main_wind_f36}
\end{figure}

\subsubsection{Mid- and Upper-Level Flow}

Above 700~hPa, winds increase substantially. The 650~hPa level reaches 25.2~kt near 108\textdegree W, while 600~hPa winds reach 30.6~kt centered at 106.8\textdegree W. At 500~hPa, the flow strengthens to 50.0~kt over the eastern Montana plains (105\textdegree W), accelerating rapidly to 80.5~kt at 400~hPa and 101.9~kt at 300~hPa near 103\textdegree W (Table~\ref{tab:north_wind_profile}). This strong upper-level jet streak positioned over eastern Montana has implications for the fire weather environment: the associated subsidence on the anticyclonic side of the jet entrance region can enhance drying aloft and promote deep mixing of low-humidity air toward the surface.

\begin{table}[htbp]
\centering
\caption{Wind speed profile along the E--W Main transect at 21z February 10, 2026 (FHR 39). Values in knots.}
\label{tab:north_wind_profile}
\begin{tabular}{lcccc}
\hline
Level (hPa) & Min (kt) & Mean (kt) & Max (kt) & Max Location \\
\hline
Surface (1013) & 1.3 & 5.7 & 8.8 & 103.7\textdegree W \\
925 & 1.3 & 5.8 & 9.6 & 103.5\textdegree W \\
850 & 1.8 & 7.4 & 10.9 & 104.3\textdegree W \\
800 & 3.0 & 9.7 & 17.0 & 104.4\textdegree W \\
700 & 9.8 & 15.6 & 20.4 & 108.3\textdegree W \\
650 & 12.3 & 18.4 & 25.2 & 108.0\textdegree W \\
600 & 16.0 & 22.7 & 30.6 & 106.8\textdegree W \\
500 & 24.1 & 37.2 & 50.0 & 105.0\textdegree W \\
400 & 48.9 & 61.0 & 80.5 & 103.0\textdegree W \\
300 & 74.5 & 86.3 & 101.9 & 103.0\textdegree W \\
\hline
\end{tabular}
\end{table}

\subsubsection{Temporal Evolution of Wind}

The diurnal wind evolution (Figure~\ref{fig:north_wind_temporal}) shows a marked progression from overnight quiescence to daytime acceleration. At 06z (FHR 24, overnight), winds are strongest aloft with a well-defined upper-level jet but minimal surface flow, consistent with a stable nocturnal boundary layer. By 15z (FHR 33, 8 AM MST), boundary layer mixing begins to erode the nocturnal inversion, and the mid-level flow starts to strengthen as the upstream trough amplifies. The wind speed maximum at the jet level shifts progressively eastward through the day, reflecting the eastward propagation of the upper-level wave. By 00z February 11 (FHR 42, 5 PM MST), the jet core has moved east of the transect, and surface winds remain light as the boundary layer begins to restabilize.

\begin{figure}[htbp]
\centering
\begin{minipage}{0.48\textwidth}
\includegraphics[width=\textwidth]{figures/north_ew_main_wind_speed_f24.png}
\end{minipage}\hfill
\begin{minipage}{0.48\textwidth}
\includegraphics[width=\textwidth]{figures/north_ew_main_wind_speed_f33.png}
\end{minipage}
\begin{minipage}{0.48\textwidth}
\includegraphics[width=\textwidth]{figures/north_ew_main_wind_speed_f39.png}
\end{minipage}\hfill
\begin{minipage}{0.48\textwidth}
\includegraphics[width=\textwidth]{figures/north_ew_main_wind_speed_f42.png}
\end{minipage}
\caption{Temporal evolution of wind speed along the E--W Main transect from 06z February 10 (top left, FHR 24) through 15z (top right, FHR 33), 21z (bottom left, FHR 39), and 00z February 11 (bottom right, FHR 42). The upper-level jet strengthens and shifts eastward through the period while surface winds remain subdued.}
\label{fig:north_wind_temporal}
\end{figure}

\subsubsection{Terrain Interactions}

The terrain-following transect from the Bitterroot Range (114\textdegree W) to the Powder River Basin (104.5\textdegree W) reveals important orographic wind interactions (Figure~\ref{fig:north_terrain_wind}). Lee-side acceleration is evident downstream of the Rocky Mountain front, with enhanced wind speeds in the 800--700~hPa layer east of the Continental Divide. The complex terrain of western Montana forces channeling of the westerly flow through valleys, creating localized areas of enhanced surface wind where canyon exits open onto the Montana plains. These terrain-channeled winds, while not reaching critical thresholds in this forecast, represent areas where fire spread could be locally enhanced.

\begin{figure}[htbp]
\centering
\includegraphics[width=\textwidth]{figures/north_terrain_wind_speed_f39.png}
\caption{HRRR wind speed cross-section along the terrain-following NW--SE transect (47.5\textdegree N, 114\textdegree W to 46\textdegree N, 104.5\textdegree W) at 21z February 10, 2026. The transect captures the transition from the high Rockies to the Great Plains, showing terrain-modulated wind structures.}
\label{fig:north_terrain_wind}
\end{figure}

%=============================================================================
\subsection{Humidity and Moisture Analysis}
\label{sec:northern:rh}

\subsubsection{Surface Relative Humidity}

The HRRR relative humidity cross-sections reveal an environment that is moderately dry at the surface but with markedly drier conditions aloft (Figure~\ref{fig:north_ew_main_rh_f39}). Along the E--W Main transect at peak fire danger (21z, FHR 39), surface relative humidity ranges from 40.9\% to 65.8\% with a mean of 54.8\%. Critically, \textbf{no grid points along this transect fall below the 15\% or even 20\% RH threshold at the surface}, and the minimum surface RH of 40.9\% occurs near 112\textdegree W on the lee side of the Continental Divide. While these surface values are well above critical fire weather thresholds, the moderate dryness across the western half of the transect (RH 40--50\%) combined with other factors warrants continued monitoring.

\begin{figure}[htbp]
\centering
\includegraphics[width=\textwidth]{figures/north_ew_main_rh_f39.png}
\caption{HRRR relative humidity cross-section along the E--W Main transect at 21z February 10, 2026 (FHR 39). Green shading indicates higher humidity; brown tones indicate drier air. The dashed contours mark key RH thresholds. Note the pronounced dry layer between 650--500~hPa with values below 20\%.}
\label{fig:north_ew_main_rh_f39}
\end{figure}

\subsubsection{Elevated Dry Layer and Entrainment Potential}

The most significant moisture feature is the deep dry layer extending from approximately 700~hPa to 500~hPa. At 700~hPa, 21 of 252 grid points (8.3\%) exhibit RH below 20\%, with a minimum of 19.0\%. The drying intensifies dramatically with altitude: at 650~hPa, 148 points (58.7\%) fall below 20\% RH with a minimum of 13.3\%, and at 600~hPa, 146 points (57.9\%) are below 20\% with a minimum of just 10.4\%. At 500~hPa, 118 points (46.8\%) are below 20\% with an extreme minimum of only 6.5\% (Table~\ref{tab:north_rh_profile}).

\begin{table}[htbp]
\centering
\caption{Relative humidity profile along the E--W Main transect at 21z February 10, 2026 (FHR 39).}
\label{tab:north_rh_profile}
\begin{tabular}{lcccc}
\hline
Level (hPa) & Min (\%) & Mean (\%) & Max (\%) & Points $<$20\% \\
\hline
Surface (1013) & 40.9 & 54.8 & 65.8 & 0 \\
900 & 40.9 & 57.3 & 72.5 & 0 \\
850 & 41.5 & 63.1 & 89.6 & 0 \\
800 & 40.4 & 57.8 & 87.4 & 0 \\
750 & 28.7 & 44.2 & 70.0 & 0 \\
700 & 19.0 & 32.0 & 55.7 & 21 \\
650 & 13.3 & 25.8 & 48.5 & 148 \\
600 & 10.4 & 21.1 & 43.6 & 146 \\
500 & 6.5 & 30.2 & 68.5 & 118 \\
\hline
\end{tabular}
\end{table}

This elevated dry layer is critically important for fire weather: as daytime convective mixing deepens the planetary boundary layer through peak heating, very dry air from the 700--600~hPa level can be entrained downward toward the surface. If the mixed layer top reaches 700~hPa (approximately 3~km AGL over the Montana plains), the resulting surface RH could drop substantially below the current HRRR surface forecast values. This mechanism is a well-documented pathway for rapid fire weather deterioration in the Northern Rockies.

\subsubsection{Spatial Variability of Moisture}

The column-mean RH analysis reveals a pronounced west-to-east moisture gradient. The driest column occurs near 112\textdegree W (lee of the Divide) with a mean tropospheric RH of 41.6\%, while the moistest columns are in eastern Montana near 104\textdegree W (mean 71.9\%). This gradient reflects the orographic drying effect of the Rocky Mountain front and the increasing distance from the Pacific moisture source.

The N--S transect along 108\textdegree W (Figure~\ref{fig:north_ns_rh_f39}) reveals that the driest conditions extend across a broad latitudinal band from approximately 46\textdegree N to 48\textdegree N, centered on central Montana. The southern portion of the transect (northern Wyoming, 43--45\textdegree N) shows slightly higher terrain-level RH values due to the more complex orography of the Absaroka and Bighorn ranges producing localized upslope moisture.

\begin{figure}[htbp]
\centering
\includegraphics[width=\textwidth]{figures/north_ns_rh_f39.png}
\caption{HRRR relative humidity cross-section along the N--S transect (49\textdegree N to 43\textdegree N, 108\textdegree W) at 21z February 10, 2026. The dry layer at 650--500~hPa is prominent across the full latitudinal extent, with the driest air at mid-levels near 46--47\textdegree N.}
\label{fig:north_ns_rh_f39}
\end{figure}

\subsubsection{Diurnal RH Evolution}

The temporal progression of relative humidity (Figure~\ref{fig:north_rh_temporal}) shows the expected diurnal cycle. At 06z (FHR 24, overnight), the surface layer maintains higher humidity under the nocturnal inversion, with the low-level moisture pool extending up to roughly 850~hPa. By 15z (FHR 33, 8 AM MST), surface heating begins to erode this moist boundary layer from below, and the mid-level dry layer begins to deepen. At 21z (FHR 39, peak heating), the boundary layer has mixed sufficiently to draw down some of the drier mid-level air, producing the observed surface RH minimum of 40.9\%. By 00z February 11 (FHR 42), the boundary layer begins to restabilize, but the mid-level dry layer persists, indicating that the elevated dryness is a synoptic-scale feature rather than a diurnally forced one.

\begin{figure}[htbp]
\centering
\begin{minipage}{0.48\textwidth}
\includegraphics[width=\textwidth]{figures/north_ew_main_rh_f24.png}
\end{minipage}\hfill
\begin{minipage}{0.48\textwidth}
\includegraphics[width=\textwidth]{figures/north_ew_main_rh_f33.png}
\end{minipage}
\begin{minipage}{0.48\textwidth}
\includegraphics[width=\textwidth]{figures/north_ew_main_rh_f39.png}
\end{minipage}\hfill
\begin{minipage}{0.48\textwidth}
\includegraphics[width=\textwidth]{figures/north_ew_main_rh_f42.png}
\end{minipage}
\caption{Temporal evolution of relative humidity along the E--W Main transect from 06z February 10 (top left) through 15z (top right), 21z (bottom left), and 00z February 11 (bottom right). The nocturnal moisture pool erodes through the day as boundary layer mixing entrains drier mid-level air toward the surface.}
\label{fig:north_rh_temporal}
\end{figure}

%=============================================================================
\subsection{Temperature Structure and Stability}
\label{sec:northern:temperature}

\subsubsection{Surface Temperature Distribution}

The HRRR temperature analysis at peak heating (21z, FHR 39) indicates surface temperatures ranging from 9.3\textdegree C (48.8\textdegree F) to 12.1\textdegree C (53.8\textdegree F) along the E--W Main transect, with a mean of 10.6\textdegree C (51.0\textdegree F). The warmest surface temperatures occur near 112\textdegree W on the lee side of the Continental Divide, consistent with chinook-type warming (Figure~\ref{fig:north_ew_main_temp_f39}). While these temperatures are moderate for February in central Montana---well above freezing and sufficient to support active fire behavior in cured fuels---they do not reach the extreme values associated with the most critical fire weather events.

\begin{figure}[htbp]
\centering
\includegraphics[width=\textwidth]{figures/north_ew_main_temperature_f39.png}
\caption{HRRR temperature cross-section along the E--W Main transect at 21z February 10, 2026 (FHR 39). Isotherms in \textdegree C. The 0\textdegree C isotherm (freezing level) is near 850~hPa over the western portion and slopes downward toward 900~hPa in the east. Note the warm surface anomaly near 112\textdegree W on the lee of the Continental Divide.}
\label{fig:north_ew_main_temp_f39}
\end{figure}

\subsubsection{Vertical Temperature Profile}

The vertical temperature structure reveals several key features. At 850~hPa, temperatures range from $-$1.9\textdegree C to 2.1\textdegree C, with above-freezing values confined to the western end of the transect where the terrain intersects this pressure level. The 700~hPa temperatures span $-$11.7\textdegree C to $-$8.4\textdegree C, and 500~hPa temperatures range from $-$28.5\textdegree C to $-$26.0\textdegree C (Table~\ref{tab:north_temp_profile}).

\begin{table}[htbp]
\centering
\caption{Temperature profile along the E--W Main transect at 21z February 10, 2026.}
\label{tab:north_temp_profile}
\begin{tabular}{lccc}
\hline
Level & Min (\textdegree C) & Max (\textdegree C) & Range \\
\hline
Surface & 9.3 (48.8\textdegree F) & 12.1 (53.8\textdegree F) & 2.8\textdegree C \\
850 hPa & $-$1.9 & 2.1 & 4.0\textdegree C \\
700 hPa & $-$11.7 & $-$8.4 & 3.3\textdegree C \\
500 hPa & $-$28.5 & $-$26.0 & 2.5\textdegree C \\
\hline
\end{tabular}
\end{table}

The N--S transect (Figure~\ref{fig:north_ns_temp_f39}) shows a modest north-to-south temperature gradient at the surface, with the warmest conditions in the lower-elevation valleys of central Montana (near 46--47\textdegree N) and cooler temperatures at both ends of the transect---in the higher terrain of northern Wyoming (Absaroka/Bighorn ranges) and along the Canadian border.

\begin{figure}[htbp]
\centering
\includegraphics[width=\textwidth]{figures/north_ns_temperature_f39.png}
\caption{HRRR temperature cross-section along the N--S transect (49\textdegree N to 43\textdegree N, 108\textdegree W) at 21z February 10, 2026. The complex terrain of northern Wyoming (right side) elevates the surface into colder air, while the Montana plains (center) experience the warmest surface temperatures.}
\label{fig:north_ns_temp_f39}
\end{figure}

\subsubsection{Lapse Rate and Atmospheric Stability}

The lapse rate cross-sections (Figure~\ref{fig:north_lapse_rate_f39}) reveal near-dry-adiabatic conditions (7--8~\textdegree C/km) in the lowest 1--2~km AGL across much of central Montana during peak heating, with steeper lapse rates ($>$8~\textdegree C/km) developing in the surface layer over the plains east of the Divide. Along the terrain-following transect (Figure~\ref{fig:north_terrain_lapse}), lapse rates exceeding 8~\textdegree C/km are observed near the surface over the eastern Montana grasslands, indicating the development of a superadiabatic surface layer. These steep near-surface lapse rates promote vigorous turbulent mixing and vertical transport of momentum, which can bring stronger winds from aloft down to the surface in gusts.

The 850--700~hPa layer lapse rate (computed from the temperature data: approximately $-$1.9\textdegree C at 850~hPa to $-$8.4\textdegree C at 700~hPa, a difference of 6.5\textdegree C over roughly 1.5~km) yields approximately 4.3~\textdegree C/km, which is conditionally stable. However, the surface-to-700~hPa lapse rate (10.6\textdegree C at the surface to $-$8.4\textdegree C at 700~hPa over roughly 3~km AGL) is approximately 6.3~\textdegree C/km, approaching moist adiabatic values and indicating a moderately unstable lower troposphere.

\begin{figure}[htbp]
\centering
\includegraphics[width=\textwidth]{figures/north_ew_main_lapse_rate_f39.png}
\caption{HRRR lapse rate cross-section along the E--W Main transect at 21z February 10, 2026. Values near or exceeding 8~\textdegree C/km (warm colors, pink-red shading near the surface) indicate superadiabatic conditions promoting vigorous turbulent mixing and potential downward momentum transport.}
\label{fig:north_lapse_rate_f39}
\end{figure}

\begin{figure}[htbp]
\centering
\includegraphics[width=\textwidth]{figures/north_terrain_lapse_rate_f39.png}
\caption{HRRR lapse rate cross-section along the terrain-following NW--SE transect at 21z February 10, 2026. Steep lapse rates ($>$8~\textdegree C/km) are concentrated over the lower-elevation grasslands east of the Rocky Mountain front, where maximum surface heating occurs.}
\label{fig:north_terrain_lapse}
\end{figure}

%=============================================================================
\subsection{Vertical Motion and Subsidence}
\label{sec:northern:omega}

\subsubsection{Large-Scale Vertical Motion Pattern}

The omega (vertical velocity) cross-sections reveal a complex pattern of vertical motion associated with mountain wave activity and synoptic-scale forcing (Figure~\ref{fig:north_ew_main_omega_f39}). The most striking feature is the deep column of alternating ascent and descent extending through the full troposphere, with the most vigorous vertical motions anchored to the terrain of the Rocky Mountain front. Strong upward motion (negative omega, blue shading) is evident over and immediately upstream of the Continental Divide, while compensating subsidence (positive omega, red shading) dominates the lee side.

\begin{figure}[htbp]
\centering
\includegraphics[width=\textwidth]{figures/north_ew_main_omega_f39.png}
\caption{HRRR omega (vertical velocity) cross-section along the E--W Main transect at 21z February 10, 2026. Blue shading indicates upward motion; red shading indicates subsidence. The mountain wave pattern is prominent over the Rockies, with lee-side subsidence extending east to approximately 108\textdegree W. Convective-scale vertical velocities are visible in the boundary layer over the plains.}
\label{fig:north_ew_main_omega_f39}
\end{figure}

The terrain-following transect (Figure~\ref{fig:north_terrain_omega_f39}) provides the clearest view of the mountain wave structure. Deep subsidence extends from the mountaintops downstream for 200--300~km, with the sinking motion penetrating from 500~hPa all the way to the surface in some locations. This lee-side subsidence is a primary mechanism for transporting dry mid-level air toward the surface, and it explains the RH minimum observed near 112\textdegree W on the E--W Main transect. Gravity wave activity produces a series of rotor-like circulations in the 800--700~hPa layer, visible as alternating bands of ascent and descent downstream of the primary mountain barrier.

\begin{figure}[htbp]
\centering
\includegraphics[width=\textwidth]{figures/north_terrain_omega_f39.png}
\caption{HRRR omega cross-section along the terrain-following NW--SE transect at 21z February 10, 2026. The mountain wave response to the westerly flow over the Rockies produces deep lee-side subsidence extending well into the Montana plains. Note the vertically propagating wave pattern with alternating ascent/descent cells.}
\label{fig:north_terrain_omega_f39}
\end{figure}

\subsubsection{North--South Subsidence Variability}

The N--S omega cross-section (Figure~\ref{fig:north_ns_omega_f39}) reveals that the subsidence pattern varies substantially with latitude. The strongest subsidence at 108\textdegree W is concentrated between 46\textdegree N and 48\textdegree N---the latitude band where the terrain gradient between the Rockies and plains is steepest. South of 46\textdegree N, the higher and more continuous terrain of Wyoming produces a different wave response, with stronger low-level upward motion over the Absaroka and Wind River ranges but less organized subsidence over the basins. North of 48\textdegree N, the terrain becomes less abrupt and the wave response weakens, producing weaker subsidence.

\begin{figure}[htbp]
\centering
\includegraphics[width=\textwidth]{figures/north_ns_omega_f39.png}
\caption{HRRR omega cross-section along the N--S transect (108\textdegree W) at 21z February 10, 2026. Vigorous ascent and descent are anchored to the complex terrain, with the strongest subsidence centered on the central Montana plains (46--48\textdegree N).}
\label{fig:north_ns_omega_f39}
\end{figure}

The interplay between synoptic-scale subsidence (associated with the upper-level ridge/jet pattern) and terrain-forced subsidence (mountain waves) creates a reinforcing mechanism: synoptic descent deepens the dry layer aloft while terrain-forced descent transports this dry air toward the surface on the lee side. This coupling is a characteristic feature of Northern Rockies fire weather episodes, though in this case the surface moisture content remains sufficient to prevent extreme fire danger conditions at the surface.

%=============================================================================
\subsection{Vapor Pressure Deficit and Fire Weather Composite}
\label{sec:northern:vpd}

\subsubsection{VPD Analysis}

Vapor pressure deficit (VPD) integrates the combined effects of temperature and humidity into a single metric of atmospheric drying potential that is directly relevant to fuel moisture and fire behavior. Along the E--W Main transect at peak heating (Figure~\ref{fig:north_ew_main_vpd_f39}), surface VPD values range from 4.03~hPa to 8.35~hPa with a mean of 5.83~hPa. The maximum surface VPD of 8.35~hPa occurs at 112\textdegree W, coinciding with the lee-side location of warmest temperatures and lowest relative humidity.

\begin{figure}[htbp]
\centering
\includegraphics[width=\textwidth]{figures/north_ew_main_vpd_f39.png}
\caption{HRRR vapor pressure deficit cross-section along the E--W Main transect at 21z February 10, 2026. VPD values increase with height as temperature and humidity combine to produce a strong moisture gradient. The green shading indicates lower VPD (moister); yellows and browns indicate higher VPD (drier).}
\label{fig:north_ew_main_vpd_f39}
\end{figure}

These surface VPD values are below the 20~hPa extreme threshold and the 15~hPa high-risk threshold---no grid points along the transect exceed either benchmark. However, the 850~hPa VPD reaches 4.16~hPa, and VPD values increase substantially in the mid-troposphere where the dry layer resides. The maximum low-level VPD (surface to 700~hPa) of 8.35~hPa, while elevated, indicates that the atmospheric demand for moisture is moderate. For the February time period, when fine fuels may have accumulated winter moisture, these VPD values suggest a limited but non-negligible potential for fire spread in receptive fuels, particularly in south-facing exposures that have dried most rapidly.

\subsubsection{Fire Weather Composite}

The fire weather composite product integrates wind speed, RH, VPD, temperature, and lapse rate into a single visualization that highlights areas of coincident fire-critical conditions (Figure~\ref{fig:north_firewx_composite}). At 21z on the E--W Main transect, the composite shows a predominantly warm-colored (elevated risk) pattern in the lower and middle troposphere, with the most intense fire weather signal in the 700--500~hPa layer where very low RH combines with stronger winds and steep lapse rates. The cross-hatched pattern in the composite indicates where multiple parameters simultaneously approach or exceed fire weather criteria.

\begin{figure}[htbp]
\centering
\begin{minipage}{0.48\textwidth}
\includegraphics[width=\textwidth]{figures/north_ew_main_fire_wx_f36.png}
\end{minipage}\hfill
\begin{minipage}{0.48\textwidth}
\includegraphics[width=\textwidth]{figures/north_ew_main_fire_wx_f39.png}
\end{minipage}
\caption{HRRR fire weather composite cross-sections along the E--W Main transect at 18z (left, FHR 36) and 21z (right, FHR 39). The composite integrates RH, wind, temperature, VPD, and lapse rate. Cross-hatching indicates areas where multiple parameters approach critical fire weather thresholds simultaneously. Note the intensification of the fire weather signal from 18z to 21z as boundary layer mixing deepens.}
\label{fig:north_firewx_composite}
\end{figure}

At the surface level, the fire weather composite shows a gradient from moderate risk in the western portion (lee of the Divide) to lower risk in the eastern portion where higher RH and lighter winds prevail. The critical observation is that while no single parameter reaches its critical threshold at the surface, the combination of moderate warmth (10--12\textdegree C), reduced RH (41--55\%), steep lapse rates ($>$7~\textdegree C/km), and a deep reservoir of extremely dry air aloft (RH 6--13\% at 600--500~hPa) creates a preconditioned environment. A modest increase in wind speed or additional surface heating could rapidly push conditions toward critical fire weather thresholds.

\subsubsection{Multi-Transect Fire Weather Assessment}

Comparison across all four transects at FHR 39 reveals spatial coherence in the fire weather pattern:

\begin{itemize}
\item \textbf{E--W Main (47\textdegree N):} Moderate surface fire weather risk, primarily driven by lee-side drying west of 108\textdegree W. Maximum risk is in the immediate lee of the Continental Divide near 112\textdegree W.
\item \textbf{E--W Southern (45\textdegree N):} Similar pattern but with higher terrain intercepting the dry layer at lower altitudes. The fire weather composite (Figure~\ref{fig:north_ew_south_firewx}) shows a slightly reduced surface risk compared to the main transect due to higher terrain-induced RH in the complex orography of southern Montana. However, valley floors between mountain ranges could experience locally enhanced drying.
\item \textbf{N--S (108\textdegree W):} The fire weather signal is strongest in the 46--48\textdegree N latitude band (Figure~\ref{fig:north_ns_firewx}), weakening both northward toward the Canadian border and southward into the Wyoming mountains. This confirms that central Montana between the Rocky Mountain front and the Missouri River breaks is the primary area of concern.
\item \textbf{Terrain (NW--SE):} The terrain transect (Figure~\ref{fig:north_terrain_firewx}) shows the fire weather risk increasing downstream along the Rocky Mountain front, with the most intense signal in the 200--500~km range east of the mountain crest where lee-side effects maximize.
\end{itemize}

\begin{figure}[htbp]
\centering
\begin{minipage}{0.48\textwidth}
\includegraphics[width=\textwidth]{figures/north_ew_south_fire_wx_f39.png}
\end{minipage}\hfill
\begin{minipage}{0.48\textwidth}
\includegraphics[width=\textwidth]{figures/north_ns_fire_wx_f39.png}
\end{minipage}
\begin{minipage}{0.48\textwidth}
\includegraphics[width=\textwidth]{figures/north_terrain_fire_wx_f39.png}
\end{minipage}
\caption{Fire weather composite cross-sections at 21z February 10 for the E--W Southern transect (top left), N--S transect (top right), and terrain-following transect (bottom). All three transects confirm the elevated mid-level fire weather signal with the surface risk concentrated in the lee of the Rocky Mountain front.}
\label{fig:north_ew_south_firewx}
\label{fig:north_ns_firewx}
\label{fig:north_terrain_firewx}
\end{figure}

%=============================================================================
\subsection{Summary and Fire Weather Assessment}
\label{sec:northern:summary}

The HRRR cross-section analysis for February 10, 2026 reveals an environment over the Northern Rockies that is \textit{conditionally elevated} for fire weather but does not meet criteria for a classic critical fire weather event. The key findings are:

\begin{enumerate}
\item \textbf{Winds:} Surface winds remain well below critical thresholds (max 8.8~kt vs.\ 25~kt critical), though 700~hPa winds of 20.4~kt indicate potential for downward momentum transport during deep mixing events. The upper-level jet (100+~kt at 300~hPa) is positioned over eastern Montana but its direct surface impact is minimal.

\item \textbf{Humidity:} Surface RH remains above 40\% across the entire transect---significantly above the 15\% critical threshold. However, a deep reservoir of extremely dry air exists at 650--500~hPa (minimum RH 6.5\%), and 58\% of mid-level grid points have RH below 20\%. This dry air mass could be entrained to the surface if mixing depth exceeds forecast expectations.

\item \textbf{Temperature:} Surface temperatures of 9--12\textdegree C (49--54\textdegree F) are above freezing and sufficient to support fire activity in dry fuels. Lee-side warming near the Continental Divide produces the warmest and driest surface conditions.

\item \textbf{VPD:} Surface VPD values of 4--8~hPa are moderate and well below the 20~hPa extreme threshold, indicating limited atmospheric drying demand on fuels.

\item \textbf{Vertical Motion:} Mountain wave--induced subsidence on the lee side of the Rockies transports dry mid-level air toward the surface, concentrated between 46--48\textdegree N. This terrain-forced subsidence reinforces the synoptic-scale descent.

\item \textbf{Lapse Rates:} Near-dry-adiabatic to superadiabatic lapse rates ($>$8~\textdegree C/km) develop in the surface layer during peak heating, promoting vigorous mixing that could bring drier and windier conditions from aloft to the surface.
\end{enumerate}

The primary fire weather concern for February 10 is the juxtaposition of a deep, extremely dry layer aloft (especially 650--500~hPa) with steep lapse rates and mountain wave subsidence that could drive surface drying beyond what the standard model surface fields indicate. While the HRRR surface forecast does not show RH values approaching critical levels, forecasters should be aware that rapid deepening of the mixed layer---especially in the lee of the Continental Divide near 112\textdegree W---could produce brief periods of lower RH and gustier winds at the surface than the deterministic forecast suggests. This is most likely between 19z and 23z (12 PM to 4 PM MST) when surface heating is maximal.

The overall fire weather risk for the Northern Rockies on February 10, 2026 is assessed as \textbf{ELEVATED but SUB-CRITICAL}, with localized areas of enhanced concern in the immediate lee of the Rocky Mountain front across central Montana (approximately 112\textdegree W to 109\textdegree W, 46\textdegree N to 48\textdegree N). No Red Flag Warning criteria are met based on the HRRR deterministic forecast, but a Fire Weather Watch may be warranted for the lee-side areas if observations during the morning transition show lower humidity or stronger winds than forecast.
