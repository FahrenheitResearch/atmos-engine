\section{Forecast Discussion and Operational Implications}
\label{sec:forecast}

\subsection{Synthesis of Cross-Section Analysis}

The three-dimensional atmospheric analysis presented in this report reveals the vertical structure underlying the SPC Critical Fire Weather Outlook for February 10, 2026. Standard two-dimensional forecast products---surface maps, plan-view model output, and point forecasts---convey the horizontal extent and timing of fire weather threats. Cross-section analysis adds a critical third dimension: the depth, altitude, and vertical connectivity of dangerous conditions through the atmospheric column.

Key findings from the cross-section analysis include:

\begin{enumerate}
    \item \textbf{Vertical extent of critical conditions:} The cross-sections reveal whether critical fire weather parameters (low RH, strong winds, atmospheric instability) are confined to a shallow surface layer or extend through a deep atmospheric column. Deep critical layers indicate that turbulent mixing can continuously resupply the surface with dry, windy air from aloft, sustaining extreme fire behavior even as surface conditions might otherwise moderate.

    \item \textbf{Subsidence structure:} The omega (vertical velocity) cross-sections identify where large-scale sinking motion is concentrated. Subsidence adiabatically warms and desiccates the descending air, and its vertical placement determines the depth of the fire weather threat. Subsidence originating above 500~hPa and extending to the surface creates the most extreme conditions.

    \item \textbf{Terrain-atmosphere interactions:} The cross-sections spanning complex terrain reveal how mountains, canyons, and basins modify the flow. Downslope acceleration on the lee of terrain barriers, gap winds through passes, and valley channeling are all visible in the vertical plane but invisible on surface maps.

    \item \textbf{Inversion and mixing analysis:} Temperature cross-sections reveal the presence, altitude, and strength of inversions that cap or release boundary layer turbulence. The erosion of morning inversions through solar heating is a critical precursor to the onset of extreme fire weather, and cross-sections can identify where and when this mixing occurs.
\end{enumerate}

\subsection{Peak Fire Danger Windows}

Based on the temporal evolution analysis (Section~\ref{sec:quantitative}), the peak fire danger windows for each critical area are:

\begin{table}[H]
\centering
\caption{Predicted peak fire danger windows, 10 February 2026.}
\label{tab:peak_windows}
\begin{tabular}{lcc}
\toprule
\textbf{Region} & \textbf{Peak Window (UTC)} & \textbf{Peak Window (Local)} \\
\midrule
Northern Rockies (MT/WY) & 19z--23z & 12pm--4pm MST \\
Four Corners (AZ/NM/CO) & 20z--00z & 1pm--5pm MST \\
\bottomrule
\end{tabular}
\end{table}

The offset between the two regions reflects differences in terrain orientation, elevation, and the timing of maximum insolation-driven boundary layer mixing at each latitude. The Northern Rockies peak earlier due to lower sun angle and faster inversion erosion in the drier air mass.

\subsection{What Cross-Sections Reveal That Surface Maps Cannot}

Traditional fire weather forecasts rely on surface observations, plan-view model fields, and sounding data from widely spaced radiosonde sites. Cross-section analysis fills critical gaps:

\subsubsection*{Gap 1: Continuous Vertical Structure Along the Fire Path}
Radiosondes launch from fixed locations (often hundreds of kilometers from the fire). Cross-sections provide continuous vertical profiles along any desired path, revealing structure between sounding sites.

\subsubsection*{Gap 2: Terrain-Channeled Flow}
Surface wind observations reflect only the lowest few meters. Cross-sections show the full depth of terrain-channeled flow, including elevated jets that can mix to the surface as the boundary layer deepens.

\subsubsection*{Gap 3: Subsidence Identification}
Subsidence is invisible on surface maps but clearly visible in omega cross-sections. Identifying where subsidence originates, how deep it extends, and how it interacts with terrain is critical for understanding fire weather extremity.

\subsubsection*{Gap 4: Moisture Depth}
Surface dew point is a single number. RH and VPD cross-sections reveal whether the dry air extends through the full column or is confined to a shallow layer---a critical distinction for predicting how long extreme conditions will persist.

\subsection{Operational Recommendations}

Based on this analysis, the following operational considerations are highlighted:

\begin{enumerate}
    \item \textbf{New fire starts:} Any new ignitions during the peak windows identified above will face the most challenging suppression conditions. Pre-positioning of suppression resources before the peak window is strongly recommended.

    \item \textbf{Existing fires:} Active fires in or near the critical areas should anticipate rapid spread rates and potential for extreme fire behavior including:
    \begin{itemize}
        \item Long-range spotting driven by strong mid-level winds
        \item Rapid rates of spread on receptive fuels
        \item Potential for fire whirls where convergent flow meets unstable lapse rates
    \end{itemize}

    \item \textbf{Prescribed fire:} All prescribed fire activities in the highlighted regions should be suspended for February 10, 2026.

    \item \textbf{Monitoring:} RAWS stations and fire weather forecasters should monitor for:
    \begin{itemize}
        \item Morning inversion breakup timing (cross-section analysis suggests this is the trigger for rapid deterioration)
        \item Wind shifts associated with the passage of any weak disturbances embedded in the flow
        \item RH recovery failure into the overnight hours, which would extend the critical window
    \end{itemize}
\end{enumerate}

\subsection{Confidence and Uncertainty}

This analysis is based on a single HRRR model cycle (20260209 06z) at forecast hours 24--48. Key uncertainty considerations:

\begin{itemize}
    \item \textbf{Model cycle timing:} Later HRRR cycles (12z, 18z) may adjust the position and intensity of critical parameters. Forecasters should update this analysis as newer cycles become available.
    \item \textbf{HRRR resolution:} At 3-km grid spacing, some terrain-channeling effects in narrow canyons (<3 km width) may be underresolved.
    \item \textbf{Boundary layer mixing:} Models can struggle with the timing of morning inversion erosion, which controls the onset of critical surface conditions. Actual onset may be 1--2 hours earlier or later than modeled.
    \item \textbf{Fire-atmosphere coupling:} Once a fire begins, it modifies the local atmosphere through heat release and pyroconvection. The cross-sections presented here represent the ambient (pre-fire) atmosphere.
\end{itemize}

\medskip
\noindent\textit{This forecast was generated autonomously by an AI research agent using the wxsection.com atmospheric cross-section platform. It supplements but does not replace official NWS/SPC fire weather guidance.}
