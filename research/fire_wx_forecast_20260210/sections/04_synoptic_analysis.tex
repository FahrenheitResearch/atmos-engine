% =============================================================================
% Section 4: Synoptic-Scale Atmospheric Pattern
% Fire Weather Forecast: February 10, 2026
% HRRR 06z cycle cross-section analysis
% =============================================================================

\section{Synoptic-Scale Atmospheric Pattern}
\label{sec:synoptic}

The large-scale atmospheric configuration on February 10, 2026, features a potent
upper-level jet stream driving a progressive trough--ridge couplet across the western
United States. HRRR cross-sections spanning from the Pacific Ocean to the Great Plains
reveal the three-dimensional architecture of this pattern and its role in generating
critical fire weather conditions across the Interior West. The analysis that follows
draws on five continental-scale transects (2,000--2,700~km each) at multiple forecast
hours, providing an unprecedented vertical view of the synoptic forcing.

% -----------------------------------------------------------------------------
\subsection{Upper-Level Flow and Jet Stream Configuration}
\label{sec:synoptic:jet}

The polar jet stream dominates the upper troposphere across the western CONUS on
February 10. Along the 36\textdegree N east--west transect at 18Z (F36, peak heating),
the jet core reaches \textbf{122.5~kt at 250~hPa} near 123\textdegree W---directly
offshore of the central California coast (Figure~\ref{fig:synoptic_ew_south_ws}).
This places the left-exit region of the jet streak over the Great Basin and
northern Arizona/New Mexico, a classic configuration for large-scale subsidence
and surface pressure falls that enhance fire weather potential.

\begin{figure}[htbp]
  \centering
  \includegraphics[width=\textwidth]{figures/synoptic_ew_south_wind_speed_f36.png}
  \caption{Wind speed cross-section along 36\textdegree N from the Pacific (125\textdegree W)
    to the southern Great Plains (95\textdegree W) at 18Z February 10 (F36). The jet
    core exceeds 120~kt at 250~hPa near the California coast. Note the secondary
    jet maximum near 96\textdegree W at 225~hPa (115~kt), indicating a split flow
    pattern. Low-level winds reach 55~kt at 700~hPa near 121\textdegree W.}
  \label{fig:synoptic_ew_south_ws}
\end{figure}

A key feature of this jet configuration is the pronounced \textbf{wind speed minimum}
over the Intermountain West between 115\textdegree W and 110\textdegree W. Along the
southern transect, upper-level winds drop to just 62~kt at 200~hPa near 115\textdegree W
and 67~kt at 150~hPa near 110\textdegree W, compared to 120+ kt on either flank. This
reflects the ridge axis positioned over the Intermountain West, with deceleration in
the upper-level flow as air traverses the ridge crest. A secondary jet maximum of
106--115~kt reappears at 225~hPa east of 100\textdegree W, indicating the entrance
region of the downstream jet streak over the central Plains. This split-flow
configuration channels divergent upper-level flow across the entire fire weather
threat area.

The north--south transect along 110\textdegree W (Figure~\ref{fig:synoptic_ns_west_ws})
reveals the jet stream's latitudinal position. The jet core lies at
\textbf{44.3\textdegree N at 225~hPa with 118~kt}, centered over southern Montana.
This is slightly south of the climatological mean jet position for early February,
indicating an amplified trough--ridge pattern. The strong wind speed gradient between
the jet core and the lower latitudes (winds fall from 118~kt at 44\textdegree N to
under 40~kt by 35\textdegree N) highlights a tight baroclinic zone that separates
cold Canadian air from warmer subtropical air to the south.

\begin{figure}[htbp]
  \centering
  \includegraphics[width=\textwidth]{figures/synoptic_ns_west_wind_speed_f36.png}
  \caption{Wind speed cross-section along 110\textdegree W from southern Canada
    (50\textdegree N) to the US--Mexico border (30\textdegree N) at 18Z February 10.
    The polar jet core peaks at 118~kt near 44\textdegree N at 225~hPa. Low-level
    winds are modest (10--23~kt below 700~hPa) across the Rocky Mountain corridor.}
  \label{fig:synoptic_ns_west_ws}
\end{figure}

The diagonal transect from 40\textdegree N, 125\textdegree W to 45\textdegree N,
100\textdegree W (Figure~\ref{fig:synoptic_diag_ws}) captures the jet streak
structure from its entrance region over the Pacific to the exit region over the
northern Plains. This southwest-to-northeast slice reveals a broad, intense jet
with speeds exceeding 80~kt through a deep layer (300--200~hPa) across the entire
2,000+~km transect. The highest winds (approaching 90~kt) tilt northeastward with
height, consistent with warm advection and a developing trough east of the ridge axis.

\begin{figure}[htbp]
  \centering
  \includegraphics[width=\textwidth]{figures/synoptic_diag_wind_speed_f36.png}
  \caption{Wind speed along the diagonal transect (40\textdegree N, 125\textdegree W
    to 45\textdegree N, 100\textdegree W) at 18Z February 10. This SW--NE slice
    captures the full jet streak from Pacific entrance to Plains exit region.
    Sustained 80+ kt winds through a deep (300--200~hPa) layer drive large-scale
    forcing for subsidence downstream.}
  \label{fig:synoptic_diag_ws}
\end{figure}

The Pacific coastal transect (50\textdegree N to 30\textdegree N along 120\textdegree W;
Figure~\ref{fig:synoptic_ns_pacific_ws}) shows the upstream jet configuration. The
jet core is displaced notably southward at the coast compared to the interior,
with maximum winds centered near 37--40\textdegree N, reflecting the curvature of
the jet around the base of the incoming Pacific trough. This configuration places
the strongest upper-level forcing for ascent just offshore, with divergent flow
(and thus subsidence) downstream across the interior West.

\begin{figure}[htbp]
  \centering
  \includegraphics[width=\textwidth]{figures/synoptic_ns_pacific_wind_speed_f36.png}
  \caption{Wind speed cross-section along the Pacific coast (120\textdegree W,
    50\textdegree N to 30\textdegree N) at 18Z February 10. The jet core sits
    near 37--40\textdegree N, with strong southwesterly flow impinging on the
    California coast. Low-level onshore flow is evident below 800~hPa.}
  \label{fig:synoptic_ns_pacific_ws}
\end{figure}

The northern east--west transect along 47\textdegree N
(Figure~\ref{fig:synoptic_ew_north_ws}) captures the polar jet in the heart
of the baroclinic zone. The jet at this latitude is broader and extends farther
east than the subtropical branch, with 60--80~kt winds spanning 1,500~km from
the Pacific coast to the northern Rockies. This persistent upper-level flow
contributes to sustained downslope wind events along the eastern slopes of the
northern Rockies and creates the dynamic forcing necessary for mountain wave
generation over Montana and Wyoming.

\begin{figure}[htbp]
  \centering
  \includegraphics[width=\textwidth]{figures/synoptic_ew_north_wind_speed_f36.png}
  \caption{Wind speed cross-section along 47\textdegree N from the Pacific
    (125\textdegree W) to the northern Great Plains (95\textdegree W) at 18Z
    February 10. The polar jet is broad, with 60--80~kt winds extending across
    much of the northern tier.}
  \label{fig:synoptic_ew_north_ws}
\end{figure}

% -----------------------------------------------------------------------------
\subsection{Large-Scale Subsidence and the Western Ridge}
\label{sec:synoptic:subsidence}

The vertical motion field reveals the critical role of large-scale subsidence in
creating fire weather conditions. Along the 36\textdegree N transect at 18Z
(Figure~\ref{fig:synoptic_ew_south_omega}), the omega field shows a striking
east--west dipole. West of 120\textdegree W, vigorous ascending motion dominates
the mid-troposphere, associated with the incoming Pacific trough and frontal
forcing. The strongest ascent reaches \textbf{$-$4.88~Pa~s$^{-1}$} (approximately
$-$17.6~hPa~hr$^{-1}$) at 825~hPa near 123\textdegree W, indicating active
precipitation processes along the California coast.

\begin{figure}[htbp]
  \centering
  \includegraphics[width=\textwidth]{figures/synoptic_ew_south_omega_f36.png}
  \caption{Vertical velocity (omega) cross-section along 36\textdegree N at 18Z
    February 10. Blue shading (negative omega) indicates ascent; red/warm shading
    (positive omega) indicates subsidence. Note the strong ascent near the coast
    and the broad region of mid-level subsidence east of 110\textdegree W. Gravity
    wave signatures are visible as alternating ascent/descent columns in the
    upper troposphere over terrain.}
  \label{fig:synoptic_ew_south_omega}
\end{figure}

East of 115\textdegree W, the pattern reverses. Mid-level (500--700~hPa) average
omega becomes positive (subsiding) across the Interior West: +0.176~Pa~s$^{-1}$
at 110\textdegree W, +0.024~Pa~s$^{-1}$ at 105\textdegree W, and
+0.305~Pa~s$^{-1}$ at 100\textdegree W. While these area-averaged values appear
modest, the cross-section reveals intense localized subsidence columns embedded
within the broader pattern, particularly above terrain features where orographic
gravity waves produce organized downward motion extending through the full
tropospheric depth. The maximum subsidence along the southern transect reaches
+2.77~Pa~s$^{-1}$ at 950~hPa near 120\textdegree W, associated with downslope
flow on the lee side of the Coast Ranges.

The north--south omega transect along 110\textdegree W
(Figure~\ref{fig:synoptic_ns_west_omega}) reveals the latitudinal structure of
the subsidence pattern. Mid-level subsidence is most pronounced at higher
latitudes (47--49\textdegree N), where values reach +0.225 to +0.385~Pa~s$^{-1}$
averaged through the 500--700~hPa layer. This is consistent with the ridge axis
position and the post-trough subsidence that characterizes downslope wind events
along the Montana/Wyoming front ranges. At lower latitudes (33--35\textdegree N),
weak ascent prevails in the mid-levels, suggesting that the fire weather threat
in the Southwest arises more from dry air advection and thermal mechanisms than
from dynamic subsidence.

\begin{figure}[htbp]
  \centering
  \includegraphics[width=\textwidth]{figures/synoptic_ns_west_omega_f36.png}
  \caption{Vertical velocity (omega) cross-section along 110\textdegree W at 18Z
    February 10. Mid-level subsidence dominates the northern tier (47--50\textdegree N),
    while weak ascent characterizes the southern latitudes. The omega field is heavily
    modulated by orographic gravity waves across the Rocky Mountain terrain.}
  \label{fig:synoptic_ns_west_omega}
\end{figure}

The diagonal transect omega field (Figure~\ref{fig:synoptic_diag_omega}) provides
a particularly compelling view of the subsidence pattern. This slice from the
Pacific coast to the northern Plains crosses the ridge axis at an oblique angle,
revealing the transition from vigorous coastal ascent to deep subsidence in the
lee of the Cascades and northern Rockies. The alternating columns of ascent and
descent in the upper troposphere (above 500~hPa) are gravity wave signatures
forced by the strong cross-mountain flow, which transport momentum and energy
downward from the jet stream to the surface---a key mechanism for generating
severe downslope windstorms.

\begin{figure}[htbp]
  \centering
  \includegraphics[width=\textwidth]{figures/synoptic_diag_omega_f36.png}
  \caption{Vertical velocity along the diagonal transect (40\textdegree N,
    125\textdegree W to 45\textdegree N, 100\textdegree W) at 18Z February 10.
    Deep gravity wave signatures are evident in the alternating ascent/descent
    columns above the terrain. The low-level subsidence east of the Cascades
    and northern Rockies is a key driver of fire weather conditions.}
  \label{fig:synoptic_diag_omega}
\end{figure}

% -----------------------------------------------------------------------------
\subsection{Moisture Distribution and Dry Air Intrusion}
\label{sec:synoptic:moisture}

The relative humidity cross-sections reveal a deep dry air mass entrenched across
the Interior West, sharply contrasting with moist Pacific air along the coast. At
18Z along 36\textdegree N (Figure~\ref{fig:synoptic_ew_south_rh}), the moisture
distribution shows three distinct regimes:

\begin{enumerate}
  \item \textbf{Moist Pacific air west of 121\textdegree W}: Average RH below
    500~hPa exceeds 55\%, with values reaching 90\% in the lower troposphere near
    the coast. This moisture is associated with the approaching Pacific trough.

  \item \textbf{Transition zone (121--116\textdegree W)}: A sharp drying gradient
    occurs across the California Coast Ranges and into the Great Basin, with average
    RH dropping from 55\% to 25\% over just 400~km. This is the critical fire
    weather boundary.

  \item \textbf{Deep dry layer east of 115\textdegree W}: Average RH below 500~hPa
    falls to 25--30\% across the entire Interior West, with minimum values of
    16--20\% at 550--650~hPa. At 105\textdegree W, the dry layer (RH $<$ 20\%)
    extends from 575~hPa to the tropopause---a \textbf{525~hPa-deep column} of
    extremely dry air overlying the fire weather threat area. At 110\textdegree W,
    the dry column spans 300~hPa in depth.
\end{enumerate}

\begin{figure}[htbp]
  \centering
  \includegraphics[width=\textwidth]{figures/synoptic_ew_south_rh_f36.png}
  \caption{Relative humidity cross-section along 36\textdegree N at 18Z February 10.
    Green shading indicates moist air; brown shading indicates dry air ($<$30\% RH).
    Note the moist plume near the coast and the deep column of dry air ($<$20\% RH)
    east of 115\textdegree W extending from the mid-troposphere to the tropopause.}
  \label{fig:synoptic_ew_south_rh}
\end{figure}

The northern transect along 47\textdegree N (Figure~\ref{fig:synoptic_ew_north_rh})
shows a similar but even more pronounced moisture deprivation. The moist Pacific air
is shallower at higher latitudes, and the dry continental air mass extends closer
to the coast. The entire column east of the Cascade crest exhibits RH values below
30\%, with a near-total absence of mid-level moisture. This deep, dry column is
the hallmark of the post-frontal continental air mass that has swept the northern
Rockies over the preceding 24--48 hours.

\begin{figure}[htbp]
  \centering
  \includegraphics[width=\textwidth]{figures/synoptic_ew_north_rh_f36.png}
  \caption{Relative humidity cross-section along 47\textdegree N at 18Z February 10.
    The dry continental air mass dominates the entire column east of the Cascades,
    with moist Pacific air confined to the coastal zone west of 121\textdegree W.}
  \label{fig:synoptic_ew_north_rh}
\end{figure}

The theta-e cross-sections (equivalent potential temperature) along 36\textdegree N
(Figure~\ref{fig:synoptic_ew_south_thetae}) and along 110\textdegree W
(Figure~\ref{fig:synoptic_ns_west_thetae}) clarify the air mass boundaries.
The 36\textdegree N transect shows relatively high theta-e values (310--320~K) in
the low levels near the Pacific coast, with a sharp drop to 290--300~K across the
Interior West. The tightly packed theta-e contours between 120\textdegree W and
116\textdegree W mark the frontal zone and the boundary between moist Pacific air
and dry continental air. Along 110\textdegree W, theta-e decreases markedly from
south to north, with the strongest gradient between 40\textdegree N and
45\textdegree N---coincident with the jet stream axis and the primary baroclinic
zone.

\begin{figure}[htbp]
  \centering
  \includegraphics[width=\textwidth]{figures/synoptic_ew_south_theta_e_f36.png}
  \caption{Equivalent potential temperature ($\theta_e$) cross-section along
    36\textdegree N at 18Z February 10. The tight $\theta_e$ gradient between
    120\textdegree W and 116\textdegree W marks the air mass boundary between
    moist Pacific and dry continental air.}
  \label{fig:synoptic_ew_south_thetae}
\end{figure}

\begin{figure}[htbp]
  \centering
  \includegraphics[width=\textwidth]{figures/synoptic_ns_west_theta_e_f36.png}
  \caption{Equivalent potential temperature cross-section along 110\textdegree W
    at 18Z February 10. The strong north--south $\theta_e$ gradient between
    40\textdegree N and 45\textdegree N coincides with the polar jet axis and
    separates cold, dry Canadian air from warmer subtropical air.}
  \label{fig:synoptic_ns_west_thetae}
\end{figure}

The potential vorticity (PV) analysis along the southern transect
(Figure~\ref{fig:synoptic_ew_south_pv}) and diagonal transect
(Figure~\ref{fig:synoptic_diag_pv}) reveals the tropopause structure and
stratospheric intrusion potential. The PV field shows a notable depression
of the tropopause (lowering of the 1.5--2~PVU surface) between 110\textdegree W
and 100\textdegree W, consistent with the base of the downstream trough. This
tropopause fold brings stratospheric air---with its characteristically low
humidity and high static stability---closer to the surface. When this
stratospheric dry air is mixed to the surface through turbulent processes
and mountain waves, it contributes to the extremely low surface humidity values
that characterize critical fire weather events.

\begin{figure}[htbp]
  \centering
  \includegraphics[width=\textwidth]{figures/synoptic_ew_south_pv_f36.png}
  \caption{Potential vorticity cross-section along 36\textdegree N at 18Z
    February 10. The tropopause (2~PVU surface) descends from 250~hPa over the
    Pacific to below 300~hPa over the central Rockies, indicating a tropopause
    fold associated with the downstream trough.}
  \label{fig:synoptic_ew_south_pv}
\end{figure}

\begin{figure}[htbp]
  \centering
  \includegraphics[width=\textwidth]{figures/synoptic_diag_pv_f36.png}
  \caption{Potential vorticity cross-section along the diagonal transect at 18Z
    February 10. The tropopause fold is evident as the descent of high-PV
    stratospheric air into the upper troposphere east of the ridge axis.}
  \label{fig:synoptic_diag_pv}
\end{figure}

% -----------------------------------------------------------------------------
\subsection{Temporal Evolution Through February 10}
\label{sec:synoptic:evolution}

The synoptic pattern evolves significantly through the diurnal cycle of February 10.
Cross-sections at 6-hour intervals (06Z, 12Z, 18Z, 00Z Feb 11) reveal the
progressive eastward translation of the trough--ridge couplet and the intensification
of fire weather forcing during the afternoon hours.

\subsubsection*{Jet Stream Progression}

At 06Z February 10 (F24; Figure~\ref{fig:synoptic_ew_south_ws_f24}), the jet
stream along 36\textdegree N is still largely confined to west of 120\textdegree W,
with a more zonal flow pattern across the interior. The jet core is less intense
and positioned farther west than at peak time. By 18Z (F36), the jet has amplified
and the left-exit region has shifted eastward, placing the strongest divergent
forcing directly over the fire weather threat area. By 00Z February 11 (F42;
Figure~\ref{fig:synoptic_ew_south_ws_f42}), the jet core has intensified further
and moved slightly eastward, maintaining strong upper-level forcing into the
overnight hours.

\begin{figure}[htbp]
  \centering
  \includegraphics[width=\textwidth]{figures/synoptic_ew_south_wind_speed_f24.png}
  \caption{Wind speed cross-section along 36\textdegree N at 06Z February 10 (F24).
    The jet stream is confined to the coastal zone, with weaker flow across the
    interior. This represents the pre-event setup.}
  \label{fig:synoptic_ew_south_ws_f24}
\end{figure}

\begin{figure}[htbp]
  \centering
  \includegraphics[width=\textwidth]{figures/synoptic_ew_south_wind_speed_f42.png}
  \caption{Wind speed cross-section along 36\textdegree N at 00Z February 11 (F42).
    The jet core has amplified and shifted eastward, maintaining strong upper-level
    forcing through the evening hours. The low-level jet near the coast remains
    vigorous.}
  \label{fig:synoptic_ew_south_ws_f42}
\end{figure}

Along 110\textdegree W, the jet core migrates southward through the period, from
approximately 46\textdegree N at 06Z to 44\textdegree N by 18Z
(Figure~\ref{fig:synoptic_ns_west_ws_evolution}), consistent with the progressive
southeastward translation of the upper-level trough. This southward shift brings
stronger upper-level flow and associated dynamic forcing over the central and
southern Rockies during the afternoon, coincident with peak surface heating.

\begin{figure}[htbp]
  \centering
  \begin{minipage}[t]{0.48\textwidth}
    \includegraphics[width=\textwidth]{figures/synoptic_ns_west_wind_speed_f24.png}
  \end{minipage}
  \hfill
  \begin{minipage}[t]{0.48\textwidth}
    \includegraphics[width=\textwidth]{figures/synoptic_ns_west_wind_speed_f42.png}
  \end{minipage}
  \caption{Wind speed cross-sections along 110\textdegree W at 06Z February 10
    (left, F24) and 00Z February 11 (right, F42). The jet core migrates southward
    and maintains its intensity through the 24-hour period, sustaining upper-level
    forcing over the Rocky Mountain corridor.}
  \label{fig:synoptic_ns_west_ws_evolution}
\end{figure}

\subsubsection*{Subsidence and Vertical Motion Evolution}

The omega evolution along 36\textdegree N illustrates the intensification of coastal
ascent as the Pacific trough deepens. At 06Z (F24;
Figure~\ref{fig:synoptic_ew_south_omega_f24}), subsidence is already established
east of 110\textdegree W, but the ascent--subsidence couplet is less organized.
At 18Z (F36), the pattern reaches full maturity with strong organized ascent at the
coast and broad subsidence across the interior. By 00Z February 11 (F42;
Figure~\ref{fig:synoptic_ew_south_omega_f42}), the ascent corridor has pushed
inland to approximately 118\textdegree W, suggesting that the leading edge of the
Pacific moisture plume is advancing eastward---potentially modifying fire weather
conditions in the Great Basin by late evening, though the strongest subsidence
persists over the southern Rockies.

\begin{figure}[htbp]
  \centering
  \begin{minipage}[t]{0.48\textwidth}
    \includegraphics[width=\textwidth]{figures/synoptic_ew_south_omega_f24.png}
  \end{minipage}
  \hfill
  \begin{minipage}[t]{0.48\textwidth}
    \includegraphics[width=\textwidth]{figures/synoptic_ew_south_omega_f42.png}
  \end{minipage}
  \caption{Omega cross-sections along 36\textdegree N at 06Z February 10 (left, F24)
    and 00Z February 11 (right, F42). The ascent corridor pushes inland through the
    period while subsidence persists over the interior Rockies.}
  \label{fig:synoptic_ew_south_omega_f24}
\end{figure}

\subsubsection*{Moisture Evolution}

The relative humidity evolution along 36\textdegree N (Figure~\ref{fig:synoptic_rh_evolution})
captures the advancing moisture front. At 06Z (F24), dry air dominates the entire
interior from the coast eastward, with only shallow low-level moisture near the
Pacific. By 18Z (F36), the moist plume has deepened along the coast to 500~hPa but
remains sharply confined west of 121\textdegree W. By 00Z February 11 (F42), moisture
begins to erode into the Great Basin above 700~hPa, but the deep dry column
(RH $<$ 20\%) persists east of 110\textdegree W. This temporal evolution indicates
that while Pacific moisture will eventually spread inland with the trough passage,
critical fire weather conditions persist through the entire afternoon and evening
of February 10 across the central and southern Rockies, the Great Basin, and the
Southwest.

\begin{figure}[htbp]
  \centering
  \begin{minipage}[t]{0.48\textwidth}
    \includegraphics[width=\textwidth]{figures/synoptic_ew_south_rh_f24.png}
  \end{minipage}
  \hfill
  \begin{minipage}[t]{0.48\textwidth}
    \includegraphics[width=\textwidth]{figures/synoptic_ew_south_rh_f42.png}
  \end{minipage}
  \caption{Relative humidity cross-sections along 36\textdegree N at 06Z February 10
    (left, F24) and 00Z February 11 (right, F42). Pacific moisture deepens and
    advances inland through the period, but the deep dry column persists across
    the Rockies and southern Plains throughout.}
  \label{fig:synoptic_rh_evolution}
\end{figure}

% -----------------------------------------------------------------------------
\subsection{Synoptic Mechanisms Driving Critical Fire Weather}
\label{sec:synoptic:mechanisms}

The cross-section analysis reveals four distinct but interconnected mechanisms
through which this synoptic pattern generates critical fire weather conditions
across the western US:

\subsubsection*{Northern Rockies: Chinook/Foehn Wind Setup (MT/WY)}

The pattern over the northern Rockies bears the hallmarks of a classic
\textbf{Chinook (foehn) wind event}. The 110\textdegree W transect shows the jet
stream positioned directly over Montana (44\textdegree N) with 118~kt winds at
225~hPa, providing the upper-level momentum necessary to drive cross-barrier flow.
Mid-level subsidence of +0.225 to +0.385~Pa~s$^{-1}$ (averaged through
500--700~hPa) at 47--49\textdegree N creates a favorable environment for downslope
acceleration. The omega cross-sections reveal organized gravity wave signatures
(alternating ascent/descent columns extending from the surface to 300~hPa) over the
northern Rocky Mountain terrain, confirming that mountain wave activity is actively
transporting jet-level momentum to the surface.

This is not a post-frontal setup for the northern tier; rather, the front has
already passed, leaving a dry, stable continental air mass. The jet stream
overrunning this stable air above the terrain generates the classic ``hydraulic
jump'' Chinook mechanism: air descending the lee slopes is adiabatically warmed
and dried, with relative humidity values plummeting as the air compresses. The
low-level wind profile along 110\textdegree W shows relatively modest surface
winds (18--23~kt at 700~hPa near 47--49\textdegree N), suggesting that the most
intense downslope winds are confined to localized terrain-forced acceleration
zones along the eastern slopes of the Rockies, rather than manifesting as a
broad-scale surface wind event.

\subsubsection*{Southern Rockies and Great Basin: Subsidence Warming and Dry Air Advection}

For the central and southern fire weather zones (Utah, Colorado, New Mexico,
Arizona), the mechanism is distinct from the northern Chinook. The cross-sections
show weaker dynamic subsidence at these latitudes (omega weakly positive or near
zero at 37--41\textdegree N along 110\textdegree W), but an exceptionally deep
dry column. The 525~hPa-deep layer of RH $<$ 20\% at 105\textdegree W indicates
that mid-tropospheric drying extends from 575~hPa to the stratosphere. This
dry air mass has been progressively modified through radiational cooling,
subsidence in the upstream ridge, and entrainment of stratospheric air through
the tropopause fold evident in the PV cross-sections.

The fire weather threat here arises primarily from:
\begin{itemize}
  \item \textbf{Adiabatic descent of dry air}: As the deep dry layer is mixed
    downward through diurnal heating and turbulent processes, surface dew points
    plummet and relative humidity drops to critical levels (10--15\%).
  \item \textbf{Terrain-enhanced gustiness}: While the broad-scale low-level
    wind field is moderate (17--19~kt at 700~hPa), terrain channeling through
    canyons and gaps can locally amplify winds to 40+ kt.
  \item \textbf{Afternoon destabilization}: Solar heating of elevated terrain
    erodes the surface inversion, allowing the dry mid-level air to mix to the
    surface, completing the transition to critical fire weather conditions.
\end{itemize}

\subsubsection*{Southwest (AZ/NM): Subtropical Jet Influence and Deep Drying}

The southern portions of Arizona and New Mexico sit south of the polar jet axis
but under the influence of the \textbf{subtropical jet} and its associated dry
air advection. The 36\textdegree N wind speed transect shows a secondary speed
maximum re-emerging east of 105\textdegree W at 225--250~hPa, consistent with
subtropical jet influence at lower latitudes. The relative humidity data reveal
minimum values of 14--16\% at 550--600~hPa at 100--105\textdegree W, with the
dry layer extending remarkably deep into the lower troposphere.

The temperature cross-sections along 110\textdegree W and 120\textdegree W
(Figures~\ref{fig:synoptic_ns_west_temp} and \ref{fig:synoptic_ns_pacific_temp})
show the strong north--south thermal gradient, with warm air ($>$0\textdegree C at
700~hPa) extending as far north as 35\textdegree N along the Rockies. This
relatively warm mid-level air mass over the Southwest inhibits precipitation
development and promotes efficient evaporation of any residual moisture, further
desiccating fuels. The combination of subtropical jet-driven subsidence, deep
drying, and warm mid-level temperatures creates a fire weather environment in the
Southwest that is fundamentally different from the Chinook-driven threat in the north,
but equally dangerous.

\begin{figure}[htbp]
  \centering
  \begin{minipage}[t]{0.48\textwidth}
    \includegraphics[width=\textwidth]{figures/synoptic_ns_west_temperature_f36.png}
  \end{minipage}
  \hfill
  \begin{minipage}[t]{0.48\textwidth}
    \includegraphics[width=\textwidth]{figures/synoptic_ns_pacific_temperature_f36.png}
  \end{minipage}
  \caption{Temperature cross-sections along 110\textdegree W (left) and
    120\textdegree W (right) at 18Z February 10. The strong north--south thermal
    gradient is evident, with warm mid-level air extending to 35\textdegree N
    along the Rockies. The Pacific coastal transect shows the temperature
    contrast between the maritime air mass and the continental interior.}
  \label{fig:synoptic_ns_west_temp}
  \label{fig:synoptic_ns_pacific_temp}
\end{figure}

% -----------------------------------------------------------------------------
\subsection{Pattern Comparison to Historical Fire Weather Events}
\label{sec:synoptic:historical}

The synoptic configuration on February 10, 2026, shares key characteristics with
several notable historical fire weather events in the western United States. The
combination of a 120+ kt jet streak positioned over the coastal zone, a downstream
ridge over the Great Basin, and an approaching Pacific trough is a well-documented
recipe for severe fire weather across multiple zones simultaneously.

\subsubsection*{Comparison to the January 2025 Los Angeles Windstorm}

The broad synoptic pattern---an amplified trough--ridge couplet with a jet core
exceeding 120~kt near the California coast---recalls the devastating
January 7--8, 2025, Palisades/Eaton fire event. In both cases, the jet stream
left-exit region overlies the fire weather zone, and the offshore pressure gradient
drives strong downslope winds. However, the February 10, 2026, pattern differs in
that the jet core is centered at 36\textdegree N rather than 34\textdegree N, and
the low-level wind maximum (55~kt at 700~hPa) is located farther north. This shifts
the primary downslope wind threat from the transverse ranges of southern California
to the Sierra Nevada and the Coast Ranges of central California, with diminished
(though nonzero) Santa Ana wind potential.

\subsubsection*{Comparison to Northern Rockies Chinook Events}

The 110\textdegree W transect closely resembles the classic Chinook pattern
documented by \citet{brinkmann1974} and more recently during the October 2017
northern Rockies fire event. The jet core at 44\textdegree N with 118~kt winds,
mid-level subsidence at 47--49\textdegree N, and post-frontal dry air mass
are canonical features. What distinguishes the February 2026 event is its
multi-regional scope: the jet stream and ridge configuration simultaneously
create fire weather conditions from Montana to Arizona, a spatial extent that
strains firefighting resource allocation and mutual aid agreements.

\subsubsection*{Climate Context}

A mid-February fire weather event of this magnitude is climatologically unusual
but not unprecedented. The increasing frequency of warm, dry, windy episodes
during the traditional ``off-season'' (November--March) in the western US has been
documented in recent climate analyses. The presence of a 120+ kt jet streak
driving simultaneous fire weather across 5+ western states in February represents
the type of high-impact event that challenges traditional seasonal fire
management paradigms.

\vspace{1em}

\noindent\textbf{Summary.} The HRRR cross-section analysis reveals a synoptic
pattern characterized by (1) a 120+ kt polar jet core at 250~hPa near the
California coast with its left-exit region over the Interior West, (2) organized
mid-level subsidence of +0.2 to +0.4~Pa~s$^{-1}$ across the northern Rockies,
(3) a deep dry column (RH $<$ 20\%) extending 300--525~hPa in depth from the
mid-troposphere to the tropopause across the fire weather zone, and (4) a
tropopause fold bringing stratospheric dry air below 300~hPa east of the ridge
axis. These synoptic-scale processes create the atmospheric environment within
which mesoscale terrain interactions generate critical fire weather conditions at
the surface.
