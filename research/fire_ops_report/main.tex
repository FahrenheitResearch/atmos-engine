\documentclass[11pt, letterpaper]{article}

% --- Encoding and fonts ---
\usepackage[utf8]{inputenc}
\usepackage[T1]{fontenc}
\usepackage{lmodern}

% --- Page layout ---
\usepackage[margin=1in]{geometry}
\usepackage{setspace}
\onehalfspacing

% --- Graphics and color ---
\usepackage{graphicx}
\usepackage[dvipsnames]{xcolor}
\usepackage{subcaption}

% --- Tables ---
\usepackage{booktabs}
\usepackage{array}
\usepackage{tabularx}
\usepackage{adjustbox}

% --- Headers and footers ---
\usepackage{fancyhdr}
\pagestyle{fancy}
\fancyhf{}
\lhead{\small wxsection.com}
\rhead{\small Operational Fire Weather Report: Eastern NM / TX Panhandle}
\cfoot{\thepage}
\renewcommand{\headrulewidth}{0.4pt}

% --- Alert/callout boxes ---
\usepackage{tcolorbox}

% --- Hyperlinks ---
\usepackage[colorlinks=true, linkcolor=blue!70!black, urlcolor=blue!70!black, citecolor=blue!70!black]{hyperref}

% --- Misc ---
\usepackage{enumitem}
\usepackage{parskip}

% --- Title configuration ---
\newcommand{\reporttypecolor}{red!70!black}
\newcommand{\reporttypelabel}{FORECAST}

\title{\textcolor{\reporttypecolor}{\large\textsc{\reporttypelabel}\\[0.3em]\LARGE\textbf{Operational Fire Weather Report: Eastern NM / TX Panhandle}}}
\author{wxsection.com AI Agent Swarm (5 parallel agents)}
\date{February 9, 2026 -- 19:30 UTC}

\begin{document}

\maketitle
\thispagestyle{fancy}

\begin{abstract}
Two active fires are burning under SPC CRITICAL fire weather conditions across eastern New Mexico and the Texas Panhandle on February 9, 2026. A fire along Amarillo Lake near North Hughes Street has forced school evacuations and road closures. A larger fire burns east of Tucumcari, NM in open Quay County terrain. Surface observations show sustained winds of 22 kt gusting to 35 kt (40 mph) with relative humidity as low as 7-11\% and temperatures 25 deg F above normal. The entire troposphere is desiccated with no moisture at any level. HRRR analysis shows a nocturnal low-level jet that will load momentum overnight with NO meaningful RH recovery, creating a secondary extreme fire weather window tomorrow morning. This report was generated autonomously by 5 parallel AI agents using 6 data sources, producing 40+ cross-section products, Google Street View imagery, and GFS extended-range forecasts.
\end{abstract}

\section{Active Fires and Alerts}

Two fires are burning in the SPC Critical fire weather zone:

\begin{enumerate}[leftmargin=2em]
\item \textbf{Amarillo Lake Fire (Amarillo, TX)} --- Fire along Amarillo Lake near North Hughes St / NW 5th Ave. Amarillo PD blocking traffic, North Heights High School students bused to other campuses. Response ongoing.
\item \textbf{Fire East of Tucumcari, NM (Quay County)} --- Larger fire in open terrain east of Tucumcari. Limited reporting; area under active Red Flag Warning with 7--13\% RH and 30~kt gusts.
\end{enumerate}

\subsection{NWS Red Flag Warnings}
Three simultaneous Red Flag Warnings are active:
\begin{itemize}[leftmargin=2em]
\item \textbf{NWS Albuquerque} (NMZ104, 123, 125, 126) --- San Miguel, Guadalupe, Quay, Curry counties. SW 20--25 mph, gusts 35 mph. RH 7--13\%. 11 AM--6 PM MST.
\item \textbf{NWS Amarillo} (20 zones, TX/OK panhandles) --- SW 20--25 mph, gusts \textbf{40 mph}. RH as low as \textbf{7\%}. Temps 70s--80s\,$^\circ$F. ERC 70th--89th percentile (4/5). Until 7 PM CST.
\item \textbf{NWS Lubbock} (Parmer, Castro, Bailey) --- SW 20--25 mph, gusts 35 mph. RH 9\%. Until 7 PM CST.
\end{itemize}

\subsection{SPC Fire Weather Outlook}
Day 1: \textbf{CRITICAL} for eastern NM through TX Panhandle to western KS. Both Tucumcari and Amarillo are squarely within the Critical zone. Day 2: No highlighted areas---this is a sharp one-day event driven by a passing system.

\section{Surface Observations}

METAR observations at 1953 UTC (1:53 PM CST / 12:53 PM MST) reveal textbook critical fire weather conditions across the fire corridor.

\begin{table}[htbp]
\centering
\small
\begin{adjustbox}{max width=\textwidth}
\begin{tabular}{lrrrrrrr}
\toprule
\textbf{Station} & \textbf{Location} & \textbf{Wind} & \textbf{Gust} & \textbf{Temp} & \textbf{Dewpt} & \textbf{RH} & \textbf{Altimeter} \\
\midrule
KTCC & Tucumcari NM & 210/22kt & 30kt (pk 35) & 74F (23C) & 16F (-9C) & \textasciitilde{}11\% & 29.97" \\
KAMA & Amarillo TX & 230/19kt & 32kt (pk 33) & 76F (24C) & 17F (-8C) & \textasciitilde{}10\% & 30.02" \\
KCQC & Clines Corners NM & 300/18kt & 28kt (pk 31) & 60F (16C) & 11F (-12C) & \textasciitilde{}14\% & 30.17" \\
KLBB & Lubbock TX & 220/13kt & 20kt & 74F (23C) & 19F (-7C) & \textasciitilde{}12\% & 30.08" \\
KABQ & Albuquerque NM & 340/4kt & --- & 60F (16C) & 23F (-5C) & \textasciitilde{}24\% & 30.17" \\
\bottomrule
\end{tabular}
\end{adjustbox}
\caption{Surface Observations at 1953 UTC, February 9, 2026}
\label{tab:metar}
\end{table}

\subsection{Key Observational Findings}

\begin{itemize}[leftmargin=2em]
\item \textbf{Peak gusts: 35~kt (40~mph) at Tucumcari at 19:19Z, 33~kt (38~mph) at Amarillo at 19:17Z} --- exceeding Red Flag criteria.
\item \textbf{Temperature-dewpoint spread: 30--33\,$^\circ$C (55--60\,$^\circ$F)} --- extreme atmospheric desiccation.
\item \textbf{Pressure falling rapidly:} KTCC dropped 4~mb in 3 hours; KAMA dropped 3.1~mb/3hr --- deepening trough driving wind acceleration.
\item \textbf{Wind shift approaching from the west:} Clines Corners (105.7$^\circ$W) reports WNW 300$^\circ$ while fire locations show SW 210--230$^\circ$. The trough/cold front is propagating eastward toward the fires.
\item \textbf{Conditions still worsening} --- observations taken early-to-mid afternoon; peak heating 1--2 hours away.
\end{itemize}

\section{HRRR Cross-Section Analysis}

The following cross-sections were generated from the HRRR 3-km model (cycle 20260209 19Z) along a W--E transect at 35.17$^\circ$N from 105$^\circ$W to 102$^\circ$W, cutting through the Sangre de Cristo foothills, across the Canadian Escarpment, through Tucumcari, and onto the High Plains.

\begin{figure}[htbp]
\centering
\begin{subfigure}[t]{0.47\textwidth}
\centering
\includegraphics[width=\textwidth]{figures/tucumcari_omega_WE.png}
\caption{Omega (vertical velocity). Weak subsidence across most of the transect reinforcing dryness.}
\end{subfigure}
\hfill
\begin{subfigure}[t]{0.47\textwidth}
\centering
\includegraphics[width=\textwidth]{figures/tucumcari_isen_rh_WE.png}
\caption{Isentropic RH. Compressed isentropes near terrain indicate downslope acceleration. Dry on all theta surfaces.}
\end{subfigure}
\caption{Vertical velocity and isentropic analysis through the Tucumcari fire zone}
\label{fig:omega-isen}
\end{figure}

\begin{figure}[htbp]
\centering
\includegraphics[width=0.95\textwidth]{figures/tucumcari_firewx_WE.png}
\caption{HRRR Fire Weather Composite, W-E through Tucumcari (35.17N). Deep red/dark red indicates RH below 20\% through the full troposphere. Cross-hatching shows RH below 15\% extending from surface through 700 hPa. Strong SW wind barbs throughout. Terrain drops from \textasciitilde{}2000m (west) to \textasciitilde{}1200m (east) --- classic downslope acceleration zone.}
\label{fig:firewx-we}
\end{figure}

\begin{figure}[htbp]
\centering
\includegraphics[width=0.95\textwidth]{figures/tucumcari_wind_WE.png}
\caption{HRRR Wind Speed, W-E through Tucumcari. Surface winds 15-25 kt across the transect. 700 hPa layer shows 30-40 kt --- the momentum source mixing to the surface. Low-level jet visible at 800-700 hPa. Strongest surface winds in the 150-250 km zone east of the terrain break.}
\label{fig:wind-we}
\end{figure}

\begin{figure}[htbp]
\centering
\includegraphics[width=0.95\textwidth]{figures/tucumcari_rh_WE.png}
\caption{HRRR Relative Humidity, W-E through Tucumcari. The entire atmosphere is bone dry --- uniformly below 20\% RH from surface to 400 hPa. No moisture at any level. This is a deep, well-mixed dry airmass with zero chance of precipitation.}
\label{fig:rh-we}
\end{figure}

\begin{figure}[htbp]
\centering
\includegraphics[width=0.95\textwidth]{figures/tucumcari_isen_wind_WE.png}
\caption{HRRR Isentropic Wind Speed, W-E through Tucumcari. Strong wind maximum at 310-320K theta levels (40+ kt). Momentum transfer from mountain crest to plains visible. Classic downslope windstorm signature with compressed isentropes and accelerating flow.}
\label{fig:isen-wind}
\end{figure}

\begin{figure}[htbp]
\centering
\includegraphics[width=0.95\textwidth]{figures/tucumcari_wind_NS.png}
\caption{HRRR Wind Speed, N-S through fire area at 103.3W (36.5N to 34N). Jet stream directly overhead at 200-300 hPa (80+ kt). Strongest surface winds centered on the 35-36N zone --- right over the fire area. No frontal boundaries providing relief.}
\label{fig:wind-ns}
\end{figure}

\subsection{Tucumcari to Amarillo Corridor}

The 175-km corridor between the two fires shows uniformly critical conditions with no terrain sheltering.

\begin{figure}[htbp]
\centering
\begin{subfigure}[t]{0.47\textwidth}
\centering
\includegraphics[width=\textwidth]{figures/tucumcari_amarillo_firewx.png}
\caption{Fire Weather Composite. Uniformly critical conditions along the entire corridor.}
\end{subfigure}
\hfill
\begin{subfigure}[t]{0.47\textwidth}
\centering
\includegraphics[width=\textwidth]{figures/tucumcari_amarillo_wind.png}
\caption{Wind Speed. 15-25 kt sustained winds with no localized sheltering.}
\end{subfigure}
\caption{Tucumcari to Amarillo transect: the entire corridor is one continuous fire danger zone}
\label{fig:corridor}
\end{figure}

\begin{figure}[htbp]
\centering
\includegraphics[width=0.95\textwidth]{figures/tucumcari_amarillo_temp.png}
\caption{Temperature cross-section, Tucumcari to Amarillo. 0C isotherm near 630 hPa. Surface temperatures 18-24C (65-76F) --- 25F above normal for February.}
\label{fig:corridor-temp}
\end{figure}

\section{HRRR Fire Risk Analysis}

Quantitative fire risk scores from the wxsection.com fire risk engine, based on HRRR data along 100-km transects centered on each fire location.

\begin{table}[htbp]
\centering
\small
\begin{tabular}{lrr}
\toprule
\textbf{Metric} & \textbf{Current (FHR 0)} & \textbf{Peak (FHR +6)} \\
\midrule
Risk Score & 40/100 (MODERATE) & 58/100 (ELEVATED) \\
Min RH & 12.3\% & 6.1\% (below 8\% critical) \\
VPD & 20.6 hPa & 21.4 hPa (extreme) \\
Max Wind & 9.7 kt & 17.8 kt \\
\bottomrule
\end{tabular}
\caption{HRRR Fire Risk Scores: Tucumcari NM}
\label{tab:risk-tucumcari}
\end{table}

\begin{table}[htbp]
\centering
\small
\begin{tabular}{lrr}
\toprule
\textbf{Metric} & \textbf{Current (FHR 0)} & \textbf{Peak (FHR +6)} \\
\midrule
Risk Score & 45/100 (MODERATE) & 44/100 (MODERATE) \\
Min RH & 10.6\% & 10.9\% \\
VPD & 21.5 hPa & 21.5 hPa (extreme) \\
Max Wind & 9.9 kt & 10.2 kt \\
\bottomrule
\end{tabular}
\caption{HRRR Fire Risk Scores: Amarillo TX}
\label{tab:risk-amarillo}
\end{table}

\subsection{National Context}

The NM/TX corridor is the \textbf{\#1 fire risk in the nation} by a wide margin. National scan scores: High Plains South 60, Texas Panhandle 59, Arizona 41, all other regions 7--19 (LOW). Note: HRRR point risk scores understate observed conditions---surface observations show 22G30-35~kt vs model 10~kt mean. The model captures the thermodynamic extremes (RH 6--11\%, VPD 20+ hPa) but underresolves peak gusts.

\section{Temporal Evolution (HRRR FHR 0--18)}

Seven forecast timesteps at 3-hour intervals track the fire weather evolution from this afternoon through tomorrow morning. The critical finding: \textbf{there is no meaningful overnight humidity recovery}.

\begin{figure}[htbp]
\centering
\begin{subfigure}[t]{0.47\textwidth}
\centering
\includegraphics[width=\textwidth]{figures/tucumcari_firewx_fhr00.png}
\caption{FHR 00 (19Z, 1 PM CST): Critical conditions established. Deep cross-hatching.}
\end{subfigure}
\hfill
\begin{subfigure}[t]{0.47\textwidth}
\centering
\includegraphics[width=\textwidth]{figures/tucumcari_firewx_fhr09.png}
\caption{FHR 09 (04Z, 10 PM CST): Winds remain elevated. LLJ developing at 700-800 hPa.}
\end{subfigure}
\caption{Fire weather composite evolution: afternoon through evening}
\label{fig:temporal-1}
\end{figure}

\begin{figure}[htbp]
\centering
\begin{subfigure}[t]{0.47\textwidth}
\centering
\includegraphics[width=\textwidth]{figures/tucumcari_wind_fhr00.png}
\caption{FHR 00: Moderate surface winds, 700 hPa jet mixing down.}
\end{subfigure}
\hfill
\begin{subfigure}[t]{0.47\textwidth}
\centering
\includegraphics[width=\textwidth]{figures/tucumcari_wind_fhr09.png}
\caption{FHR 09: Nocturnal LLJ developing at 700-800 hPa.}
\end{subfigure}

\begin{subfigure}[t]{0.47\textwidth}
\centering
\includegraphics[width=\textwidth]{figures/tucumcari_wind_fhr18.png}
\caption{FHR 18: Peak nocturnal LLJ. Momentum loaded aloft.}
\end{subfigure}
\hfill
\begin{subfigure}[t]{0.47\textwidth}
\centering
\includegraphics[width=\textwidth]{figures/tucumcari_rh_fhr18.png}
\caption{FHR 18: RH still below 25\% in boundary layer at dawn.}
\end{subfigure}
\caption{Wind speed and RH evolution through the overnight period. The nocturnal LLJ concentrates momentum at 700--800 hPa with no boundary layer humidity recovery.}
\label{fig:temporal-wind-grid}
\end{figure}

\begin{figure}[htbp]
\centering
\includegraphics[width=0.95\textwidth]{figures/tucumcari_firewx_fhr18.png}
\caption{FHR 18 (13Z, 7 AM CST Feb 10): Even at dawn, fire wx composite still shows deep red shading and cross-hatching. The atmosphere is primed for another extreme fire weather day as soon as morning heating begins. The nocturnal LLJ is loaded and ready to mix down.}
\label{fig:temporal-fhr18}
\end{figure}

\section{GFS Extended-Range Forecast}

The GFS 0.25$^\circ$ model (cycle 20260209 12Z) provides multi-day context beyond the HRRR 18-hour window.

\begin{figure}[htbp]
\centering
\includegraphics[width=0.85\textwidth]{figures/tucumcari_firewx_fhr06.png}
\caption{GFS Fire Weather Composite, FHR 06 (valid 18Z Feb 9)}
\end{figure}

\begin{figure}[htbp]
\centering
\includegraphics[width=0.85\textwidth]{figures/tucumcari_firewx_fhr24.png}
\caption{GFS Fire Weather Composite, FHR 24 (valid 12Z Feb 10)}
\end{figure}

\begin{figure}[htbp]
\centering
\includegraphics[width=0.85\textwidth]{figures/tucumcari_firewx_fhr30.png}
\caption{GFS Fire Weather Composite, FHR 30 (valid 18Z Feb 10)}
\end{figure}

\begin{figure}[htbp]
\centering
\includegraphics[width=0.85\textwidth]{figures/tucumcari_firewx_fhr48.png}
\caption{GFS Fire Weather Composite, FHR 48 (valid 12Z Feb 11)}
\end{figure}

\begin{figure}[htbp]
\centering
\includegraphics[width=0.85\textwidth]{figures/tucumcari_firewx_fhr72.png}
\caption{GFS Fire Weather Composite, FHR 72 (valid 12Z Feb 12)}
\end{figure}

\begin{figure}[htbp]
\centering
\includegraphics[width=0.85\textwidth]{figures/tucumcari_firewx_fhr96.png}
\caption{GFS Fire Weather Composite, FHR 96 (valid 12Z Feb 13)}
\end{figure}

\begin{figure}[htbp]
\centering
\includegraphics[width=0.85\textwidth]{figures/tucumcari_rh_fhr06.png}
\caption{GFS Relative Humidity, FHR 06 (valid 18Z Feb 9)}
\end{figure}

\begin{figure}[htbp]
\centering
\includegraphics[width=0.85\textwidth]{figures/tucumcari_rh_fhr24.png}
\caption{GFS Relative Humidity, FHR 24 (valid 12Z Feb 10)}
\end{figure}

\begin{figure}[htbp]
\centering
\includegraphics[width=0.85\textwidth]{figures/tucumcari_rh_fhr48.png}
\caption{GFS Relative Humidity, FHR 48 (valid 12Z Feb 11)}
\end{figure}

\begin{figure}[htbp]
\centering
\includegraphics[width=0.85\textwidth]{figures/tucumcari_rh_fhr72.png}
\caption{GFS Relative Humidity, FHR 72 (valid 12Z Feb 12)}
\end{figure}

\begin{figure}[htbp]
\centering
\includegraphics[width=0.85\textwidth]{figures/tucumcari_wind_fhr06.png}
\caption{GFS Wind Speed, FHR 06 (valid 18Z Feb 9)}
\end{figure}

\begin{figure}[htbp]
\centering
\includegraphics[width=0.85\textwidth]{figures/tucumcari_wind_fhr24.png}
\caption{GFS Wind Speed, FHR 24 (valid 12Z Feb 10)}
\end{figure}

\subsection{GFS Animated Evolution}

GFS animated GIFs showing multi-day fire weather evolution are available in the \texttt{gfs/} subdirectory but cannot be embedded in PDF. Files: \texttt{tucumcari\_firewx\_gfs\_evolution.gif}, \texttt{tucumcari\_rh\_gfs\_evolution.gif}

\section{Google Street View Fuel Assessment}

Ground-level imagery from Google Street View provides direct assessment of fuel types, terrain, and built environment in the fire-affected areas. These images show the landscape conditions that fire weather conditions are acting upon.

\begin{figure}[htbp]
\centering
\begin{subfigure}[t]{0.47\textwidth}
\centering
\includegraphics[width=\textwidth]{figures/02_tucumcari_town.jpg}
\caption{02 Tucumcari Town}
\end{subfigure}
\hfill
\begin{subfigure}[t]{0.47\textwidth}
\centering
\includegraphics[width=\textwidth]{figures/04_amarillo_lake_area.jpg}
\caption{04 Amarillo Lake Area}
\end{subfigure}
\caption{Street View imagery: 02 Tucumcari Town and 04 Amarillo Lake Area}
\end{figure}

\begin{figure}[htbp]
\centering
\begin{subfigure}[t]{0.47\textwidth}
\centering
\includegraphics[width=\textwidth]{figures/05_hughes_st_amarillo.jpg}
\caption{05 Hughes St Amarillo}
\end{subfigure}
\hfill
\begin{subfigure}[t]{0.47\textwidth}
\centering
\includegraphics[width=\textwidth]{figures/06_i40_corridor_v2.jpg}
\caption{06 I40 Corridor V2}
\end{subfigure}
\caption{Street View imagery: 05 Hughes St Amarillo and 06 I40 Corridor V2}
\end{figure}

\begin{figure}[htbp]
\centering
\begin{subfigure}[t]{0.47\textwidth}
\centering
\includegraphics[width=\textwidth]{figures/06_i40_corridor_v3.jpg}
\caption{06 I40 Corridor V3}
\end{subfigure}
\hfill
\begin{subfigure}[t]{0.47\textwidth}
\centering
\includegraphics[width=\textwidth]{figures/09_tucumcari_route66.jpg}
\caption{09 Tucumcari Route66}
\end{subfigure}
\caption{Street View imagery: 06 I40 Corridor V3 and 09 Tucumcari Route66}
\end{figure}

\begin{figure}[htbp]
\centering
\begin{subfigure}[t]{0.47\textwidth}
\centering
\includegraphics[width=\textwidth]{figures/10_tucumcari_north.jpg}
\caption{10 Tucumcari North}
\end{subfigure}
\hfill
\begin{subfigure}[t]{0.47\textwidth}
\centering
\includegraphics[width=\textwidth]{figures/11_tucumcari_i40.jpg}
\caption{11 Tucumcari I40}
\end{subfigure}
\caption{Street View imagery: 10 Tucumcari North and 11 Tucumcari I40}
\end{figure}

\begin{figure}[htbp]
\centering
\begin{subfigure}[t]{0.47\textwidth}
\centering
\includegraphics[width=\textwidth]{figures/12_san_jon_nm.jpg}
\caption{12 San Jon Nm}
\end{subfigure}
\hfill
\begin{subfigure}[t]{0.47\textwidth}
\centering
\includegraphics[width=\textwidth]{figures/13_amarillo_fire_zone_w.jpg}
\caption{13 Amarillo Fire Zone W}
\end{subfigure}
\caption{Street View imagery: 12 San Jon Nm and 13 Amarillo Fire Zone W}
\end{figure}

\begin{figure}[htbp]
\centering
\begin{subfigure}[t]{0.47\textwidth}
\centering
\includegraphics[width=\textwidth]{figures/14_amarillo_coulter_dr.jpg}
\caption{14 Amarillo Coulter Dr}
\end{subfigure}
\hfill
\begin{subfigure}[t]{0.47\textwidth}
\centering
\includegraphics[width=\textwidth]{figures/15_amarillo_soncy_rd.jpg}
\caption{15 Amarillo Soncy Rd}
\end{subfigure}
\caption{Street View imagery: 14 Amarillo Coulter Dr and 15 Amarillo Soncy Rd}
\end{figure}

\begin{figure}[htbp]
\centering
\begin{subfigure}[t]{0.47\textwidth}
\centering
\includegraphics[width=\textwidth]{figures/17_bushland_tx.jpg}
\caption{17 Bushland Tx}
\end{subfigure}
\hfill
\begin{subfigure}[t]{0.47\textwidth}
\centering
\includegraphics[width=\textwidth]{figures/19_canyon_tx.jpg}
\caption{19 Canyon Tx}
\end{subfigure}
\caption{Street View imagery: 17 Bushland Tx and 19 Canyon Tx}
\end{figure}

\begin{figure}[htbp]
\centering
\includegraphics[width=0.95\textwidth]{figures/22_east_amarillo_lake.jpg}
\caption{22 East Amarillo Lake}
\end{figure}

\section{Fire Behavior Assessment}

\subsection{Tucumcari Area Fire (Quay County)}
\begin{itemize}[leftmargin=2em]
\item \textbf{Fuel type:} Cured winter grassland on the eastern NM High Plains
\item \textbf{Terrain:} Flat to gently rolling, 1200--1300\,m elevation, no barriers
\item \textbf{Spread potential:} EXTREME --- 11\% RH + 22G30~kt + cured grass = rapid rates
\item \textbf{Direction of spread:} NE to ENE (winds from 210$^\circ$)
\item \textbf{Escape potential:} Very high --- flat terrain, sparse population, limited suppression
\item \textbf{Structure threat:} Scattered ranch structures; I-40 corridor at risk
\end{itemize}

\subsection{Amarillo Lake Fire}
\begin{itemize}[leftmargin=2em]
\item \textbf{Location:} Urban/wildland interface near North Hughes St / NW 5th Ave
\item \textbf{Fuel type:} Grass and brush around lake, urban structures downwind
\item \textbf{Conditions:} 10\% RH, 19G32~kt --- extreme for urban interface fire
\item \textbf{School impact:} North Heights HS evacuated (students bused)
\item \textbf{Traffic:} Hughes St blocked by APD
\end{itemize}

\section{Forecast and Operational Outlook}

\begin{tcolorbox}[colback=red!10, colframe=red!80!black, title={\textbf{WIND SHIFT WARNING}}]
CRITICAL: A cold front arrives around midnight (06Z) with a sharp wind shift to northerly. This could stress the southern flanks of any active fires. Fire behavior may shift dramatically at frontal passage.
\end{tcolorbox}

\subsection{Timeline}
\begin{description}[leftmargin=2em]
\item[NOW through 01Z (7 PM CST):] \textbf{Peak danger.} SW 20--25G35~kt, RH 7--11\%. Boundary layer mixing bringing 700~hPa momentum to surface.
\item[01Z--06Z (evening):] Pre-frontal winds remain elevated. LLJ developing at 700--800~hPa.
\item[~06Z (midnight):] Cold front passage. Wind shift to N/NW. \textbf{Southern fire flanks at risk.}
\item[06Z--13Z (overnight):] \textbf{NO meaningful RH recovery.} Boundary layer stays 10--25\% even overnight. Fuels will not recover moisture. LLJ peaks.
\item[After 16Z Feb 10 (10 AM MST):] \textbf{Next-morning surge risk.} Loaded LLJ mixes down with still-dry column. Secondary extreme window possible.
\item[Day 2--3 (Feb 10--11):] SPC shows no highlighted areas. Pattern breaks.
\item[Mid-week:] Rain and storms return. Genuine relief.
\end{description}

\subsection{Pattern Similarity}
This setup closely resembles the \textbf{Smokehouse Creek Fire} (February 26, 2024) --- Texas Panhandle, 1M+ acres. Same SW wind pattern, extreme low RH, cured winter grass. Key difference: today's ERC is 70th--89th percentile (vs 90th+ for Smokehouse Creek).

\subsection{Seasonal Context (NIFC Monthly Outlook)}
\begin{itemize}[leftmargin=2em]
\item Above-normal fine fuel loading over eastern NM
\item Above-normal significant fire potential through March--April
\item La Ni\~na pattern driving heightened risk
\item NM snowpack at 40--50\% of normal
\item TDEM has pre-deployed state wildfire response resources across western TX
\end{itemize}

\section{Data Sources and Methodology}

This report was generated autonomously by the wxsection.com AI Agent Research Platform using 5 parallel intelligence agents:

\begin{enumerate}[leftmargin=2em]
\item \textbf{SPC/NWS Intelligence Agent} --- SPC fire weather outlooks (Day 1/2), NWS alerts (NM, TX), HRRR fire risk scoring, national fire risk scan
\item \textbf{Surface Observations Agent} --- METAR data from 6 stations with trend analysis, peak gust extraction, pressure tendency
\item \textbf{Cross-Section Agent} --- 10 static HRRR cross-sections (wind, RH, fire wx, omega, temperature, isentropic) across multiple transects
\item \textbf{Temporal Evolution Agent} --- 21 HRRR forecast frames (7 FHRs $\times$ 3 products) + 3 animated GIFs
\item \textbf{Web Intelligence Agent} --- Open-source fire intelligence from 15+ sources including Texas A\&M Forest Service, NIFC situation reports, local news
\end{enumerate}

\textbf{Additional products:} GFS 0.25$^\circ$ extended-range cross-sections, Google Street View fuel assessment imagery.

\textbf{Models:} HRRR 3-km (cycle 20260209 19Z), GFS 0.25$^\circ$ (cycle 20260209 12Z).

\textbf{Platform:} wxsection.com atmospheric cross-section generator with AI agent research tools (MCP server, 22 tools, 6 API endpoints).

\vfill
\begin{center}
\small\textit{Generated by wxsection.com AI Atmospheric Research Agent}
\end{center}

\end{document}