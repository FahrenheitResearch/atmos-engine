\documentclass[11pt, letterpaper]{article}

% --- Encoding and fonts ---
\usepackage[utf8]{inputenc}
\usepackage[T1]{fontenc}
\usepackage{lmodern}

% --- Page layout ---
\usepackage[margin=1in]{geometry}
\usepackage{setspace}
\onehalfspacing

% --- Graphics and color ---
\usepackage{graphicx}
\usepackage[dvipsnames]{xcolor}
\usepackage{subcaption}

% --- Tables ---
\usepackage{booktabs}
\usepackage{array}
\usepackage{tabularx}
\usepackage{adjustbox}

% --- Headers and footers ---
\usepackage{fancyhdr}
\pagestyle{fancy}
\fancyhf{}
\lhead{\small wxsection.com}
\rhead{\small Oklahoma Multi-Fire Operational Intelligence Report}
\cfoot{\thepage}
\renewcommand{\headrulewidth}{0.4pt}

% --- Alert/callout boxes ---
\usepackage{tcolorbox}

% --- Hyperlinks ---
\usepackage[colorlinks=true, linkcolor=blue!70!black, urlcolor=blue!70!black, citecolor=blue!70!black]{hyperref}

% --- Misc ---
\usepackage{enumitem}
\usepackage{parskip}

% --- Title configuration ---
\newcommand{\reporttypecolor}{red!70!black}
\newcommand{\reporttypelabel}{FORECAST}

\title{\textcolor{\reporttypecolor}{\large\textsc{\reporttypelabel}\\[0.3em]\LARGE\textbf{Oklahoma Multi-Fire Operational Intelligence Report}}}
\author{wxsection.com AI Agent Swarm (7 parallel agents)}
\date{February 9, 2026 --- 21:30 UTC (3:30 PM CST)}

\begin{document}

\maketitle
\thispagestyle{fancy}

\begin{abstract}
Multiple wildfire starts across the Oklahoma City metropolitan area and central Oklahoma on February 9, 2026, driven by extreme fire weather conditions. A fire near Newalla, OK (SE of OKC) is among the most significant, with additional starts reported across the region. Surface observations show sustained winds of 15--25 kt with gusts exceeding 30 kt, relative humidity below 15\textbackslash\{\}\%, and temperatures in the 70s°F --- 20--30°F above normal for February. The SPC has issued CRITICAL fire weather conditions covering central and western Oklahoma. This report was generated by a 7-agent AI swarm using HRRR 3-km (21Z cycle), GFS 0.25°, NWS/SPC intelligence, METAR observations, Google Street View imagery, fire risk scoring, and the new isentropic ascent analysis. Products include 70+ cross-sections, animated GIFs, fire risk scores, and ground-level imagery across all affected areas.
\end{abstract}

\section{Active Fires and Situation Overview}

\begin{tcolorbox}[colback=red!10, colframe=red!80!black, title={\textbf{MULTI-FIRE OPERATIONAL ALERT}}]
MULTIPLE FIRE STARTS: Oklahoma City metro area experiencing multiple wildfire ignitions under SPC CRITICAL fire weather conditions. Fire near Newalla, OK (SE of OKC) is active. Winds SW 15--25 kt gusting 30+ kt, RH below 15\textbackslash\{\}\%, temps 70s°F. All fires wind-driven with rapid spread potential.
\end{tcolorbox}

\subsection{Fire Locations}
\begin{itemize}[leftmargin=2em]
\item \textbf{Newalla Fire:} Near Newalla, OK (35.36$^\circ$N, 97.18$^\circ$W) --- SE of Oklahoma City, semi-rural area
\item \textbf{Additional OKC-area starts:} Multiple ignitions reported across the metro under extreme conditions
\item \textbf{Fire spread direction:} SW winds driving fires NE --- toward populated areas
\item \textbf{Terrain:} Rolling grassland with scattered timber, rural-to-suburban transition zones
\end{itemize}

\subsection{NWS/SPC Fire Weather Alerts}
\begin{itemize}[leftmargin=2em]
\item \textbf{SPC Day 1:} CRITICAL fire weather --- Southern High Plains / OK-TX Panhandle
\item \textbf{SPC Day 1:} ELEVATED fire weather --- broader Southern Plains including central OK
\item \textbf{Red Flag Warning} (NWS Amarillo) --- OK Panhandle: Cimarron, Texas, Beaver counties
\item \textbf{Red Flag Warning} (NWS Norman) --- W OK: Roger Mills, Harper, Ellis, Woodward
\item \textbf{Red Flag Warning} (NWS Dodge City) --- SW Kansas (13 counties)
\item \textbf{Newalla/OKC:} NOT in formal RFW area but NWS Norman flags ``elevated to near-critical'' conditions extending into central Oklahoma
\item \textbf{Key conditions:} RH 7\% (panhandle) to 11--15\% (OKC), winds SW 15--25 mph gusting 35--40 mph, temps 78--82°F
\item \textbf{ERC:} 70th--89th percentile, Fire Environment 6/10
\item \textbf{SPC Day 2:} Cold front brings relief --- NO critical areas forecast Tuesday
\item \textbf{CRITICAL CONCERN:} Wind shift from approaching cold front later this evening
\end{itemize}


\section{Current Surface Observations}

METAR observations from Oklahoma stations nearest the fire areas. Key parameters
for fire weather: wind speed/gusts, relative humidity, temperature, and visibility.

\begin{table}[htbp]
\centering
\small
\begin{adjustbox}{max width=\textwidth}
\begin{tabular}{lrrrrrr}
\toprule
\textbf{Station} & \textbf{Time} & \textbf{Wind (kt)} & \textbf{T°F} & \textbf{Td°F} & \textbf{RH} & \textbf{Vis (SM)} \\
\midrule
KOKC (OKC) & 2152Z & 190/13G22 & 80 & 39 & \textbackslash\{\}textbf\{14\textbackslash\{\}\%\} & 10+ \\
KTIK (Tinker) & 2055Z & 220/12 & 81 & 40 & \textbackslash\{\}textbf\{15\textbackslash\{\}\%\} & 10+ \\
KPWA (Wiley Post) & 2153Z & 210/14G20 & 80 & 36 & \textbackslash\{\}textbf\{11\textbackslash\{\}\%\} & 10+ \\
KOUN (Norman) & 2145Z & 170/11G16 & 81 & 39 & \textbackslash\{\}textbf\{15\textbackslash\{\}\%\} & 10+ \\
KGOK (Guthrie) & 2153Z & 200/14 & 81 & 39 & \textbackslash\{\}textbf\{13\textbackslash\{\}\%\} & 10+ \\
KSWO (Lawton) & 2153Z & 210/13 & 81 & 36 & \textbackslash\{\}textbf\{11\textbackslash\{\}\%\} & 10+ \\
KTUL (Tulsa) & 2153Z & 180/10 & 79 & 40 & 17\textbackslash\{\}\% & 10+ \\
KCSM (Clinton) & 2153Z & 190/16G25 & 78 & 35 & \textbackslash\{\}textbf\{12\textbackslash\{\}\%\} & 10+ \\
KPNC (Ponca City) & 2153Z & 200/10G21 & 79 & 40 & 17\textbackslash\{\}\% & 10+ \\
KFSM (Ft Smith) & 2153Z & 040/10 & 76 & 48 & 22\textbackslash\{\}\% & 10+ \\
\bottomrule
\end{tabular}
\end{adjustbox}
\caption{METAR observations at 2153 UTC (3:53 PM CST), February 9, 2026. Winds in kt (dir/spd or dir/spdGgust). Bold RH = critical (\$<\$15\textbackslash\{\}\%).}
\label{tab:metar}
\end{table}

\subsection{Key Observations}


\begin{itemize}[leftmargin=2em]
\item \textbf{ALL stations 11--15\% RH} in the OKC metro area --- uniformly critical fire weather
\item \textbf{Temperatures 78--81°F} --- 20--30°F above February normal (unprecedented)
\item \textbf{KPWA lowest RH at 11\%} with gusty 14G20kt --- extremely dangerous
\item \textbf{KCSM (W OK) strongest gusts} at 25kt --- fire weather worsens westward
\item \textbf{KFSM (Ft Smith AR)} higher moisture (22\% RH, 48°F dewpoint) marks eastern edge
\item \textbf{Altimeters falling} (29.92--29.94) --- approaching trough/cold front
\item \textbf{Wind backing from S to SSW} across the metro --- consistent with approaching front
\end{itemize}

\section{HRRR Cross-Section Analysis: Newalla Fire Area}

High-resolution (3-km) HRRR cross-sections through the Newalla fire area,
showing the vertical atmospheric structure driving fire behavior. Multiple transects
capture the fire environment from different angles.

\subsection{Fire Weather Composite}


The fire weather composite shows RH shading (red = critically dry) with wind speed
contours and cross-hatched zones where RH $<$ 15\% AND wind $>$ 25 kt --- the most
dangerous fire weather conditions.

\begin{figure}[htbp]
\centering
\begin{subfigure}[t]{0.47\textwidth}
\centering
\includegraphics[width=\textwidth]{figures/newalla_we_fire_wx.png}
\caption{W--E through Newalla fire}
\end{subfigure}
\hfill
\begin{subfigure}[t]{0.47\textwidth}
\centering
\includegraphics[width=\textwidth]{figures/newalla_swne_fire_wx.png}
\caption{SW--NE along wind/spread axis}
\end{subfigure}

\begin{subfigure}[t]{0.47\textwidth}
\centering
\includegraphics[width=\textwidth]{figures/okc_ns_fire_wx.png}
\caption{N--S through OKC metro}
\end{subfigure}
\hfill
\begin{subfigure}[t]{0.47\textwidth}
\centering
\includegraphics[width=\textwidth]{figures/oklahoma_we_fire_wx.png}
\caption{Broad Oklahoma W--E transect}
\end{subfigure}
\caption{Fire weather composite cross-sections through Oklahoma fire areas}
\label{fig:fire_wx_grid}
\end{figure}

\begin{figure}[htbp]
\centering
\includegraphics[width=0.95\textwidth]{figures/seok_fire_wx.png}
\caption{SE Oklahoma fire corridor}
\end{figure}

\subsection{Wind Speed Analysis}


Wind speed cross-sections showing the boundary layer wind structure, low-level jet
position, and surface wind maxima driving fire spread.

\begin{figure}[htbp]
\centering
\begin{subfigure}[t]{0.47\textwidth}
\centering
\includegraphics[width=\textwidth]{figures/newalla_we_wind.png}
\caption{W--E wind speed through Newalla}
\end{subfigure}
\hfill
\begin{subfigure}[t]{0.47\textwidth}
\centering
\includegraphics[width=\textwidth]{figures/newalla_swne_wind.png}
\caption{SW--NE wind along spread axis}
\end{subfigure}

\begin{subfigure}[t]{0.47\textwidth}
\centering
\includegraphics[width=\textwidth]{figures/okc_ns_wind.png}
\caption{N--S wind through OKC metro}
\end{subfigure}
\hfill
\begin{subfigure}[t]{0.47\textwidth}
\centering
\includegraphics[width=\textwidth]{figures/oklahoma_we_wind.png}
\caption{Broad Oklahoma W--E wind}
\end{subfigure}

\begin{subfigure}[t]{0.47\textwidth}
\centering
\includegraphics[width=\textwidth]{figures/seok_wind.png}
\caption{SE Oklahoma corridor wind}
\end{subfigure}
\caption{Wind speed cross-sections showing boundary layer wind structure}
\label{fig:wind_grid}
\end{figure}

\subsection{Relative Humidity Structure}


RH cross-sections reveal the vertical extent of the dry column.
Values below 15\% (red flag threshold) shown across the depth of the boundary layer.

\begin{figure}[htbp]
\centering
\begin{subfigure}[t]{0.47\textwidth}
\centering
\includegraphics[width=\textwidth]{figures/newalla_we_rh.png}
\caption{W--E RH through Newalla}
\end{subfigure}
\hfill
\begin{subfigure}[t]{0.47\textwidth}
\centering
\includegraphics[width=\textwidth]{figures/newalla_swne_rh.png}
\caption{SW--NE RH along spread axis}
\end{subfigure}

\begin{subfigure}[t]{0.47\textwidth}
\centering
\includegraphics[width=\textwidth]{figures/okc_ns_rh.png}
\caption{N--S RH through OKC}
\end{subfigure}
\hfill
\begin{subfigure}[t]{0.47\textwidth}
\centering
\includegraphics[width=\textwidth]{figures/oklahoma_we_rh.png}
\caption{Broad Oklahoma W--E RH}
\end{subfigure}

\begin{subfigure}[t]{0.47\textwidth}
\centering
\includegraphics[width=\textwidth]{figures/seok_rh.png}
\caption{SE Oklahoma corridor RH}
\end{subfigure}
\caption{Relative humidity cross-sections showing dry column depth}
\label{fig:rh_grid}
\end{figure}

\subsection{Isentropic Ascent Analysis}


The new isentropic ascent product shows RH shading with omega (vertical velocity)
contours: blue = ascent, red dashed = descent. Purple zones highlight moist ascent
regions (RH $>$ 70\% with upward motion). Thin colored contours show the
geostrophic wind perpendicular to the section (jet position). This analysis reveals
whether any moisture return or precipitation is possible to moderate fire conditions.

\begin{figure}[htbp]
\centering
\begin{subfigure}[t]{0.47\textwidth}
\centering
\includegraphics[width=\textwidth]{figures/newalla_we_isentropic_ascent.png}
\caption{W--E isentropic ascent through Newalla}
\end{subfigure}
\hfill
\begin{subfigure}[t]{0.47\textwidth}
\centering
\includegraphics[width=\textwidth]{figures/newalla_swne_isentropic_ascent.png}
\caption{SW--NE isentropic ascent along fire axis}
\end{subfigure}

\begin{subfigure}[t]{0.47\textwidth}
\centering
\includegraphics[width=\textwidth]{figures/okc_ns_isentropic_ascent.png}
\caption{N--S isentropic ascent through OKC metro}
\end{subfigure}
\hfill
\begin{subfigure}[t]{0.47\textwidth}
\centering
\includegraphics[width=\textwidth]{figures/oklahoma_we_isentropic_ascent.png}
\caption{Broad Oklahoma W--E isentropic ascent}
\end{subfigure}

\begin{subfigure}[t]{0.47\textwidth}
\centering
\includegraphics[width=\textwidth]{figures/seok_isentropic_ascent.png}
\caption{SE Oklahoma corridor isentropic ascent}
\end{subfigure}
\caption{Isentropic ascent analysis showing vertical motion and moisture interaction}
\label{fig:isen_grid}
\end{figure}

\subsection{Vapor Pressure Deficit}


VPD measures the drying power of the atmosphere --- higher values indicate more
aggressive fuel desiccation. Values above 30 hPa indicate extreme fire danger.

\begin{figure}[htbp]
\centering
\begin{subfigure}[t]{0.47\textwidth}
\centering
\includegraphics[width=\textwidth]{figures/newalla_we_vpd.png}
\caption{W--E VPD through Newalla}
\end{subfigure}
\hfill
\begin{subfigure}[t]{0.47\textwidth}
\centering
\includegraphics[width=\textwidth]{figures/newalla_swne_vpd.png}
\caption{SW--NE VPD along fire axis}
\end{subfigure}

\begin{subfigure}[t]{0.47\textwidth}
\centering
\includegraphics[width=\textwidth]{figures/okc_ns_vpd.png}
\caption{N--S VPD through OKC metro}
\end{subfigure}
\hfill
\begin{subfigure}[t]{0.47\textwidth}
\centering
\includegraphics[width=\textwidth]{figures/oklahoma_we_vpd.png}
\caption{Broad Oklahoma W--E VPD}
\end{subfigure}

\begin{subfigure}[t]{0.47\textwidth}
\centering
\includegraphics[width=\textwidth]{figures/seok_vpd.png}
\caption{SE Oklahoma corridor VPD}
\end{subfigure}
\caption{Vapor pressure deficit showing atmospheric drying intensity}
\label{fig:vpd_grid}
\end{figure}

\subsection{Additional Atmospheric Products}


Temperature, omega (vertical velocity), dewpoint depression, lapse rate, theta-e,
and moisture transport provide deeper insight into the fire environment.

\begin{figure}[htbp]
\centering
\begin{subfigure}[t]{0.47\textwidth}
\centering
\includegraphics[width=\textwidth]{figures/newalla_we_temp.png}
\caption{Temperature (Newalla W--E)}
\end{subfigure}
\hfill
\begin{subfigure}[t]{0.47\textwidth}
\centering
\includegraphics[width=\textwidth]{figures/newalla_we_omega.png}
\caption{Omega/vertical velocity (Newalla W--E)}
\end{subfigure}

\begin{subfigure}[t]{0.47\textwidth}
\centering
\includegraphics[width=\textwidth]{figures/newalla_we_dewpoint_dep.png}
\caption{Dewpoint depression (Newalla W--E)}
\end{subfigure}
\hfill
\begin{subfigure}[t]{0.47\textwidth}
\centering
\includegraphics[width=\textwidth]{figures/newalla_we_lapse.png}
\caption{Lapse rate (Newalla W--E)}
\end{subfigure}
\caption{Additional atmospheric products (1--4)}
\label{fig:extra_0}
\end{figure}

\begin{figure}[htbp]
\centering
\begin{subfigure}[t]{0.47\textwidth}
\centering
\includegraphics[width=\textwidth]{figures/newalla_we_theta_e.png}
\caption{Theta-e (Newalla W--E)}
\end{subfigure}
\hfill
\begin{subfigure}[t]{0.47\textwidth}
\centering
\includegraphics[width=\textwidth]{figures/newalla_we_moisture.png}
\caption{Moisture transport (Newalla W--E)}
\end{subfigure}

\begin{subfigure}[t]{0.47\textwidth}
\centering
\includegraphics[width=\textwidth]{figures/oklahoma_we_temp.png}
\caption{Temperature (Oklahoma W--E)}
\end{subfigure}
\hfill
\begin{subfigure}[t]{0.47\textwidth}
\centering
\includegraphics[width=\textwidth]{figures/oklahoma_we_omega.png}
\caption{Omega (Oklahoma W--E)}
\end{subfigure}
\caption{Additional atmospheric products (5--8)}
\label{fig:extra_4}
\end{figure}

\begin{figure}[htbp]
\centering
\begin{subfigure}[t]{0.47\textwidth}
\centering
\includegraphics[width=\textwidth]{figures/oklahoma_we_dewpoint_dep.png}
\caption{Dewpoint depression (Oklahoma W--E)}
\end{subfigure}
\hfill
\begin{subfigure}[t]{0.47\textwidth}
\centering
\includegraphics[width=\textwidth]{figures/oklahoma_we_lapse.png}
\caption{Lapse rate (Oklahoma W--E)}
\end{subfigure}

\begin{subfigure}[t]{0.47\textwidth}
\centering
\includegraphics[width=\textwidth]{figures/oklahoma_we_theta_e.png}
\caption{Theta-e (Oklahoma W--E)}
\end{subfigure}
\hfill
\begin{subfigure}[t]{0.47\textwidth}
\centering
\includegraphics[width=\textwidth]{figures/oklahoma_we_isentropic_ascent.png}
\caption{Isentropic ascent (Oklahoma W--E)}
\end{subfigure}
\caption{Additional atmospheric products (9--12)}
\label{fig:extra_8}
\end{figure}

\section{Fire Risk Scoring}

Quantitative fire risk scores (0--100) computed from HRRR atmospheric data
along each transect. Scores integrate wind speed, RH, VPD, temperature anomaly,
and fuel moisture indicators. Scores above 50 indicate HIGH risk; above 70 is EXTREME.

\subsection{HRRR Fire Risk Scores}


Fire risk scores computed from HRRR cross-section data along each transect.
Note: Scores integrate the full vertical column, which can dilute extreme
surface conditions (e.g., surface RH of 11--15\% averaged with 40--50\% RH
aloft produces misleadingly low scores). Surface observations confirm
conditions are far more dangerous than column-averaged scores suggest.

\begin{table}[htbp]
\centering
\small
\begin{adjustbox}{max width=\textwidth}
\begin{tabular}{lrrrrr}
\toprule
\textbf{Transect} & \textbf{Score} & \textbf{Level} & \textbf{RH Range (\textbackslash\{\}\%)} & \textbf{Max Wind (m/s)} & \textbf{Factors} \\
\midrule
Newalla W-E Transect & 11 & LOW & 43--56 & 0.6 & Elevated VPD: 11.82 hPa \\
OKC Metro N-S Transect & 10 & LOW & 48--57 & 0.6 & Elevated VPD: 10.81 hPa \\
Broader Oklahoma W-E Transect & 13 & LOW & 41--56 & 2.8 & Elevated VPD: 11.49 hPa \\
\bottomrule
\end{tabular}
\end{adjustbox}
\caption{HRRR fire risk scores by transect. Scores 0--100 (column-averaged). Surface conditions are significantly worse than column scores indicate.}
\label{tab:hrrr_risk}
\end{table}

\subsection{GFS Extended Fire Risk}


GFS-based fire risk scores for extended outlook. Same column-averaging
caveats apply --- surface conditions are more extreme than scores suggest.

\begin{table}[htbp]
\centering
\small
\begin{tabular}{lrrrr}
\toprule
\textbf{Transect} & \textbf{Score} & \textbf{Level} & \textbf{RH Range (\textbackslash\{\}\%)} & \textbf{Factors} \\
\midrule
GFS Fire Risk Scores — Oklahoma Fire Weather & ? & ? & ? &  \\
NEWALLA TRANSECT & 8 & LOW & 45--48 & Elevated VPD: 9.49 hPa \\
BROADER OKLAHOMA TRANSECT & 12 & LOW & 36--51 & Elevated VPD: 11.24 hPa \\
OKLAHOMA REGIONAL BOX & 9 & LOW & 42--52 & Elevated VPD: 9.95 hPa \\
\bottomrule
\end{tabular}
\caption{GFS extended fire risk scores by transect.}
\label{tab:gfs_risk}
\end{table}

\section{Temporal Evolution: HRRR Forecast}

HRRR cross-sections at multiple forecast hours show how fire weather conditions
evolve through the evening and overnight. Key concerns: evening wind shift,
overnight low-level jet loading, and whether any humidity recovery occurs.

\begin{figure}[htbp]
\centering
\begin{subfigure}[t]{0.32\textwidth}
\centering
\includegraphics[width=\textwidth]{figures/newalla_fhr00_fire_wx.png}
\caption{F00 fire weather}
\end{subfigure}
\hfill
\begin{subfigure}[t]{0.32\textwidth}
\centering
\includegraphics[width=\textwidth]{figures/newalla_fhr03_fire_wx.png}
\caption{F03 fire weather}
\end{subfigure}
\hfill
\begin{subfigure}[t]{0.32\textwidth}
\centering
\includegraphics[width=\textwidth]{figures/newalla_fhr06_fire_wx.png}
\caption{F06 fire weather}
\end{subfigure}

\begin{subfigure}[t]{0.32\textwidth}
\centering
\includegraphics[width=\textwidth]{figures/newalla_fhr09_fire_wx.png}
\caption{F09 fire weather}
\end{subfigure}
\hfill
\begin{subfigure}[t]{0.32\textwidth}
\centering
\includegraphics[width=\textwidth]{figures/newalla_fhr12_fire_wx.png}
\caption{F12 fire weather}
\end{subfigure}
\hfill
\begin{subfigure}[t]{0.32\textwidth}
\centering
\includegraphics[width=\textwidth]{figures/newalla_fhr15_fire_wx.png}
\caption{F15 fire weather}
\end{subfigure}
\caption{Newalla fire weather composite: temporal evolution (HRRR 21Z cycle)}
\label{fig:temporal_fire}
\end{figure}

\begin{figure}[htbp]
\centering
\begin{subfigure}[t]{0.32\textwidth}
\centering
\includegraphics[width=\textwidth]{figures/newalla_fhr00_wind.png}
\caption{F00 wind speed}
\end{subfigure}
\hfill
\begin{subfigure}[t]{0.32\textwidth}
\centering
\includegraphics[width=\textwidth]{figures/newalla_fhr03_wind.png}
\caption{F03 wind speed}
\end{subfigure}
\hfill
\begin{subfigure}[t]{0.32\textwidth}
\centering
\includegraphics[width=\textwidth]{figures/newalla_fhr06_wind.png}
\caption{F06 wind speed}
\end{subfigure}

\begin{subfigure}[t]{0.32\textwidth}
\centering
\includegraphics[width=\textwidth]{figures/newalla_fhr09_wind.png}
\caption{F09 wind speed}
\end{subfigure}
\hfill
\begin{subfigure}[t]{0.32\textwidth}
\centering
\includegraphics[width=\textwidth]{figures/newalla_fhr12_wind.png}
\caption{F12 wind speed}
\end{subfigure}
\hfill
\begin{subfigure}[t]{0.32\textwidth}
\centering
\includegraphics[width=\textwidth]{figures/newalla_fhr15_wind.png}
\caption{F15 wind speed}
\end{subfigure}
\caption{Newalla wind speed: temporal evolution}
\label{fig:temporal_wind}
\end{figure}

\begin{figure}[htbp]
\centering
\begin{subfigure}[t]{0.32\textwidth}
\centering
\includegraphics[width=\textwidth]{figures/newalla_fhr00_rh.png}
\caption{F00 RH}
\end{subfigure}
\hfill
\begin{subfigure}[t]{0.32\textwidth}
\centering
\includegraphics[width=\textwidth]{figures/newalla_fhr03_rh.png}
\caption{F03 RH}
\end{subfigure}
\hfill
\begin{subfigure}[t]{0.32\textwidth}
\centering
\includegraphics[width=\textwidth]{figures/newalla_fhr06_rh.png}
\caption{F06 RH}
\end{subfigure}

\begin{subfigure}[t]{0.32\textwidth}
\centering
\includegraphics[width=\textwidth]{figures/newalla_fhr09_rh.png}
\caption{F09 RH}
\end{subfigure}
\hfill
\begin{subfigure}[t]{0.32\textwidth}
\centering
\includegraphics[width=\textwidth]{figures/newalla_fhr12_rh.png}
\caption{F12 RH}
\end{subfigure}
\hfill
\begin{subfigure}[t]{0.32\textwidth}
\centering
\includegraphics[width=\textwidth]{figures/newalla_fhr15_rh.png}
\caption{F15 RH}
\end{subfigure}
\caption{Newalla relative humidity: temporal evolution}
\label{fig:temporal_rh}
\end{figure}

\begin{figure}[htbp]
\centering
\begin{subfigure}[t]{0.47\textwidth}
\centering
\includegraphics[width=\textwidth]{figures/okc_ns_fhr00_fire_wx.png}
\caption{OKC N--S F00 fire wx}
\end{subfigure}
\hfill
\begin{subfigure}[t]{0.47\textwidth}
\centering
\includegraphics[width=\textwidth]{figures/okc_ns_fhr00_wind.png}
\caption{OKC N--S F00 wind}
\end{subfigure}

\begin{subfigure}[t]{0.47\textwidth}
\centering
\includegraphics[width=\textwidth]{figures/okc_ns_fhr06_fire_wx.png}
\caption{OKC N--S F06 fire wx}
\end{subfigure}
\hfill
\begin{subfigure}[t]{0.47\textwidth}
\centering
\includegraphics[width=\textwidth]{figures/okc_ns_fhr06_wind.png}
\caption{OKC N--S F06 wind}
\end{subfigure}

\begin{subfigure}[t]{0.47\textwidth}
\centering
\includegraphics[width=\textwidth]{figures/okc_ns_fhr12_fire_wx.png}
\caption{OKC N--S F12 fire wx}
\end{subfigure}
\hfill
\begin{subfigure}[t]{0.47\textwidth}
\centering
\includegraphics[width=\textwidth]{figures/okc_ns_fhr12_wind.png}
\caption{OKC N--S F12 wind}
\end{subfigure}
\caption{OKC metro N--S temporal evolution}
\label{fig:temporal_okc}
\end{figure}

\section{GFS Extended-Range Outlook}

GFS 0.25$^\circ$ cross-sections provide a multi-day outlook for fire weather
conditions. Key question: when does the pattern break and moisture return?

\begin{figure}[htbp]
\centering
\begin{subfigure}[t]{0.47\textwidth}
\centering
\includegraphics[width=\textwidth]{figures/gfs_newalla_fhr06_fire_wx.png}
\caption{GFS F06 fire weather}
\end{subfigure}
\hfill
\begin{subfigure}[t]{0.47\textwidth}
\centering
\includegraphics[width=\textwidth]{figures/gfs_newalla_fhr24_fire_wx.png}
\caption{GFS F24 fire weather}
\end{subfigure}

\begin{subfigure}[t]{0.47\textwidth}
\centering
\includegraphics[width=\textwidth]{figures/gfs_newalla_fhr48_fire_wx.png}
\caption{GFS F48 fire weather}
\end{subfigure}
\hfill
\begin{subfigure}[t]{0.47\textwidth}
\centering
\includegraphics[width=\textwidth]{figures/gfs_newalla_fhr72_fire_wx.png}
\caption{GFS F72 fire weather}
\end{subfigure}
\caption{GFS extended fire weather outlook: Newalla transect}
\label{fig:gfs_fire}
\end{figure}

\begin{figure}[htbp]
\centering
\begin{subfigure}[t]{0.47\textwidth}
\centering
\includegraphics[width=\textwidth]{figures/gfs_newalla_fhr06_rh.png}
\caption{GFS F06 RH}
\end{subfigure}
\hfill
\begin{subfigure}[t]{0.47\textwidth}
\centering
\includegraphics[width=\textwidth]{figures/gfs_newalla_fhr24_rh.png}
\caption{GFS F24 RH}
\end{subfigure}

\begin{subfigure}[t]{0.47\textwidth}
\centering
\includegraphics[width=\textwidth]{figures/gfs_newalla_fhr48_rh.png}
\caption{GFS F48 RH}
\end{subfigure}
\hfill
\begin{subfigure}[t]{0.47\textwidth}
\centering
\includegraphics[width=\textwidth]{figures/gfs_newalla_fhr72_rh.png}
\caption{GFS F72 RH}
\end{subfigure}
\caption{GFS extended RH outlook: moisture recovery timeline}
\label{fig:gfs_rh}
\end{figure}

\begin{figure}[htbp]
\centering
\begin{subfigure}[t]{0.47\textwidth}
\centering
\includegraphics[width=\textwidth]{figures/gfs_oklahoma_fhr06_fire_wx.png}
\caption{GFS OK F06 fire wx}
\end{subfigure}
\hfill
\begin{subfigure}[t]{0.47\textwidth}
\centering
\includegraphics[width=\textwidth]{figures/gfs_oklahoma_fhr06_rh.png}
\caption{GFS OK F06 RH}
\end{subfigure}

\begin{subfigure}[t]{0.47\textwidth}
\centering
\includegraphics[width=\textwidth]{figures/gfs_oklahoma_fhr06_wind_speed.png}
\caption{GFS OK F06 wind}
\end{subfigure}
\hfill
\begin{subfigure}[t]{0.47\textwidth}
\centering
\includegraphics[width=\textwidth]{figures/gfs_oklahoma_fhr24_fire_wx.png}
\caption{GFS OK F24 fire wx}
\end{subfigure}
\caption{GFS broader Oklahoma outlook (1--4)}
\label{fig:gfs_ok_0}
\end{figure}

\begin{figure}[htbp]
\centering
\begin{subfigure}[t]{0.47\textwidth}
\centering
\includegraphics[width=\textwidth]{figures/gfs_oklahoma_fhr24_rh.png}
\caption{GFS OK F24 RH}
\end{subfigure}
\hfill
\begin{subfigure}[t]{0.47\textwidth}
\centering
\includegraphics[width=\textwidth]{figures/gfs_oklahoma_fhr24_wind_speed.png}
\caption{GFS OK F24 wind}
\end{subfigure}

\begin{subfigure}[t]{0.47\textwidth}
\centering
\includegraphics[width=\textwidth]{figures/gfs_oklahoma_fhr48_fire_wx.png}
\caption{GFS OK F48 fire wx}
\end{subfigure}
\hfill
\begin{subfigure}[t]{0.47\textwidth}
\centering
\includegraphics[width=\textwidth]{figures/gfs_oklahoma_fhr48_rh.png}
\caption{GFS OK F48 RH}
\end{subfigure}
\caption{GFS broader Oklahoma outlook (5--8)}
\label{fig:gfs_ok_4}
\end{figure}

\begin{figure}[htbp]
\centering
\begin{subfigure}[t]{0.47\textwidth}
\centering
\includegraphics[width=\textwidth]{figures/gfs_oklahoma_fhr48_wind_speed.png}
\caption{GFS OK F48 wind}
\end{subfigure}
\hfill
\begin{subfigure}[t]{0.47\textwidth}
\centering
\includegraphics[width=\textwidth]{figures/gfs_oklahoma_fhr72_fire_wx.png}
\caption{GFS OK F72 fire wx}
\end{subfigure}

\begin{subfigure}[t]{0.47\textwidth}
\centering
\includegraphics[width=\textwidth]{figures/gfs_oklahoma_fhr72_rh.png}
\caption{GFS OK F72 RH}
\end{subfigure}
\hfill
\begin{subfigure}[t]{0.47\textwidth}
\centering
\includegraphics[width=\textwidth]{figures/gfs_oklahoma_fhr72_wind_speed.png}
\caption{GFS OK F72 wind}
\end{subfigure}
\caption{GFS broader Oklahoma outlook (9--12)}
\label{fig:gfs_ok_8}
\end{figure}

\section{Forecast and Operational Outlook}


\subsection{Immediate Concerns (Next 6 Hours)}
\begin{itemize}[leftmargin=2em]
\item \textbf{Continued extreme fire spread:} SW winds 15--25 kt with gusts to 25 kt, RH 11--15\%
\item \textbf{RH still crashing:} KOKC dropped 20\%$\rightarrow$14\%, KPWA 14\%$\rightarrow$11\% in past 3 hours
\item \textbf{Dewpoints falling:} Down 3--6°F in 3 hours --- dry air advection continuing
\item \textbf{Wind-driven runs:} All active fires spreading NE toward populated areas
\item \textbf{New ignitions:} Extreme conditions support rapid ignition from any source (I-40 corridor, powerlines, equipment)
\end{itemize}

\subsection{Evening Transition (6--12 Hours)}
\begin{tcolorbox}[colback=red!10, colframe=red!80!black, title={\textbf{CRITICAL: WIND SHIFT WARNING}}]
FRONTAL WIND SHIFT: Cold front passage expected 03--09Z (9 PM -- 3 AM CST). Winds will shift from S/SSW 12--22 kt to N/NNW 12--25 kt (gusts to 25 kt). This converts ALL current south flanks into new headfires running south. Low-level wind shear: 40--45 kt from 220--240° at 2000 ft AGL noted in TAFs. This is the MOST DANGEROUS period for active fires.
\end{tcolorbox}
\begin{itemize}[leftmargin=2em]
\item \textbf{Wind shift timing:} Frontal passage 03--09Z tonight (from TAF analysis)
\item \textbf{Pre-frontal:} S/SSW 12--22 kt --- fires spreading N/NE
\item \textbf{Post-frontal:} N/NNW 12--25 kt gusting 25 kt --- fires reverse, all south flanks become headfires
\item \textbf{Low-level jet:} 40--45 kt at 2000 ft AGL from 220--240° (pre-frontal LLJ loading)
\item \textbf{Humidity recovery:} Uncertain --- cold front may not bring significant moisture
\item \textbf{Falling pressures:} 0.06--0.10 inHg drop in 3 hours confirms approaching front
\end{itemize}

\subsection{Extended (24--72 Hours)}
\begin{itemize}[leftmargin=2em]
\item \textbf{Pattern persistence:} GFS indicates continued dry/warm anomaly for 24--48h
\item \textbf{Cold front timing:} Full pattern change and moisture return timeline
  shown in GFS extended outlook section
\item \textbf{Rekindle risk:} Even if winds diminish, low RH can sustain smoldering
  fires that rekindle with next wind event
\end{itemize}

\subsection{Areas of Greatest Concern for New Fire Starts}
\begin{itemize}[leftmargin=2em]
\item \textbf{SE OKC corridor} (Newalla, Draper, Choctaw, Harrah) --- active fires + WUI
\item \textbf{Rural grasslands} E/SE of OKC --- continuous fine fuels, limited access
\item \textbf{I-40 corridor} --- human ignition sources + extreme spread conditions
\item \textbf{Any area downwind} of existing fires during wind shifts
\end{itemize}


\section{Data Sources and Methods}


\begin{itemize}[leftmargin=2em]
\item \textbf{HRRR:} 3-km horizontal resolution, 21Z cycle, FHR 00--15 (15-hour forecast window)
\item \textbf{GFS:} 0.25$^\circ$ ($\sim$28 km), 12Z cycle, FHR 6--384 (extended outlook to 16 days)
\item \textbf{Cross-sections:} 5 transects $\times$ 5+ products per transect = 25+ static images
\item \textbf{Temporal evolution:} 6 forecast hours $\times$ 3 products = 18 temporal frames + 4 animated GIFs
\item \textbf{GFS extended:} 2 transects $\times$ 4 FHRs $\times$ 3 products = 24 GFS images + 2 GIFs
\item \textbf{Surface obs:} METAR from 10 Oklahoma stations via aviationweather.gov
\item \textbf{Fire weather intel:} NWS alerts (OUN, TSA, AMA), SPC fire weather outlook, NIFC
\item \textbf{Street View:} 12 Google Street View images of fire-affected areas
\item \textbf{Fire risk scoring:} wxsection.com fire risk algorithm (0--100 scale)
\item \textbf{Isentropic ascent:} New analysis product showing RH + omega + geostrophic wind
\item \textbf{Report generation:} 7 parallel AI agents, compiled via wxsection.com ReportBuilder
\end{itemize}

\subsection{Products Generated}

\begin{verbatim}
Total figures: 85 PNG + 0 JPG + 0 GIF
\end{verbatim}


\vfill
\begin{center}
\small\textit{Generated by wxsection.com AI Atmospheric Research Agent}
\end{center}

\end{document}