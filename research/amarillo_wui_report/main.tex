\documentclass[11pt, letterpaper]{article}

% --- Encoding and fonts ---
\usepackage[utf8]{inputenc}
\usepackage[T1]{fontenc}
\usepackage{lmodern}

% --- Page layout ---
\usepackage[margin=1in]{geometry}
\usepackage{setspace}
\onehalfspacing

% --- Graphics and color ---
\usepackage{graphicx}
\usepackage[dvipsnames]{xcolor}
\usepackage{subcaption}

% --- Tables ---
\usepackage{booktabs}
\usepackage{array}
\usepackage{tabularx}
\usepackage{adjustbox}

% --- Headers and footers ---
\usepackage{fancyhdr}
\pagestyle{fancy}
\fancyhf{}
\lhead{\small wxsection.com}
\rhead{\small Amarillo Lake WUI Fire: Operational Intelligence Report}
\cfoot{\thepage}
\renewcommand{\headrulewidth}{0.4pt}

% --- Alert/callout boxes ---
\usepackage{tcolorbox}

% --- Hyperlinks ---
\usepackage[colorlinks=true, linkcolor=blue!70!black, urlcolor=blue!70!black, citecolor=blue!70!black]{hyperref}

% --- Misc ---
\usepackage{enumitem}
\usepackage{parskip}

% --- Title configuration ---
\newcommand{\reporttypecolor}{red!70!black}
\newcommand{\reporttypelabel}{FORECAST}

\title{\textcolor{\reporttypecolor}{\large\textsc{\reporttypelabel}\\[0.3em]\LARGE\textbf{Amarillo Lake WUI Fire: Operational Intelligence Report}}}
\author{wxsection.com AI Agent Swarm}
\date{February 9, 2026 --- 21:00 UTC (3:00 PM CST)}

\begin{document}

\maketitle
\thispagestyle{fancy}

\begin{abstract}
A wildland-urban interface fire originating near Amarillo Lake has jumped West Amarillo Boulevard and is threatening residential neighborhoods in northwest Amarillo, TX. North Heights High School has been evacuated with students bused to other campuses. Amarillo PD has blocked traffic on North Hughes Street. Surface winds are sustained at 19 kt gusting to 30-33 kt from the southwest (230 deg) with relative humidity at 10\% and temperatures 76 deg F --- 25 deg F above normal for February. The SPC has this area under a CRITICAL fire weather outlook. Three NWS Red Flag Warnings are active covering the Texas Panhandle. HRRR cross-section analysis shows a deep dry column with no moisture at any level and a 700 hPa low-level jet providing the momentum source for surface gusts. The nocturnal low-level jet will load overnight with no RH recovery, creating continued risk. This report was generated by 3 parallel AI agents using HRRR 3-km, GFS 0.25 deg, Google Street View, NWS/SPC alerts, and METAR observations.
\end{abstract}

\section{Situation Overview}

\begin{tcolorbox}[colback=red!10, colframe=red!80!black, title={\textbf{WUI FIRE ALERT}}]
WILDLAND-URBAN INTERFACE FIRE: Fire has jumped West Amarillo Boulevard and is spreading NE into residential neighborhoods. North Heights High School evacuated. North Hughes Street blocked by APD. Active fire under SPC CRITICAL conditions with 10\% RH and 30+ kt gusts.
\end{tcolorbox}

\subsection{Fire Location and Status}
\begin{itemize}[leftmargin=2em]
\item \textbf{Origin:} Amarillo Lake, near North Hughes St / NW 5th Ave (35.21$^\circ$N, 101.845$^\circ$W)
\item \textbf{Jump point:} Fire has crossed West Amarillo Boulevard heading NE
\item \textbf{Wind-driven spread:} SW at 230$^\circ$, 19~kt sustained, gusts 30--33~kt $\rightarrow$ fire spreading \textbf{NE into residential areas}
\item \textbf{Evacuations:} North Heights High School --- students bused to other campuses
\item \textbf{Road closures:} N Hughes St at W Amarillo Blvd blocked by APD motorcycle officers
\item \textbf{Source:} MyHighPlains.com confirmed AFD response at 1:40 PM CST
\end{itemize}

\subsection{Active NWS Alerts}
\begin{itemize}[leftmargin=2em]
\item \textbf{Red Flag Warning} (NWS Amarillo) --- TX/OK Panhandles including \textbf{Potter County} (Amarillo). SW 20--25 mph, gusts \textbf{40 mph}. RH as low as \textbf{7\%}. Temps 70s--80s$^\circ$F. ERC 70th--89th percentile. Fire Environment 6/10. Until 7 PM CST.
\item \textbf{Red Flag Warning} (NWS Lubbock) --- SW TX Panhandle. SW 20--25 mph, gusts 35 mph. RH 9\%. Until 7 PM CST.
\item \textbf{SPC Day 1 CRITICAL} fire weather outlook covers the entire TX Panhandle.
\end{itemize}

\section{Current Surface Conditions}

Latest METAR observation from Amarillo International Airport (KAMA), 8 km SE of the fire:

\begin{table}[htbp]
\centering
\small
\begin{tabular}{lrr}
\toprule
\textbf{Parameter} & \textbf{Value} & \textbf{Assessment} \\
\midrule
Wind & 230 deg / 19 kt G30 kt & CRITICAL (SW, driving fire NE) \\
Peak Gust & 33 kt (38 mph) at 2030Z & Exceeds RFW threshold \\
Temperature & 24.4C (76F) & 25F above Feb normal \\
Dewpoint & -8.9C (16F) & Extreme dryness \\
RH (est.) & \textasciitilde{}10\% & CRITICAL (below 13\% RFW) \\
Visibility & 10+ SM & Clear --- no precip possible \\
Altimeter & 29.99 inHg & Falling (trough approach) \\
SLP & 1012.6 mb & Down from 1016 earlier \\
Sky & CLR & Full solar heating \\
\bottomrule
\end{tabular}
\caption{KAMA METAR at 2053 UTC (2:53 PM CST), February 9, 2026}
\label{tab:metar}
\end{table}

\subsection{Conditions Trend}

\begin{itemize}[leftmargin=2em]
\item \textbf{Pressure falling:} SLP dropped from 1016$\rightarrow$1013 mb in 3 hours --- deepening trough driving wind acceleration
\item \textbf{Wind steady:} Sustained 19--22~kt with gusts 30--35~kt since 19Z. No relaxation expected until after 01Z (7 PM CST).
\item \textbf{RH bottoming:} 10--11\% --- at or near diurnal minimum. Will recover slightly after sunset but \textbf{stay below 25\% overnight} per HRRR.
\item \textbf{Wind shift approaching:} Clines Corners NM (upwind) already showing WNW 300$^\circ$. When this cold front reaches Amarillo ($\sim$06Z / midnight), winds shift to N/NW --- \textbf{fire flanks become new head}.
\end{itemize}

\section{HRRR Cross-Section Analysis}

Cross-sections from the HRRR 3-km model (latest available cycle) through the Amarillo fire zone. Two primary transects: (1) W--E at 35.21$^\circ$N from 102.5$^\circ$W to 101.2$^\circ$W, covering the upwind terrain through Amarillo to downwind areas; (2) SW--NE from 35.15$^\circ$N/102$^\circ$W to 35.35$^\circ$N/101.6$^\circ$W, aligned with the wind-driven fire spread direction.

\begin{figure}[htbp]
\centering
\includegraphics[width=0.95\textwidth]{figures/ama_firewx_swne.png}
\caption{Fire Weather Composite, SW-NE through fire zone aligned with wind direction (230 deg). Cross-hatching shows extreme low RH extending through the full boundary layer. This is the fire's spread corridor.}
\label{fig:ama_firewx_swne}
\end{figure}

\begin{figure}[htbp]
\centering
\includegraphics[width=0.95\textwidth]{figures/ama_wind_swne.png}
\caption{Wind Speed, SW-NE through fire zone. Shows the low-level momentum driving fire spread from the origin toward residential neighborhoods NE of Amarillo Blvd.}
\label{fig:ama_wind_swne}
\end{figure}

\begin{figure}[htbp]
\centering
\includegraphics[width=0.95\textwidth]{figures/ama_rh_swne.png}
\caption{Relative Humidity, SW-NE through fire zone. The entire troposphere below 500 hPa is desiccated --- uniformly below 20\% RH with no moisture source at any level.}
\label{fig:ama_rh_swne}
\end{figure}

\begin{figure}[htbp]
\centering
\includegraphics[width=0.95\textwidth]{figures/ama_firewx_we.png}
\caption{Fire Weather Composite, W-E through Amarillo at 35.21N. Deep red/cross-hatched from surface through 500 hPa across the entire transect. Terrain slopes downward from west, enhancing downslope drying and wind acceleration.}
\label{fig:ama_firewx_we}
\end{figure}

\begin{figure}[htbp]
\centering
\includegraphics[width=0.95\textwidth]{figures/ama_wind_we.png}
\caption{Wind Speed, W-E through Amarillo. Surface winds 15-25 kt with a 700 hPa low-level jet at 30-40 kt providing the momentum source. The jet mixes down to the surface during afternoon heating.}
\label{fig:ama_wind_we}
\end{figure}

\begin{figure}[htbp]
\centering
\includegraphics[width=0.95\textwidth]{figures/ama_rh_we.png}
\caption{Relative Humidity, W-E through Amarillo. Bone dry through the entire column --- no level shows RH above 30\%. Zero chance of precipitation or meaningful humidity recovery.}
\label{fig:ama_rh_we}
\end{figure}

\begin{figure}[htbp]
\centering
\includegraphics[width=0.95\textwidth]{figures/ama_temp_we.png}
\caption{Temperature, W-E through Amarillo. Surface temperatures 18-24C (65-76F). 0C isotherm near 630 hPa. Warm for February --- enhancing fire behavior.}
\label{fig:ama_temp_we}
\end{figure}

\subsection{Isentropic Analysis (Downslope Dynamics)}

Isentropic cross-sections reveal the downslope wind mechanics driving this fire. The Y-axis shows potential temperature ($\theta$) surfaces with pressure contoured as overlays. Compressed isentropes indicate acceleration; widely-spaced isentropes indicate stability.

\begin{figure}[htbp]
\centering
\includegraphics[width=0.95\textwidth]{figures/ama_isen_wind_we.png}
\caption{Isentropic Wind Speed. Strong wind maximum at 310-320K theta levels. Compressed isentropes near the terrain break show classic downslope acceleration --- momentum transfers from the mountain crest onto the High Plains.}
\end{figure}

\begin{figure}[htbp]
\centering
\includegraphics[width=0.95\textwidth]{figures/ama_isen_rh_we.png}
\caption{Isentropic RH. Dry air on all theta surfaces from 300K to 320K. The low-level isentropic surfaces show the mechanism: dry air from above the terrain crest descends adiabatically, warming and drying as it flows onto the plains.}
\end{figure}

\subsection{N-S Transect (Wind Shift Approach)}

\begin{figure}[htbp]
\centering
\includegraphics[width=0.95\textwidth]{figures/ama_wind_ns.png}
\caption{Wind Speed, N-S through Amarillo at 101.845W (35.5N to 34.9N). Shows the wind field structure that the approaching cold front will modify. When the front arrives, northerly winds will replace the current southwesterly flow, causing the fire's southern flank to become the new head.}
\end{figure}

\section{Temporal Evolution}

HRRR forecast sequence showing how fire weather conditions evolve at Amarillo over the next 12 hours.

\begin{figure}[htbp]
\centering
\begin{subfigure}[t]{0.47\textwidth}
\centering
\includegraphics[width=\textwidth]{figures/ama_firewx_fhr00.png}
\caption{FHR 00 (+0h)}
\end{subfigure}
\hfill
\begin{subfigure}[t]{0.47\textwidth}
\centering
\includegraphics[width=\textwidth]{figures/ama_firewx_fhr03.png}
\caption{FHR 03 (+3h)}
\end{subfigure}

\begin{subfigure}[t]{0.47\textwidth}
\centering
\includegraphics[width=\textwidth]{figures/ama_firewx_fhr06.png}
\caption{FHR 06 (+6h)}
\end{subfigure}
\hfill
\begin{subfigure}[t]{0.47\textwidth}
\centering
\includegraphics[width=\textwidth]{figures/ama_firewx_fhr09.png}
\caption{FHR 09 (+9h)}
\end{subfigure}
\caption{Fire weather composite evolution at Amarillo (W-E transect). Conditions remain critical through FHR 06 (evening), then the nocturnal LLJ develops. No meaningful RH recovery overnight.}
\label{fig:temporal-grid}
\end{figure}

\begin{figure}[htbp]
\centering
\includegraphics[width=0.95\textwidth]{figures/ama_firewx_fhr12.png}
\caption{FHR 12: Overnight. LLJ loaded at 700-800 hPa with momentum ready to mix down at sunrise.}
\label{fig:temporal-fhr12}
\end{figure}

\section{GFS Extended Forecast}

The GFS 0.25$^\circ$ model provides multi-day context beyond the HRRR 18-hour window, showing when genuine relief arrives.

\begin{figure}[htbp]
\centering
\includegraphics[width=0.85\textwidth]{figures/ama_gfs_firewx_fhr06.png}
\caption{GFS Fire Wx, FHR 06 (valid 18Z Feb 9)}
\end{figure}

\begin{figure}[htbp]
\centering
\includegraphics[width=0.85\textwidth]{figures/ama_gfs_firewx_fhr24.png}
\caption{GFS Fire Wx, FHR 24 (valid 12Z Feb 10)}
\end{figure}

\begin{figure}[htbp]
\centering
\includegraphics[width=0.85\textwidth]{figures/ama_gfs_firewx_fhr48.png}
\caption{GFS Fire Wx, FHR 48 (valid 12Z Feb 11)}
\end{figure}

\begin{figure}[htbp]
\centering
\includegraphics[width=0.85\textwidth]{figures/ama_gfs_rh_fhr24.png}
\caption{GFS RH, FHR 24 (valid 12Z Feb 10)}
\end{figure}

\begin{figure}[htbp]
\centering
\includegraphics[width=0.85\textwidth]{figures/ama_gfs_wind_fhr24.png}
\caption{GFS Wind, FHR 24 (valid 12Z Feb 10)}
\end{figure}

\section{Street View: WUI Fuel and Structure Assessment}

Ground-level imagery from Google Street View of the specific neighborhoods affected by the Amarillo Lake fire. These images show the wildland-urban interface conditions, fuel types, structure density, and vulnerability of the areas the fire is spreading into.

\begin{figure}[htbp]
\centering
\begin{subfigure}[t]{0.47\textwidth}
\centering
\includegraphics[width=\textwidth]{figures/01_hughes_at_amarillo_blvd_N.jpg}
\caption{Hughes At Amarillo Blvd N}
\end{subfigure}
\hfill
\begin{subfigure}[t]{0.47\textwidth}
\centering
\includegraphics[width=\textwidth]{figures/02_hughes_looking_south_fire.jpg}
\caption{Hughes Looking South Fire}
\end{subfigure}
\caption{WUI assessment: Hughes At Amarillo Blvd N / Hughes Looking South Fire}
\end{figure}

\begin{figure}[htbp]
\centering
\begin{subfigure}[t]{0.47\textwidth}
\centering
\includegraphics[width=\textwidth]{figures/03_amarillo_blvd_looking_west.jpg}
\caption{Amarillo Blvd Looking West}
\end{subfigure}
\hfill
\begin{subfigure}[t]{0.47\textwidth}
\centering
\includegraphics[width=\textwidth]{figures/04_amarillo_blvd_looking_east.jpg}
\caption{Amarillo Blvd Looking East}
\end{subfigure}
\caption{WUI assessment: Amarillo Blvd Looking West / Amarillo Blvd Looking East}
\end{figure}

\begin{figure}[htbp]
\centering
\begin{subfigure}[t]{0.47\textwidth}
\centering
\includegraphics[width=\textwidth]{figures/05_north_heights_hs_area.jpg}
\caption{North Heights Hs Area}
\end{subfigure}
\hfill
\begin{subfigure}[t]{0.47\textwidth}
\centering
\includegraphics[width=\textwidth]{figures/06_residential_NE_looking_back.jpg}
\caption{Residential Ne Looking Back}
\end{subfigure}
\caption{WUI assessment: North Heights Hs Area / Residential Ne Looking Back}
\end{figure}

\begin{figure}[htbp]
\centering
\begin{subfigure}[t]{0.47\textwidth}
\centering
\includegraphics[width=\textwidth]{figures/07_amarillo_lake_south.jpg}
\caption{Amarillo Lake South}
\end{subfigure}
\hfill
\begin{subfigure}[t]{0.47\textwidth}
\centering
\includegraphics[width=\textwidth]{figures/08_nw_5th_ave_near_fire.jpg}
\caption{Nw 5Th Ave Near Fire}
\end{subfigure}
\caption{WUI assessment: Amarillo Lake South / Nw 5Th Ave Near Fire}
\end{figure}

\begin{figure}[htbp]
\centering
\begin{subfigure}[t]{0.47\textwidth}
\centering
\includegraphics[width=\textwidth]{figures/09_industrial_west_of_hughes.jpg}
\caption{Industrial West Of Hughes}
\end{subfigure}
\hfill
\begin{subfigure}[t]{0.47\textwidth}
\centering
\includegraphics[width=\textwidth]{figures/10_residential_east_fire_path.jpg}
\caption{Residential East Fire Path}
\end{subfigure}
\caption{WUI assessment: Industrial West Of Hughes / Residential East Fire Path}
\end{figure}

\section{Forecast and Operational Outlook}

\begin{tcolorbox}[colback=orange!10, colframe=orange!80!black, title={\textbf{CRITICAL WIND SHIFT APPROACHING}}]
WIND SHIFT WARNING: A cold front is approaching from the west. Clines Corners NM already shows WNW winds (300 deg) while Amarillo has SW (230 deg). When the front reaches Amarillo around midnight (06Z), winds will shift sharply to northerly. The fire's current southern flank will become the new head fire, potentially changing the threat axis from NE to SOUTH into previously unburned areas.
\end{tcolorbox}

\subsection{Timeline for Amarillo}
\begin{description}[leftmargin=2em]
\item[NOW -- 01Z (7 PM CST):] \textbf{Peak danger continues.} SW 19G30+~kt, 10\% RH. Fire will continue spreading NE. Any structures in the NE path are at immediate risk. Aggressive suppression critical before wind shift.
\item[01Z -- 06Z (7 PM -- midnight):] Winds remain elevated pre-frontally. Gradual temperature drop begins. \textbf{Window for establishing fire breaks on the southern flank before wind shift.}
\item[$\sim$06Z (midnight):] \textbf{Cold front passage.} Sharp wind shift to N/NW. Any uncontained southern perimeter becomes the new head. This is the most dangerous moment for fire behavior --- erratic gusts and rapid direction change.
\item[06Z -- 13Z (overnight):] Post-frontal northerly winds. Temperature drops. \textbf{But NO meaningful RH recovery} --- boundary layer stays 10--25\% even overnight. The nocturnal LLJ loads momentum at 700~hPa.
\item[After 16Z Feb 10 (10 AM CST):] Next-morning surge risk as loaded LLJ mixes down. If fire not contained, another extreme window.
\item[Mid-week:] Rain/storms return. Genuine relief.
\end{description}

\subsection{WUI Threat Assessment}
\begin{itemize}[leftmargin=2em]
\item \textbf{Primary threat axis (current):} SW$\rightarrow$NE through residential neighborhoods N/NE of W Amarillo Blvd
\item \textbf{Secondary threat axis (post-frontal):} N$\rightarrow$S as wind shift drives fire southward
\item \textbf{Urban fuels:} Dormant lawns, bare deciduous trees, vacant lots with weedy growth, wooden fencing
\item \textbf{Structure types at risk:} Single-family residential, commercial buildings along Amarillo Blvd corridor
\item \textbf{Critical infrastructure:} North Heights High School (evacuated), Amarillo Lake, N Hughes St corridor
\end{itemize}

\section{Data Sources}

This report was generated by the wxsection.com AI Agent Research Platform using 3 parallel agents:
\begin{enumerate}[leftmargin=2em]
\item \textbf{Cross-Section Agent:} 15+ HRRR cross-sections (fire wx, wind, RH, temperature, isentropic, temporal), GIFs
\item \textbf{Street View + Web Intelligence Agent:} Google Street View of WUI impact zones, latest news
\item \textbf{GFS + Fire Risk Agent:} Extended-range GFS cross-sections, fire risk scores, NWS/SPC data
\end{enumerate}

\textbf{Models:} HRRR 3-km (latest cycle), GFS 0.25$^\circ$ (20260209 12Z). \\
\textbf{Observations:} KAMA METAR (2053Z). \\
\textbf{Alerts:} NWS api.weather.gov, SPC NOAA MapServer.

\vfill
\begin{center}
\small\textit{Generated by wxsection.com AI Atmospheric Research Agent}
\end{center}

\end{document}